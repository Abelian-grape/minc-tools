\documentstyle{report}

\title{Programmer's Reference for the \\
BIC Volume IO Library}

\author{David MacDonald}

\newcommand{\name}[1]{{\bf\tt #1}}

\newcommand{\desc}[1]{\begin{itemize}
\item[] #1
\end{itemize}}

\newcommand{\vol}{BIC Volume IO Library}

\begin{document}

\maketitle

\newpage

\tableofcontents

\newpage

\chapter{Introduction}

This document describes a library of routines available at the
McConnell Brain Imaging Centre (BIC) at the Montreal Neurological Institute.
It was developed as part of a medical imaging software testbed by
David MacDonald, with source code and considerable input from Peter
Neelin, Louis Collins, and others at the Centre.
The library, which is referred to as the \vol, comprises a set of functions
for reading and writing
volumes of medical imaging data, as well as some general support
routines useful to many programming tasks.  All images at the Brain
Imaging Centre are stored on disk in a format called {\bf MINC}, which
stands for {\bf Medical Image Net CDF}.  More information 
is available in related documentation specifically concerning this
format and
the MINC library of routines used to read and write MINC files.  The
\vol\  is built on top of the MINC library to
provide easy access to MINC files for most general operations, without
having to learn too much of
the details of the MINC library, which is a comprehensive system for handling
every conceivable case.  The \vol\  provides a structure for internal
storage of volumes and routines to access and modify volumes.  In
addition, it provides routines to manipulate tag points and
transformations and to perform input and output on these objects in
the standardized formats of the Brain Imaging Centre.

This document describes where to find the \vol,
what functionality is provided, and
how to integrate it into a user's programs.
The library is written in C source code, and is designed
to be linked with C source.  It makes calls to two other libraries:
\name{MINC} (the BIC file format library)
and \name{netcdf} (a portable file format manager).

\chapter{Compiling and Linking}

The library for linking is \name{libvolume\_io.a}, which is in
\name{/usr/local/lib}, and the related include files are in the directory,
\name{/usr/local/include} under \name{volume\_io}.  Both directories
are usually in the compiler search path.
Source files making calls to Volume IO functions must have the
following line at the top, to include the relevant type definitions and
prototypes:

{\bf\begin{verbatim}
        #include  <volume_io/volume_io.h>
\end{verbatim}}

In order to link with \vol, the relevant libraries must be
specified:
{\bf\begin{verbatim}
        cc test_volume_io.o  -o test_volume_io \
           -lvolume_io -lminc -lnetcdf -lm -lsun -lmalloc
\end{verbatim}}

The two libraries, \name{minc} and \name{netcdf}
are usually in \name{/usr/local/lib}, which is
automatically in the search path.  The \name{-lm} option includes the
math library, which is sometimes called by the \vol.  The
\name{-lsun} library provides network support useful to all programs
\footnote{The \name{-lsun} option is specific to the current version
of the operating system on the SGI's at the BIC.  When the system is
upgraded this option will become obsolete and unnecessary.}.
The \name{-lmalloc} library provides faster and more robust memory
allocation than the default.

\chapter{Style}

In order to use this library, one must become accustomed to certain
subtleties of style, some of which are peculiar to the author of this
software and some of which have evolved in ways dependent on history
of the software.  Therefore, it is important to note the following
style-related issues, for which no apologies are forthcoming.

\section{Global Variables}

The philosophy that global variables are generally a bad idea has been
adopted.  In many cases, using global variables within functions
hides the behaviour of a function, and therefore,
it is preferable to specify all relevant information to
a function in the argument list, rather than relying on any global
variables.  Secondly, modification of a global variable by a given function
precludes the use of this function in a multi-threaded environment
(two processes may attempt to modify the global variable at the same
time).  As a result of adopting the anti-global variable philosophy, the
user of the \vol\  often must specify
many arguments to a function.  This seems a small price to pay, in
return for the ability to see all factors controlling
the behaviour of each function in the argument list.

\section{Identifier Names}

The implementer of this library has an affinity to typing large
amounts of text, and generally dislikes abbreviations.  As a result,
the function and variable names occurring in the \vol\  may
be quite verbose, almost to the point of being complete grammatically
correct sentences.  In general, one will never see functions with names
such as \name{rem\_garb()}.  Instead, functions are more likely to have
names such as \name{remove\_garbage()}, and, in many cases, even
\name{take\_garbage\_to\_curb()}.  As a result, the user of the
\vol\ will have to type longer-than-usual function and variable
names, but hopefully will have a clearer insight into the associated
functionality.

\section{Structure Arguments}

There are many objects defined throughout the library as structures of
varying and possibly large sizes.  Due to the inefficiency of passing
structures by value (copying the whole structure to a function),
structures are generally passed to functions by reference.  Because of
this, there is always a potential to pass an invalid pointer to a
function which expects a pointer to a structure.  The following examples
illustrate the right and wrong ways to use such a function.  Given a
structure and a library function which initializes it,
\begin{verbatim}
typedef  struct
{
    int   a, b, c, d;
} big_struct;

void    initialize( struct  big_struct   *s )
{
    s->a = 1;   s->b = 2;   s->c = 3;    s->d = 4;
}
\end{verbatim}
the incorrect method of using this function is:
\begin{verbatim}
int  main()
{
    big_struct   *s;

    initialize( s );   /* WRONG */
}
\end{verbatim}
because the variable \name{s} is an uninitialized pointer.  The
correct method is to define a structure variable, not a pointer to a
structure, and pass a pointer to the structure:
\begin{verbatim}
int  main()
{
    big_struct   s;

    initialize( &s );
}
\end{verbatim}
Alternately, the incorrect example above could have been corrected by
allocating the \name{s} pointer before calling \name{initialize}.

\chapter{Types and Macros}

There are several types and macros defined for use with the \vol.
All function declarations in the library are preceded with either
the word \name{public} or \name{private}, which indicates whether the
function is accessible from outside the file in which it resides.
Users of the library will only be interested in those functions
preceded by \name{public}.  They are defined as follows:
{\bf\begin{verbatim}
        #define  public
        #define  private   static
\end{verbatim}}

A type for logical values is defined:
{\bf\begin{verbatim}
        typedef  int    BOOLEAN
        #define  FALSE  0
        #define  TRUE   1
        #define  OFF    FALSE
        #define  ON     TRUE
\end{verbatim}}

Other useful types defined include:
{\bf\begin{verbatim}
        typedef  double             Real;
        #define  MAX_STRING_LENGTH  200
        typedef  char               STRING[MAX_STRING_LENGTH+1];
        typedef  enum 
             { OK, ERROR, INTERNAL_ERROR, END_OF_FILE, QUIT }
                                    Status;
        typedef  char               Smallest_int;
\end{verbatim}}

Some macros useful for general programming include:

{\bf\begin{verbatim}
N_DIMENSIONS
\end{verbatim}}

\desc{A constant equal to 3, the number of dimensions in the real world.}

{\bf\begin{verbatim}
X
\end{verbatim}}

\desc{A constant equal to 0, used as an index into various XYZ structures.}

{\bf\begin{verbatim}
Y
\end{verbatim}}

\desc{A constant equal to 1, used as an index into various XYZ structures.}

{\bf\begin{verbatim}
Z
\end{verbatim}}

\desc{A constant equal to 2, used as an index into various XYZ structures.}

{\bf\begin{verbatim}
SIZEOF_STATIC_ARRAY( array )
\end{verbatim}}

\desc{returns the number of elements in a statically allocated array,
      for example,}

{\bf\begin{verbatim}
        {
            int  size;
            int  array[] = { 1, 2, 3 };

            size = SIZEOF_STATIC_ARRAY( array );     /* == 3 */
        }
\end{verbatim}}

{\bf\begin{verbatim}
ROUND( x )
\end{verbatim}}

\desc{returns the nearest integer to the \name{x}.  If
halfway in between two integers, returns the higher of the two.}

{\bf\begin{verbatim}
IS_INT( x )
\end{verbatim}}

\desc{returns \name{TRUE} if the real argument is exactly an integer.}

{\bf\begin{verbatim}
FRACTION( x )
\end{verbatim}}

\desc{returns the fractional part of the argument.}

{\bf\begin{verbatim}
FLOOR( x )
\end{verbatim}}

\desc{returns the largest integer less than or equal to the argument.}

{\bf\begin{verbatim}
CEILING( x )
\end{verbatim}}

\desc{returns the smallest integer greater than or equal to the argument.}

{\bf\begin{verbatim}
ABS( x )
\end{verbatim}}

\desc{returns the absolute value of an integer or real.}

{\bf\begin{verbatim}
MAX( x, y )
\end{verbatim}}

\desc{returns the maximum of two integers or reals.}

{\bf\begin{verbatim}
MAX3( x, y, z )
\end{verbatim}}

\desc{returns the maximum of three integers or
reals.}

{\bf\begin{verbatim}
MIN( x, y )
\end{verbatim}}

\desc{returns the minimum of two integers or reals.}

{\bf\begin{verbatim}
MIN3( x, y, z )
\end{verbatim}}

\desc{returns the minimum of three integers or reals.}

{\bf\begin{verbatim}
INTERPOLATE( alpha, a, b )
\end{verbatim}}

\desc{returns the interpolation between
\name{a} and \name{b}, where \name{alpha} is in the range zero to one.}

{\bf\begin{verbatim}
PI
\end{verbatim}}

\desc{returns the value of $\pi$.}

{\bf\begin{verbatim}
DEG_TO_RAD
\end{verbatim}}

\desc{returns the number of radians per degrees,
used to multiply an angle in degrees to result in radians.}

{\bf\begin{verbatim}
RAD_TO_DEG
\end{verbatim}}

\desc{returns the number of degrees per radian,
used to multiply an angle in radians to result in degrees.}

{\bf\begin{verbatim}
IJ( i, j, nj )
\end{verbatim}}

\desc{converts the indices $\left[ i, j \right]$
of a 2-D \name{ni} by \name{nj} array into a single index, based on
row-major order.}

{\bf\begin{verbatim}
IJK( i, j, k, nj, nk )
\end{verbatim}}

\desc{converts the indices
$\left[ i , j , k \right]$ of a 3-D \name{ni} by \name{nj} by
\name{nk} array into a single index, based on row-major order.}

{\bf\begin{verbatim}
for_less( i, start, end )
\end{verbatim}}

\desc{performs a for loop where \name{i}
starts at \name{start} and increments until it is greater than or
equal to \name{end}.
Equivalent to \name{for( i = start;  i < end; ++i )}.}

{\bf\begin{verbatim}
for_inclusive( i, start, end )
\end{verbatim}}

\desc{performs a for loop where \name{i}
starts at \name{start} and increments until it is greater than \name{end}.
Equivalent to \name{for( i = start;  i <= end; ++i )}.}

{\bf\begin{verbatim}
GLUE(x,y)
\end{verbatim}}

\desc{Special C source macro to stick two different
identifiers together, i.e., \name{GLUE(colour,\_name)} results in
\name{colour\_name}.}

{\bf\begin{verbatim}
GLUE3(x,y,z)
\end{verbatim}}

\desc{Special C source macro to stick three different
identifiers together, i.e., \name{GLUE(a,b,c)} results in \name{abc}.}

\chapter{Programming Utilities}

A set of functions which are useful for general purpose, portable
programming is provided.  These include such basic areas as strings,
time, file IO, etc.  Many parts of the \vol\  refer
to the programming utilities, and it is advisable that users of the
\vol\  try to use these functions whenever convenient.
The following is a list of all the programming utilities, grouped into
related areas, and ordered alphabetically.

\section{General File IO}

Although one may use the standard UNIX file interface (\name{fprintf},
for example), the \vol\  contains a set of routines for
doing all file operations, with the potential to be portable to
other operating systems.  There are some other minor advantages,
including the automatic expansion of \name{\~\ } and environment
variables.  Also, automatic decompression of compressed files is
provided.

{\bf\begin{verbatim}
public  void  expand_filename(
    char  filename[],
    char  expanded_filename[] )
\end{verbatim}}

\desc{Searches the argument \name{filename} for certain patterns, and
expands them appropriately, storing the resulting expanded filename in
the second argument, \name{expanded\_filename}, which must be a
different string pointer.  Any sequence of
characters starting with a \name{\~\ } up to but not including a slash,
\name{/}, will be changed to the specified users home directory.
Any occurrence of a dollar sign,
\name{\$}, will result in the following text, up to the next slash,
being expanded to the environment variable specified by the text.  This
expansion can be avoided by placing a backslash, \name{$\backslash$},
before any \name{\~\ } or \name{\$}.  In all the following functions which
take filenames, automatic expansion of filenames is performed.}

{\bf\begin{verbatim}
public  BOOLEAN  filename_extension_matches(
    char   filename[],
    char   extension[] )
\end{verbatim}}

\desc{Returns \name{TRUE} if the file ends in the given extension, preceded
by a period. e.g.,
\name{filename\_extension\_matches( "volume.mnc" , "mnc" )} returns \name{TRUE}.
Correctly handles compressed files, e.g.,
\name{filename\_extension\_matches( "volume.mnc.Z" , "mnc" )} returns \name{TRUE}.}

{\bf\begin{verbatim}
public  void  remove_directories_from_filename(
    char  filename[],
    char  filename_no_directories[] )
\end{verbatim}}

\desc{Strips the directories from the filename, placing the result in
the second argument, \name{filename\_no\_directories}.}

{\bf\begin{verbatim}
public  void  extract_directory(
    char    filename[],
    char    directory[] )
\end{verbatim}}

\desc{Extracts the directory from the filename, placing the result in
the second argument, \name{directory}.}

{\bf\begin{verbatim}
public  void  get_absolute_filename(
    char    filename[],
    char    directory[],
    char    abs_filename[] )
\end{verbatim}}

\desc{Given a filename and a current directory, constructs an
absolute filename (one starting with a slash, \name{/}.}

{\bf\begin{verbatim}
public  BOOLEAN  file_exists(
    char        filename[] )
\end{verbatim}}

\desc{Returns \name{TRUE} if the file exists, \name{FALSE} otherwise.}

{\bf\begin{verbatim}
public  void  remove_file(
    char  filename[] )
\end{verbatim}}

\desc{Removes the specified file.}

{\bf\begin{verbatim}
public  void  unlink_file(
    char        filename[] )
\end{verbatim}}

\desc{Unlinks the specified file, which results in the file being
removed only after all other references to it are closed.}

{\bf\begin{verbatim}
public  Status  open_file(
    char               filename[],
    IO_types           io_type,
    File_formats       file_format,
    FILE               **file )
public  Status  open_file_with_default_suffix(
    char               filename[],
    char               default_suffix[],
    IO_types           io_type,
    File_formats       file_format,
    FILE               **file )
\end{verbatim}}

\desc{The function \name{open\_file()} opens the specified file, where
\name{io\_type} must be one of
\name{WRITE\_FILE} or \name{READ\_FILE}, and
\name{file\_format} must be one of \name{ASCII\_FORMAT} or
\name{BINARY\_FORMAT}.  If successful, the file pointer is passed back
in the last argument and a status of \name{OK} is returned.
Otherwise, a null pointer is passed back, and a status of \name{ERROR}
is returned.  Filename expansion is automatically performed.  The
second function, \name{open\_file\_with\_default\_suffix()}, performs the
same task as \name{open\_file} with the addition that it automatically
adds the specified suffix extension, if needed.  On input, if the
specified file does not exist and the file does not have an extension,
then it looks for the specified file with the default extension.}

{\bf\begin{verbatim}
public  Status  close_file(
    FILE     *file )
\end{verbatim}}

\desc{Closes the file, returning \name{OK} if successful.}

{\bf\begin{verbatim}
public  Status  set_file_position(
    FILE     *file,
    long     byte_position )
\end{verbatim}}

\desc{Seeks to the specified position in the file, where 0 corresponds
to the beginning of the file, returning \name{OK} if successful.}

{\bf\begin{verbatim}
public  Status  flush_file(
    FILE     *file )
\end{verbatim}}

\desc{Flushes the file buffer of any pending output, returning
\name{OK} if successful.}

{\bf\begin{verbatim}
public  Status  input_character(
    FILE  *file,
    char   *ch )
\end{verbatim}}

\desc{Inputs a character from the file and passes it back as the
second argument.  If the character is the end
of file character, returns a status of \name{ERROR}, otherwise, \name{OK}.}

{\bf\begin{verbatim}
public  Status  unget_character(
    FILE  *file,
    char  ch )
\end{verbatim}}

\desc{Places the character back on the input queue.}

{\bf\begin{verbatim}
public  Status  input_nonwhite_character(
    FILE   *file,
    char   *ch )
\end{verbatim}}

\desc{Inputs the next non-white character, i.e. the next character
that is neither a blank, a tab, or a newline character.}

{\bf\begin{verbatim}
public  Status   skip_input_until(
    FILE   *file,
    char   search_char )
\end{verbatim}}

\desc{Inputs characters from the file until the specified search
character is found.}

{\bf\begin{verbatim}
public  Status  input_string(
    FILE  *file,
    char  str[],
    int   string_length,
    char  termination_char )
\end{verbatim}}

\desc{Inputs a string from the file, where the maximum storage of the
string is specified by the argument \name{string\_length}.  The file
is read into the string until either the termination character or a
newline character is found.  If the string ends at a newline
character and the termination character is not a newline, the
next character available from the file will be the newline
character.  Otherwise, the next character available is the one just
after the termination character found.
}

{\bf\begin{verbatim}
public  Status  input_quoted_string(
    FILE            *file,
    char            str[],
    int             str_length )
\end{verbatim}}

\desc{Inputs a string from a file, delimited by quotation marks,
returning \name{OK} or \name{ERROR}.  The
maximum storage available in the \name{str} pointer is specified by
the third argument, \name{str\_length}.   After successful reading of
a string, the next character available from the file is the one just
after the ending quotation mark.}

{\bf\begin{verbatim}
public  Status  input_line(
    FILE    *file,
    char    line[],
    int     str_length )
\end{verbatim}}

\desc{Inputs characters from the file into the argument \name{line},
up until the next newline character, or until \name{str\_length}
characters have been read.  If a newline was found, the next available
character will be the one after the newline.  The newline character is
not stored in the string.}

{\bf\begin{verbatim}
public  Status  input_boolean(
    FILE            *file,
    BOOLEAN         *b )
\end{verbatim}}

\desc{Inputs the next non-white character.
If it is a ``\name{f}'' or ``\name{F}'', then the value
\name{FALSE} is passed back.
If it is a ``\name{t}'' or ``\name{T}'', then the value
\name{TRUE} is passed back.  Otherwise \name{ERROR} is returned.}

{\bf\begin{verbatim}
public  Status  input_short(
    FILE            *file,
    short           *s )
public  Status  input_unsigned_short(
    FILE            *file,
    unsigned short  *s )
public  Status  input_int(
    FILE            *file,
    int             *i )
public  Status  input_real(
    FILE            *file,
    Real            *r )
public  Status  input_float(
    FILE            *file,
    float           *f )
public  Status  input_double(
    FILE            *file,
    double          *d )
\end{verbatim}}

\desc{Inputs a value of the specified type from an ascii file.}

{\bf\begin{verbatim}
public  Status  input_newline(
    FILE            *file )
\end{verbatim}}

\desc{Inputs and discards characters from the file up to and including
the next newline character.  Returns \name{OK} if a newline was found,
or \name{ERROR} if the end of file was reached first.}

{\bf\begin{verbatim}
public  Status  input_binary_data(
    FILE            *file,
    void            *data,
    size_t          element_size,
    int             n )
\end{verbatim}}

\desc{Inputs an array of data from a file, in binary format.  The
array contains \name{n} elements, each of size \name{element\_size}.
Returns either \name{OK} or \name{ERROR}.}

{\bf\begin{verbatim}
public  Status  output_character(
    FILE   *file,
    char   ch )
\end{verbatim}}

\desc{Outputs a character to the file, returning \name{OK} or
\name{ERROR}.}

{\bf\begin{verbatim}
public  Status  output_string(
    FILE  *file,
    char  str[] )
\end{verbatim}}

\desc{Outputs the specified string to the file, returning \name{OK} or
\name{ERROR}.}

{\bf\begin{verbatim}
public  Status  output_quoted_string(
    FILE            *file,
    char            str[] )
\end{verbatim}}

\desc{Outputs a quotation mark, the specified string, and a closing
quotation mark.}

{\bf\begin{verbatim}
public  Status  output_newline(
    FILE            *file )
\end{verbatim}}

\desc{Outputs a newline character, returning \name{OK} or \name{ERROR}.}

{\bf\begin{verbatim}
public  Status  output_boolean(
    FILE            *file,
    BOOLEAN         b )
\end{verbatim}}

\desc{If the argument is \name{TRUE}, then a space and the letter
``\name{T}'' is output, otherwise, a space and the letter ``\name{F}''
is output.
}

{\bf\begin{verbatim}
public  Status  output_short(
    FILE            *file,
    short           s )
public  Status  output_unsigned_short(
    FILE            *file,
    unsigned short  s )
public  Status  output_int(
    FILE            *file,
    int             i )
public  Status  output_real(
    FILE            *file,
    Real            r )
public  Status  output_float(
    FILE            *file,
    float           f )
public  Status  output_double(
    FILE            *file,
    double          d )
\end{verbatim}}

\desc{Outputs a space and then the specified value to an ascii file.}

{\bf\begin{verbatim}
public  Status  output_binary_data(
    FILE            *file,
    void            *data,
    size_t          element_size,
    int             n )
\end{verbatim}}

\desc{Outputs an array of data to a file, in binary format.  The
array contains \name{n} elements, each of size \name{element\_size}.
Returns either \name{OK} or \name{ERROR}.}

{\bf\begin{verbatim}
public  Status  io_binary_data(
    FILE            *file,
    IO_types        io_flag,
    void            *data,
    size_t          element_size,
    int             n )
\end{verbatim}}

\desc{Inputs or outputs the specified binary data, depending on
whether the argument \name{io\_flag} is \name{READ\_FILE} or
\name{WRITE\_FILE}.}

{\bf\begin{verbatim}
public  Status  io_newline(
    FILE            *file,
    IO_types        io_flag,
    File_formats    format )
\end{verbatim}}

\desc{Inputs or outputs an ascii newline character, depending on 
whether the argument \name{io\_flag} is \name{READ\_FILE} or
\name{WRITE\_FILE}.  If the \name{format} argument is
\name{BINARY\_FORMAT}, this function does nothing.}

{\bf\begin{verbatim}
public  Status  io_quoted_string(
    FILE            *file,
    IO_types        io_flag,
    File_formats    format,
    char            str[],
    int             str_length )
\end{verbatim}}

\desc{If the \name{format} argument is \name{ASCII\_FORMAT},
inputs or outputs a quotation mark delimited string, depending
on whether the argument \name{io\_flag} is \name{READ\_FILE} or
\name{WRITE\_FILE}.  IF the \name{format} argument is
\name{BINARY\_FORMAT}, inputs or outputs a string preceded by an
integer indicating the length of the string.}

{\bf\begin{verbatim}
public  Status  io_boolean(
    FILE            *file,
    IO_types        io_flag,
    File_formats    format,
    BOOLEAN         *b )
public  Status  io_short(
    FILE            *file,
    IO_types        io_flag,
    File_formats    format,
    short           *short_int )
public  Status  io_unsigned_short(
    FILE            *file,
    IO_types        io_flag,
    File_formats    format,
    unsigned short  *unsigned_short )
public  Status  io_unsigned_char(
    FILE            *file,
    IO_types        io_flag,
    File_formats    format,
    unsigned  char  *c )
public  Status  io_int(
    FILE            *file,
    IO_types        io_flag,
    File_formats    format,
    int             *i )
public  Status  io_real(
    FILE            *file,
    IO_types        io_flag,
    File_formats    format,
    Real            *r )
public  Status  io_float(
    FILE            *file,
    IO_types        io_flag,
    File_formats    format,
    float           *f )
public  Status  io_double(
    FILE            *file,
    IO_types        io_flag,
    File_formats    format,
    double          *d )
\end{verbatim}}

\desc{Inputs or outputs a binary or ascii value of the specified type.}

{\bf\begin{verbatim}
public  Status  io_ints(
    FILE            *file,
    IO_types        io_flag,
    File_formats    format,
    int             n,
    int             *ints[] )
\end{verbatim}}

\desc{Inputs or outputs an array of integers in binary or ascii
format.}

{\bf\begin{verbatim}
public  Status  io_unsigned_chars(
    FILE            *file,
    IO_types        io_flag,
    File_formats    format,
    int             n,
    unsigned char   *unsigned_chars[] )
\end{verbatim}}

\desc{Inputs or outputs an array of unsigned characters in binary or ascii
format.}

\section{Memory Allocation}

A set of macros is provided to allow easy allocation and deallocation
of memory, with up to 5 dimensional arrays.  Memory allocation
checking is also performed, to catch errors such as freeing memory
that was not allocated.  Also, the memory allocation automatically
tracks all memory allocated, so that detection of memory leaks
(orphaned memory) is possible.

\subsection{Basic Memory Allocation}

The basic macros are as follows:

{\bf\begin{verbatim}
     ALLOC( ptr, n_items )
\end{verbatim}}

\desc{Allocates \name{n\_items} elements of the correct type,
assigning the result to the argument \name{ptr}.}

{\bf\begin{verbatim}
    FREE( ptr )
\end{verbatim}}

\desc{Frees the memory pointed to by the argument \name{ptr}.}

{\bf\begin{verbatim}
    REALLOC( ptr, n_items )
\end{verbatim}}

\desc{Changes the size of the memory pointed to by \name{ptr} to be of
size \name{n\_items} elements, possibly changing the value of
\name{ptr} in the process.}

{\bf\begin{verbatim}
    ALLOC_VAR_SIZED_STRUCT( ptr, element_type, n_elements )
\end{verbatim}}

\desc{Allocates a variable sized structure, which must be of a
specific form.  The last element of the structure must be an array of
size 1, and this array will constitute the variable-sized part of the
structure.  The argument \name{element\_type} must be the type of this
last element, and the argument \name{n\_elements} is the desired
number of elements to allocate for this array, in addition to the
memory for the first part of the structure.  An example of usage is:}

\begin{verbatim}
        {
            struct {  int   a;
                      float b;
                      char  data[1];
                   }                    *ptr;

            ALLOC_VAR_SIZED_STRUCT( ptr, char, 10 );

            ptr->a = 1;
            ptr->b = 2.5;
            ptr->data[0] = 'a';
            ptr->data[9] = 'i';
        }
\end{verbatim}

{\bf\begin{verbatim}
    ALLOC2D( ptr, n1, n2 )
    ALLOC3D( ptr, n1, n2, n3 )
    ALLOC4D( ptr, n1, n2, n3, n4 )
    ALLOC5D( ptr, n1, n2, n3, n4, n5 )
\end{verbatim}}

\desc{Allocates a 2 to 5 dimensional array of size \name{n1} by \name{n2},
etc.  and stores the result in the specified pointer, \name{ptr}.  In
the 2 dimensional case, this is
accomplished with only 2 memory allocations, one to allocate \name{n1}
times \name{n2} elements for the storage, and the second to allocate
\name{n1} pointers into the first memory area.  In general, there is
one memory allocation for each dimension required.}

{\bf\begin{verbatim}
    FREE2D( ptr )
    FREE3D( ptr )
    FREE4D( ptr )
    FREE5D( ptr )
\end{verbatim}}

\desc{Frees the memory associated with the multi-dimensional array.}

\subsection{Higher Level Array Allocation}

In addition to the basic memory allocation macros described previously,
a set of useful macros for dealing with arrays of dynamically changing
size are provided:

{\bf\begin{verbatim}
    SET_ARRAY_SIZE( array, previous_n_elems, new_n_elems,
                    chunk_size )
\end{verbatim}}

\desc{This macro increases or decreases the size of an array, by
specifying the number of elements previously allocated to the array
(starts out at zero).
The \name{chunk\_size} argument defines the size of the memory chunks which
are allocated.  For instance, if \name{chunk\_size} is 100, then this
macro will only reallocate the array if the size change crosses to a
different multiple of 100, thus avoiding memory allocation every time
it is called.  This specification of the granularity of the memory
allocation must be consistently specified; if this macro is called
with a given variable and chunk size, then subsequent calls to this
macro with the same variable must specify the same chunk size.  Note also
that the number passed in as \name{new\_n\_elems} must be passed in as
\name{previous\_n\_elems} on the next call to this macro.}

{\bf\begin{verbatim}
    ADD_ELEMENT_TO_ARRAY( array, n_elems,
                          elem_to_add, chunk_size )
\end{verbatim}}

\desc{Adds the argument \name{elem\_to\_add} to the array at the
\name{n\_elems}'th index, then increments \name{n\_elems}. The
argument \name{chunk\_size} specifies the granularity of memory
allocation.}

{\bf\begin{verbatim}
    DELETE_ELEMENT_FROM_ARRAY( array, n_elems, index_to_remove,
                               chunk_size )
\end{verbatim}}

\desc{Deletes the \name{index\_to\_remove}'th element from the array,
decreasing the number of elements in the array (\name{n\_elems}) and
decreasing the memory allocation, if crossing a multiple of
\name{chunk\_size}.  Again, \name{chunk\_size} must be specified the
same for all invocations of the previous three macros involving a
given pointer.}

{\bf\begin{verbatim}
    ADD_ELEMENT_TO_ARRAY_WITH_SIZE( array, n_alloced, n_elems,
                                    elem_to_add, chunk_size )
\end{verbatim}}

\desc{Adds an element (\name{elem\_to\_add}) to the array,
incrementing \name{n\_elems}.  If necessary, the memory is increased
by the amount specified in \name{chunk\_size} and the \name{n\_alloced}
variable is incremented by this amount.  The usage of this differs
from the use of \name{ADD\_ELEMENT\_TO\_ARRAY} in that the number of
elements (\name{n\_elems}) can be decreased arbitrarily, without
causing memory to be deallocated.}

\section{Progress Reports}

In order to provide simple monitoring of the progress of a particular
processing task, a progress reporting module is available.  While a
task is in progress, the progress report prints dots across the line
indicating how close to finished the task is.  If the task is going to
take very long, (greater than 2 minutes), the progress report
periodically prints the current percentage done and the estimated time
remaining.  An example of usage, followed by function descriptions, is
presented:

\begin{verbatim}
        {
            int                 i, n_slices;
            progress_struct     progress;

            n_slices = 100;

            initialize_progress_report( &progress, FALSE,
                        n_slices, "Processing Slices" );

            for( i = 0;  i < n_slices;  ++i )
            {
                process_slice( i );

                update_progress_report( &progress, i + 1 );
            }

            terminate_progress_report( &progress );
        }
\end{verbatim}

{\bf\begin{verbatim}
public  void  initialize_progress_report(
    progress_struct   *progress,
    BOOLEAN           one_line_only,
    int               n_steps,
    char              title[] )
\end{verbatim}}

\desc{Initializes a progress report struct, specifying the number of
steps that will occur in  the processing, and the title to print out
for the progress report.  During progress report, the display will
automatically switch from the short-job mode of printing a single row
of dots across a line to the long-job mode of periodically printing
the percentage done and estimated time remaining.  If the
\name{one\_line\_only} flag is \name{TRUE}, this is disabled and only
a single row of dots will be displayed.}

{\bf\begin{verbatim}
public  void  update_progress_report(
    progress_struct   *progress,
    int               current_step )
\end{verbatim}}

\desc{Tells the progress reporting module how many steps have been
done, and causes update of the display of dots on the line or
estimated time remaining.}

{\bf\begin{verbatim}
public  void  terminate_progress_report(
    progress_struct   *progress )
\end{verbatim}}

\desc{Terminates the progress report.}

\section{Strings}

Some simple string manipulation techniques are provided:

{\bf\begin{verbatim}
public  BOOLEAN  string_ends_in(
    char   string[],
    char   ending[] )
\end{verbatim}}

\desc{Determines if the string ends in the specified \name{ending}.  For
instance, \name{string\_ends\_in( "rainfall", "fall" )} returns
\name{TRUE}.}

{\bf\begin{verbatim}
public    void   strip_blanks(
    char  str[],
    char  stripped[] )
\end{verbatim}}

\desc{Removes leading and trailing blanks from the string, placing the
result in the argument \name{stripped}.
Correctly handles case where the two string arguments point to the same memory.}

{\bf\begin{verbatim}
public  void  get_upper_case_string(
    char    string[],
    char    upper_case[] )
\end{verbatim}}

\desc{Converts the string to an all upper case string.
Correctly handles case where the two string arguments point to the same memory.}

{\bf\begin{verbatim}
public  int  find_character(
    char    string[],
    char    ch )
\end{verbatim}}

\desc{Searches for the given character in the given string \,
returning the index where it was found, or -1 if it was not found.}

{\bf\begin{verbatim}
public  BOOLEAN  blank_string(
    char   string[] )
\end{verbatim}}

\desc{Returns true if the string is empty or consists of only space,
tab, and newline characters.}

\section{Text Output}

Rather than using the standard UNIX function \name{printf} for routine
text output, a module which is similar in appearance is provided,
which allows for installing arbitrary printing output functions.  This
can be used, for instance, to mirror output to a log file for
instance, or to send text to an X window.  The primary change is to
use a function called \name{print} with exactly the same arguments as
for the standard \name{printf} function.

{\bf\begin{verbatim}
public  void  print( char format[], ... )
\end{verbatim}}

\desc{Takes the same arguments as \name{printf()}, but allows
installing of a user output function for the final stage of output.}

{\bf\begin{verbatim}
public  void  set_print_function(
    void     (*function) ( char [] ) )
\end{verbatim}}

\desc{Sets the output function, where all text from calls to
\name{print()} will be sent.  By default, there is no print function,
and output is sent to printf().}

{\bf\begin{verbatim}
public  void  push_print_function()
\end{verbatim}}

\desc{Temporarily sets the print function to go to standard out.}

{\bf\begin{verbatim}
public  void  pop_print_function()
\end{verbatim}}

\desc{Restores the previous user print function.}

\section{Time}

Some basic utilities relating to time, date, and CPU time are
provided.

{\bf\begin{verbatim}
public  Real  current_cpu_seconds( void )
\end{verbatim}}

\desc{Returns the number of CPU seconds used by the current process so
far.}

{\bf\begin{verbatim}
public  Real  current_realtime_seconds( void )
\end{verbatim}}

\desc{Returns the number of seconds that the current process has been
running.}

{\bf\begin{verbatim}
public  void  get_clock_time(
    char  time_str[] )
\end{verbatim}}

\desc{Copies the current clock time (hours and minutes) into a string.}

{\bf\begin{verbatim}
public  void  get_date(
    char  date_str[] )
\end{verbatim}}

\desc{Copies the current date into a string.}

{\bf\begin{verbatim}
public  void  format_time(
    char   str[],
    char   format[],
    Real   seconds )
\end{verbatim}}

\desc{Takes a time in seconds and a format string which has format
components, i.e. ``\%g \%s'', and prints the time into the string
\name{str}, in appropriate units of milliseconds, seconds, minutes,
days, etc.}

{\bf\begin{verbatim}
public  void  print_time(
    char   format[],
    Real   seconds )
\end{verbatim}}

\desc{Same as \name{format\_time}, but calls \name{print()} with the
result.}

{\bf\begin{verbatim}
public  void  sleep_program(
    Real seconds )
\end{verbatim}}

\desc{Suspends the program for the specified number of seconds.  Note
that on most systems, this will only be performed to the nearest
multiple of some particular time increment.  On Silicon Graphics
Systems this will be to the nearest hundredth of a second.}

\chapter{Volumes}

Processing tasks within the lab where this software was developed
deal with multi-dimensional volumes of data such as created by
Magnetic Resonance and PET scanners.  Therefore, an extensive set of
routines is provided to represent volumes, and to read and write
volumes in the MINC format.

The basic type of a volume is \name{Volume}, which is actually a
pointer to an allocated structure which contains all the information
about the type of volume, number of dimensions, voxel values, etc.  In
order to use a volume structure, the volume must first be created, then
the size of the volume set, then the large array of voxel values is
allocated.  Alternately, a volume may be automatically created by
calling the appropriate function to read a MINC file and create a
volume.

A volume has an associated number of dimensions, which must be in the range
from one to five, but typically is three.  The volume is thought of as
a multi-dimensional array of any of a variety of types, including all
sizes of integer and real types.  Even though a volume may be
stored in a 1 byte type, with values from zero to 255, there is an
optional real value mapping, which provides a mapping to any arbitrary
real range.  In this way, the real range may be any valid real
interval and is independent of the particular storage type.

Since most volumes will be created by reading from a MINC file, this
method will be presented first, followed by a description of how to
create a volume from scratch.  Finally a lower level of MINC input and
output for more advanced applications will be presented.

\section{Volume Input}

{\bf\begin{verbatim}
public  Status  input_volume(
    char                 filename[],
    int                  n_dimensions,
    char                 *dim_names[],
    nc_type              volume_nc_data_type,
    BOOLEAN              volume_signed_flag,
    Real                 volume_voxel_min,
    Real                 volume_voxel_max,
    BOOLEAN              create_volume_flag,
    Volume               *volume,
    minc_input_options   *options )
\end{verbatim}}

\desc{This routine reads a volume from a MINC file, first creating
the volume if the \name{create\_volume\_flag} is specified as
\name{TRUE} (the usual case).  The
number of dimensions is the desired number of dimensions for the
volume.  If this is less than the number of dimensions in the file,
then only the first part of the file, corresponding to this number of
dimensions, is read.  The argument \name{dim\_names} specifies in what
order the volume is to be stored in the volume variable.  For each
dimension in the stored volume, there is a corresponding name, which
is one of \name{MIxspace}, \name{MIyspace}, \name{MIzspace},
\name{ANY\_SPATIAL\_DIMENSION}, or an empty string.  These are matched
up with corresponding dimensions in the file and the dimension
ordering of the volume array is reordered on input.  So, if the user
wishes to represent the volume in X-Z-Y order, then the value passed
as the \name{dim\_names} array should be the three strings,
\name{MIxspace}, \name{MIzspace}, and \name{MIyspace}.}

\desc{The four arguments, \name{volume\_nc\_data\_type},
\name{volume\_signed\_flag}, \name{volume\_voxel\_min}, and
\name{volume\_voxel\_max} can be used to specified the desired storage
type within the volume variable, automatically converted from the
storage type in the file.  The \name{volume\_nc\_data\_type} is one of
\name{NC\_UNSPECIFIED}, \name{NC\_BYTE}, \name{NC\_SHORT}, \name{NC\_LONG},
\name{NC\_FLOAT}, or \name{NC\_DOUBLE}.
For the integer types, the
\name{volume\_signed\_flag} is \name{TRUE} if a signed type is
desired, otherwise, it is \name{FALSE}.  The \name{volume\_voxel\_min}
and \name{volume\_voxel\_max} specify the range of valid voxel values,
and are usually set equal to indicate to use the full range of the
type, e.g. zero to 255.0 for unsigned \name{NC\_BYTE}.
If \name{NC\_UNSPECIFIED} is passed, then the type, sign, and voxel
range in the file are used.}

\desc{If the \name{create\_volume\_flag} is \name{TRUE}, the usual
case, then the volume is automatically created.  Otherwise, it is
assumed that the volume already exists and will only be recreated if
its current size is different from the new size resulting from the
file, thus reducing the amount of memory allocation when reading
multiple files.}

\desc{The \name{options} argument specifies some special options to
the input process, and is usually passed just a \name{NULL} pointer,
indicating that the default options should be used.  Currently the
possible options are:}

{\bf\begin{verbatim}
public  void  set_default_minc_input_options(
    minc_input_options  *options )
\end{verbatim}}

\desc{Fills in the default options into an \name{options} structure
which will subsequently be passed to \name{input\_volume}.}

{\bf\begin{verbatim}
public  void  set_minc_input_promote_invalid_to_min_flag(
    minc_input_options  *options,
    BOOLEAN             flag )
\end{verbatim}}

\desc{By default, any voxel value which is outside the valid range of
voxel values is promoted to the minimum valid voxel value.  If this
\name{flag} is set to \name{FALSE}, this promotion is disabled.}

{\bf\begin{verbatim}
public  void  set_minc_input_vector_to_scalar_flag(
    minc_input_options  *options,
    BOOLEAN             flag )
\end{verbatim}}

\desc{By default, any volume which contains a dimension which is of
type vector, for instance, an RGB colour volume, is converted to a
scalar volume by averaging the components of each vector.  This can be
disabled by passing the \name{flag} as \name{FALSE}.}

\section{Volume Output}

Volume output is accomplished by one of two routines, depending on
whether or not the volume is treated as a modified version of another
volume or is an independent volume on its own.

{\bf\begin{verbatim}
public  Status  output_volume(
    char                  filename[],
    nc_type               file_nc_data_type,
    BOOLEAN               file_signed_flag,
    Real                  file_voxel_min,
    Real                  file_voxel_max,
    Volume                volume,
    char                  history[],
    minc_output_options   *options )
\end{verbatim}}

\desc{Outputs the specified volume to the specified filename.  If the
argument \name{file\_nc\_data\_type} is \name{NC\_UNSPECIFIED} then the
volume is stored in the MINC file in the same type as in the volume
variable.  Otherwise, the four arguments, \name{file\_nc\_data\_type},
\name{file\_signed\_flag}, \name{file\_voxel\_min}, and
\name{file\_voxel\_max}, specify the type and valid voxel range to store
the volume in the file.  If the \name{history} argument is non-null,
then it represents a description of the volume, and will be placed in
the MINC volume.  If the \name{options} argument is \name{NULL}, then
the default options will be used.  Otherwise, some specific output
options can be set through this parameter, and the following
functions:}

{\bf\begin{verbatim}
public  void  set_default_minc_output_options(
    minc_output_options  *options )
\end{verbatim}}

\desc{Sets the \name{options} structure to the default values.  The
user can then override the default values and pass the structure to
the \name{output\_volume()} function.  Currently, there is only one
output option:}

{\bf\begin{verbatim}
public  void  set_minc_output_dimensions_order(
    minc_output_options  *options,
    int                  n_dimensions,
    STRING               dimension_names[] )
\end{verbatim}}

\desc{Defines the output order of the file.  Each dimension name
string must have a matching dimension name in the volume, which
defines the order of the dimensions in the output file.  For instance,
one may have input the volume in \name{(x, y, z)} order.
To get it output in \name{(z, y, x)} order, set \name{dimension\_names} to
the three strings \name{MIzspace},
\name{MIyspace}, and \name{MIxspace}.}

If the volume is a modification of another volume currently stored in
a file, then it is more appropriate to use the following function to
output the volume:

{\bf\begin{verbatim}
public  Status  output_modified_volume(
    char                  filename[],
    nc_type               file_nc_data_type,
    BOOLEAN               file_signed_flag,
    Real                  file_voxel_min,
    Real                  file_voxel_max,
    Volume                volume,
    char                  original_filename[],
    char                  history[],
    minc_output_options   *options )
\end{verbatim}}

\desc{The only difference between this function and the other method
of volume output (\name{output\_volume()}), is that this function copies
auxiliary data from the original file (\name{original\_filename}) to
the new file.  This auxiliary data includes such items as patient
name and other scanning data, and is not read into a volume, so the
only way to correctly pass it along to a new MINC file is to use this
function.}

\section{Volume Manipulation}

Once a volume has been created and allocated, there are many macros
and functions for manipulating the volume.  Note that associated with
each volume is a valid voxel range indicating the range of values
actually stored in the volume, for instance, zero to 200 is one
possible range for an unsigned byte volume.  There is a second range,
the \name{real} range, which is the mapping of the valid voxel range
to an arbitrary real range, for instance, the zero to 200 of valid
voxels could map to -1.5 to 1.5 in the \name{real} range.  When
dealing with volumes, one is generally interested in the \name{real}
range.

{\bf\begin{verbatim}
GET_VOXEL( value, volume, v1, v2, v3, v4, v5 )
\end{verbatim}}

\desc{Given a volume and from one to five voxel indices, depending on
the volume dimensions, stores the corresponding voxel value
in the variable \name{value}, which may be of any type.  For instance,
if the volume is three
dimensional, then the final two arguments are ignored.}

{\bf\begin{verbatim}
GET_VOXEL_1D( value, volume, v1 )
GET_VOXEL_2D( value, volume, v1, v2 )
GET_VOXEL_3D( value, volume, v1, v2, v3 )
GET_VOXEL_4D( value, volume, v1, v2, v3, v4 )
GET_VOXEL_5D( value, volume, v1, v2, v3, v4, v5 )
\end{verbatim}}

\desc{More convenient versions of \name{GET\_VOXEL}.  Given a volume and
the required number of voxel indices for the
volume, assigns the corresponding voxel value to the argument
\name{value}.}

{\bf\begin{verbatim}
CONVERT_VOXEL_TO_VALUE( volume, voxel )
\end{verbatim}}

\desc{Given a volume and a voxel value, converts this to a value in
the real range, and returns it.}

{\bf\begin{verbatim}
CONVERT_VALUE_TO_VOXEL( volume, value )
\end{verbatim}}

\desc{Given a volume and a real value, converts this to a voxel value in
the valid voxel range, and returns it.}

{\bf\begin{verbatim}
GET_VALUE( value, volume, v1, v2, v3, v4, v5 )
\end{verbatim}}

\desc{Given a volume and from one to five voxel indices, depending on
the volume dimensions, stores the corresponding real value in the
variable \name{value},
which may be of any type.  For instance, if the volume is three
dimensional, then the final two arguments are ignored.}

{\bf\begin{verbatim}
GET_VALUE_1D( value, volume, v1 )
GET_VALUE_2D( value, volume, v1, v2 )
GET_VALUE_3D( value, volume, v1, v2, v3 )
GET_VALUE_4D( value, volume, v1, v2, v3, v4 )
GET_VALUE_5D( value, volume, v1, v2, v3, v4, v5 )
\end{verbatim}}

\desc{More convenient versions of \name{GET\_VALUE}.
Given a volume and the required number of voxel indices for the
volume, assigns the corresponding real value to the argument
\name{value}.  This is the most efficient method to access the volume
values (as compared to the \name{GET\_VALUE} macro, for instance).
Typically, the \name{GET\_VALUE\_3D} macro is used for
most applications.}

{\bf\begin{verbatim}
SET_VOXEL( volume, v1, v2, v3, v4, v5, value )
\end{verbatim}}

\desc{Given a volume, one to five voxel indices, and a \name{voxel}
value, assigns this value to the corresponding voxel in the volume.
Note that no conversion from the valid real range to the valid voxel
range is performed, so the user may need to use the
\name{CONVERT\_VALUE\_TO\_VOXEL} macro first.}

{\bf\begin{verbatim}
public  void  delete_volume(
    Volume   volume )
\end{verbatim}}

\desc{Deletes all the memory associated with an volume.}

{\bf\begin{verbatim}
public  nc_type  get_volume_nc_data_type(
    Volume       volume,
    BOOLEAN      *signed_flag )
\end{verbatim}}

\desc{Returns the storage type of the volume (for instance, \name{NC\_SHORT}),
and passes back an indication of whether it is signed or unsigned.}

{\bf\begin{verbatim}
public  int  get_volume_n_dimensions(
    Volume   volume )
\end{verbatim}}

\desc{Returns the number of dimensions of the volume.}

{\bf\begin{verbatim}
public  void  get_volume_sizes(
    Volume   volume,
    int      sizes[] )
\end{verbatim}}

\desc{Stores the size of each dimension in the array \name{sizes}.
This is the number of voxels in each dimension.  The array
\name{sizes} must
be at least as large as the number of dimensions of the volume.}

{\bf\begin{verbatim}
public  void  get_volume_separations(
    Volume   volume,
    Real     separations[] )
\end{verbatim}}

\desc{Stores the slice separation for each dimension in the array
\name{separations}.  The array \name{separations} must
be at least as large as the number of dimensions of the volume.}

{\bf\begin{verbatim}
public  Real  get_volume_voxel_min(
    Volume   volume )
public  Real  get_volume_voxel_max(
    Volume   volume )
public  void  get_volume_voxel_range(
    Volume     volume,
    Real       *voxel_min,
    Real       *voxel_max )
\end{verbatim}}

\desc{The first two functions return the minimum or maximum allowable
voxel value.  The third function passes back both values.}

{\bf\begin{verbatim}
public  Real  get_volume_real_min(
    Volume     volume )
public  Real  get_volume_real_max(
    Volume     volume )
public  void  get_volume_real_range(
    Volume     volume,
    Real       *min_value,
    Real       *max_value )
\end{verbatim}}

\desc{The first two functions return the minimum or maximum real
value.  The third function passes back both values.  The mapping to
this real space linearly maps the minimum \name{voxel} value to the minimum
\name{real} value and the maximum \name{voxel} value to the maximum
\name{real} value.}

{\bf\begin{verbatim}
public  char  **get_volume_dimension_names(
    Volume   volume )
\end{verbatim}}

\desc{Returns a pointer to the list of names of each dimension.}

\subsection{Volume Coordinate Systems}

A volume has two coordinate systems.  The {\bf voxel} coordinate
system is simply the n-dimensional indexing coordinate system for a
volume.  A voxel coordinate of (0.0, 0.0, 0.0), for instance, corresponds
to the centre of the first voxel in a three dimensional volume.  A voxel
coordinate of ( 99.0, 0.0, 0.0 ) corresponds to the centre of the last
voxel in the first direction of a three dimensional volume of size (
100, 200, 150 ).  The second coordinate system is an arbitrary
three dimensional coordinate system, usually referred to as the
{\bf world} coordinate system, often
the Talairach coordinate system.  The following functions provide
conversion to and from these two coordinate systems:

{\bf\begin{verbatim}
public  void  convert_voxel_to_world(
    Volume   volume,
    Real     voxel[],
    Real     *x_world,
    Real     *y_world,
    Real     *z_world )
\end{verbatim}}

\desc{Given a volume and a real valued voxel index, passes back the
corresponding world coordinate.}

{\bf\begin{verbatim}
public  void  convert_3D_voxel_to_world(
    Volume   volume,
    Real     voxel1,
    Real     voxel2,
    Real     voxel3,
    Real     *x_world,
    Real     *y_world,
    Real     *z_world )
\end{verbatim}}

\desc{Same as \name{convert\_voxel\_to\_world} except it applies only
to three dimensional volumes.}

{\bf\begin{verbatim}
public  void  convert_world_to_voxel(
    Volume   volume,
    Real     x_world,
    Real     y_world,
    Real     z_world,
    Real     voxel[] )
\end{verbatim}}

\desc{Converts a world coordinate into a voxel.  In order to use
these voxel coordinates as integer indices, for instance, as arguments
to the \name{GET\_VALUE} macros, each component of the argument
\name{voxel} must first be rounded to the nearest integer.}

{\bf\begin{verbatim}
public  void  convert_3D_world_to_voxel(
    Volume   volume,
    Real     x_world,
    Real     y_world,
    Real     z_world,
    Real     *voxel1,
    Real     *voxel2,
    Real     *voxel3 )
\end{verbatim}}

\desc{Same as \name{convert\_world\_to\_voxel} except it applies only
to three dimensional volumes.}

{\bf\begin{verbatim}
public  void  convert_voxel_normal_vector_to_world(
    Volume          volume,
    Real            voxel1,
    Real            voxel2,
    Real            voxel3,
    Real            *x_world,
    Real            *y_world,
    Real            *z_world )
\end{verbatim}}

\desc{Converts a vector which is assumed to be a surface normal to the
world coordinate system.}

\section{Volume Creation from Scratch}

In some circumstances, it is desirable to create volumes in ways
other than reading from a file.  The following functions provide
methods to create a volume from scratch or to create a volume which is
similar to an existing volume.

{\bf\begin{verbatim}
public   Volume   create_volume(
    int         n_dimensions,
    char        *dimension_names[],
    nc_type     nc_data_type,
    BOOLEAN     signed_flag,
    Real        voxel_min,
    Real        voxel_max )
\end{verbatim}}

\desc{Creates and returns a volume of the given type (for instance,
\name{NC\_BYTE}, \name{signed\_flag} equal \name{FALSE}), and given
valid voxel range.  The \name{dimension\_names} is used to describe
each dimension of the volume and is currently only used when writing
the volume to a file.}

{\bf\begin{verbatim}
public  void  set_volume_voxel_range(
    Volume   volume,
    Real     voxel_min,
    Real     voxel_max )
public  void  set_volume_real_range(
    Volume   volume,
    Real     real_min,
    Real     real_max )
\end{verbatim}}

\desc{After creation of a volume, the valid voxel range or valid real
range can subsequently be changed by using these functions.}

{\bf\begin{verbatim}
public  void  set_volume_sizes(
    Volume       volume,
    int          sizes[] )
\end{verbatim}}

\desc{Sets the sizes of the volume, the number of voxels in each
dimension.  Note that this must be done before calling the function
\name{alloc\_volume\_data} to allocate the voxels.}

{\bf\begin{verbatim}
public  void  alloc_volume_data(
    Volume   volume )
\end{verbatim}}

\desc{After the volume has been created, and its size has been set,
then this function allocates the memory for the voxels.  Note that the
voxel values are not initialized, and the user must fill the volume
with desired values.}

Associated with each volume is a transformation from \name{voxel}
space to \name{world} space.  There are several ways to define this
transformation.  The simplest is just to specify it directly:

{\bf\begin{verbatim}
public  void  set_voxel_to_world_transform(
    Volume             volume,
    General_transform  *transform )
\end{verbatim}}

\desc{Assigns the given transformation to the volume.}

{\bf\begin{verbatim}
public  void  set_volume_separations(
    Volume   volume,
    Real     separations[] )
\end{verbatim}}

\desc{Sets the inter-voxel separations in each of the volume
dimensions.  Note that this will cause the voxel-to-world
transformation to be updated accordingly.}

{\bf\begin{verbatim}
public  void  set_volume_translation(
    Volume  volume,
    Real    voxel[],
    Real    world_space_voxel_maps_to[] )
\end{verbatim}}

\desc{Sets the translation portion of the voxel-to-world
transformation.  A voxel coordinate is specified (\name{voxel}), as
well as a real world position to which it is desired that this voxel
map (\name{world\_space\_voxel\_maps\_to}).  The voxel-to-world
transformation is updated to provide this mapping.}

{\bf\begin{verbatim}
public  void  set_volume_direction_cosine(
    Volume   volume,
    int      dimension,
    Real     dir[] )
\end{verbatim}}

\desc{Sets the real world axis for a specific voxel dimension.  For
instance if \name{dimension} is 1, and \name{dir} is (0.0, 1.0, 1.0),
then voxels along the second dimension of the volume will map to the real
world axis (0.0, 1.0, 1.0) normalized to unit length, then scaled by
the volume separation for the second volume dimension.}

\subsection{Copying Volumes}

Another method of creating a volume is to simply copy the entire
volume definition from an existing volume:

{\bf\begin{verbatim}
public  Volume   copy_volume_definition(
    Volume   existing_volume,
    nc_type  nc_data_type,
    BOOLEAN  signed_flag,
    Real     voxel_min,
    Real     voxel_max )
\end{verbatim}}

\desc{Creates and returns a new volume which has the same definition
(sizes, voxel-to-world space transform, etc.), as an existing volume.
If the argument \name{nc\_data\_type} is not \name{NC\_UNSPECIFIED},
then the storage type of the new volume differs from the original and
is specified by \name{nc\_data\_type}, \name{signed\_flag},
\name{voxel\_min}, and \name{voxel\_max}.  The voxel values are allocated
but not initialized.}

\section{Source Code Example}

An examples of reading, writing, and manipulating volumes is
presented here.

This is a complete program to read a MINC volume, change all values
over 100.0 to 100.0, then write out the result in a new file.

{\small
\begin{verbatim}
#include  <volume_io.h>

int  main(
    int   argc,
    char  *argv[] )
{
    int        v1, v2, v3, sizes[MAX_DIMENSIONS];
    Real       voxel_value, voxel_100;
    Volume     volume;

    /*--- input the volume */

    if( input_volume( "volume.mnc", 3, NULL, NC_UNSPECIFIED, FALSE,
            0.0, 0.0, TRUE, &volume,
            (minc_input_options *) NULL ) != OK )
        return( 1 );

    get_volume_sizes( volume, sizes );

    /*--- change all values over 100 to 100 */

    voxel_100 = CONVERT_VALUE_TO_VOXEL( volume, 100.0 );

    for( v1 = 0;  v1 < sizes[0];  ++v1 ) {
        for( v2 = 0;  v2 < sizes[1];  ++v2 ) {
            for( v3 = 0;  v3 < sizes[2];  ++v3 ) {
                GET_VALUE_3D( voxel_value, volume, v1, v2, v3 );

                if( voxel_value > 100.0 ) {
                    SET_VOXEL_3D( volume, v1, v2, v3, voxel_100 );
                }
            }
        }
    }

    /*--- output the modified volume */

    if( output_modified_volume( "output.mnc", NC_UNSPECIFIED,
             FALSE, 0.0, 0.0, volume, "volume.mnc",
             "Modified by clamping to 100",
             (minc_output_options *) NULL ) != OK )
        return( 1 );

    return( 0 );
}
\end{verbatim}
}

\section{Multiple Volume Input}

The \name{input\_volume} function described previously is intended to
be a simple single-function interface for reading volumes for most
applications.  Occasionally, however, an application may not want to
read the whole volume at once, or may need to break the input file
into several volumes, for instance, the slices of a 3D volume.  This can
be facilitated by the following functions:

{\bf\begin{verbatim}
public  Minc_file  initialize_minc_input(
    char                 filename[],
    Volume               volume,
    minc_input_options   *options )
\end{verbatim}}

\desc{This functions opens a MINC file for input to the specified
volume.  The number of
dimensions in the file must be greater than or equal to the number of
dimensions in the volume, and each dimension name in the volume must
have a matching dimension name in the file.  A file handle is
returned, for use in the following routines.}

{\bf\begin{verbatim}
public  int  get_n_input_volumes(
    Minc_file  file )
\end{verbatim}}

\desc{Determines the total number of volumes in the opened MINC file.
For instance, if the file contains x, y, z data of size 100 by 200 by
300 and the file was opened with a two dimensional x-z volume, then
the number of volumes returned will be 200, the number of y slices.}

{\bf\begin{verbatim}
public  BOOLEAN  input_more_minc_file(
    Minc_file   file,
    Real        *fraction_done )
\end{verbatim}}

\desc{Reads part of the MINC file to the volume specified in
\name{initialize\_minc\_input}.  This function must be called until it
returns FALSE, at which point the volume has been fully read in.  In
order to read the next volume in the file, if any, the following
function must be called.}

{\bf\begin{verbatim}
public  BOOLEAN  advance_input_volume(
    Minc_file   file )
\end{verbatim}}

\desc{Advances the input file to the next volume.}

{\bf\begin{verbatim}
public  BOOLEAN  reset_input_volume(
    Minc_file   file )
\end{verbatim}}

\desc{Resets the input to the beginning of the first volume in the
file.}

{\bf\begin{verbatim}
public  Status  close_minc_input(
    Minc_file   file )
\end{verbatim}}

\desc{Closes the MINC file.}

\section{Multiple Volume Output}

Similarly to the multiple volume input, there are routines for
providing output to a file by subvolumes.

{\bf\begin{verbatim}
public  Minc_file  initialize_minc_output(
    char                   filename[],
    int                    n_dimensions,
    STRING                 dim_names[],
    int                    sizes[],
    nc_type                file_nc_data_type,
    BOOLEAN                file_signed_flag,
    Real                   file_voxel_min,
    Real                   file_voxel_max,
    Real                   real_min,
    Real                   real_max,
    General_transform      *voxel_to_world_transform,
    minc_output_options    *options )
\end{verbatim}}

\desc{Opens a MINC file for creation.  The arguments specify the
number of dimensions and the names of each dimension, as well as the
sizes in each dimensions.  The type of the file is controlled by the
four parameters, \name{file\_nc\_data\_type},
\name{file\_signed\_flag}, \name{file\_voxel\_min}, and
\name{file\_voxel\_max}.  The mapping of voxels to real values is
specified by \name{real\_min} and \name{real\_max}.  The
transformation to world coordinates is specified by the
\name{voxel\_to\_world\_transform} argument.  The \name{options} are
as specified for the \name{output\_volume} function.}

{\bf\begin{verbatim}
public  Status  copy_auxiliary_data_from_minc_file(
    Minc_file   file,
    char        filename[],
    char        history_string[] )
\end{verbatim}}

\desc{Copies the auxiliary information from the MINC file specified in
the \name{filename} argument to the open MINC \name{file}.}

{\bf\begin{verbatim}
public  Status  output_minc_volume(
    Minc_file   file,
    Volume      volume )
\end{verbatim}}

\desc{Outputs a volume to the MINC file.  The volume must have no more
dimensions than the file and the names of each dimension must match
with one of the dimensions in the file (specified by
\name{initialize\_minc\_output}).}

{\bf\begin{verbatim}
public  Status  close_minc_output(
    Minc_file   file )
\end{verbatim}}

\desc{Closes the MINC file.}

\chapter{Tag Points}

Tag points are sets of three dimensional points which are used to mark
positions of interest.  Generally these result from manually locating
corresponding positions in multiple volumes to provide a mapping among
the volumes.  Tag points are stored in an ascii format devised at the
McConnell Brain Imaging Centre, with a convention of filenames ending in
\name{.tag}.  A tag file can contain either sets of single tag points
or two tag points, depending on whether the file corresponds to one or
two volumes.  Each tag point consists of three coordinates, x, y, and
z.  Each set of one or two tag points has additional optional
information stored with it.  A tag point set may or may not include a
set of three values:  a real valued weight, an integer structure id,
and an integer patient id.  A tag point may or may not include a label
string.  Functions to read and write files of this
format are described:

{\bf\begin{verbatim}
public  Status  input_tag_points(
    FILE      *file,
    int       *n_volumes,
    int       *n_tag_points,
    Real      ***tags_volume1,
    Real      ***tags_volume2,
    Real      **weights,
    int       **structure_ids,
    int       **patient_ids,
    char      ***labels )
public  Status  input_tag_file(
    char      filename[],
    int       *n_volumes,
    int       *n_tag_points,
    Real      ***tags_volume1,
    Real      ***tags_volume2,
    Real      **weights,
    int       **structure_ids,
    int       **patient_ids,
    char      ***labels )
\end{verbatim}}

\desc{These two functions read a set of tag points from a file.  The
first form assumes the file is already open and will be closed by the
calling function, while the second form
opens and closes the file specified by a filename, with a default
extension of \name{.tag}.  The number of
volumes (number of tags in a set, currently either one or two) is
passed back in the \name{n\_volumes} argument.  The number of tag
points is passed back in the \name{n\_tag\_points} argument.  The
three dimensional coordinates of the first tag point in each set is
passed back in the argument \name{tags\_volume1}.  If the number of
volumes is two, then the second tag point in each set is passed back
in the argument \name{tags\_volume2}.  The final four arguments pass
back the auxiliary information associated with each tag point set.  If
the calling program is not interested in any one of the four data,
then it can pass in a \name{NULL} pointer and the values in the file
will not be passed back.}

{\bf\begin{verbatim}
public  void  free_tag_points(
    int       n_volumes,
    int       n_tag_points,
    Real      **tags_volume1,
    Real      **tags_volume2,
    Real      weights[],
    int       structure_ids[],
    int       patient_ids[],
    char      **labels )
\end{verbatim}}

\desc{When finished with a list of tag points, the
associated memory may be freed by calling this function.}

{\bf\begin{verbatim}
public  Status  output_tag_points(
    FILE      *file,
    char      comments[],
    int       n_volumes,
    int       n_tag_points,
    Real      **tags_volume1,
    Real      **tags_volume2,
    Real      weights[],
    int       structure_ids[],
    int       patient_ids[],
    char      **labels )
public  Status  output_tag_file(
    char      filename[],
    char      comments[],
    int       n_volumes,
    int       n_tag_points,
    Real      **tags_volume1,
    Real      **tags_volume2,
    Real      weights[],
    int       structure_ids[],
    int       patient_ids[],
    char      **labels )
\end{verbatim}}

\desc{These two functions write a list of tag points to a tag point
file.  The first form assumes that the file has already been opened
and will be closed later by the calling function, while the second
form opens and closes the file, with the default extension of
\name{.tag}.  The \name{comments} argument is any arbitrary string
documenting the contents of this file.  The number of volumes
(\name{n\_volumes}) must be either one or two.  The number of tag
points is specified by the argument \name{n\_tag\_points}.  The
positions of the sets of one or two tag points are in the arguments
\name{tags\_volume1} and \name{tags\_volume2}.  If any of the three
arguments, \name{weights}, \name{structure\_ids}, or
\name{patient\_ids}, are specified as \name{NULL}, then none of these
three pieces or information is written to the file.  Similarly, if
the \name{labels} argument is \name{NULL}, then no labels are written
to the file.}

\section{Source Code Example}

The following is an example which reads a tag volume, removes all tags
which have negative x positions, ignores the second tag point in each
set, if present, and writes the result to a new file.

{\small
\begin{verbatim}
#include  <volume_io.h>

int  main(
    int   argc,
    char  *argv[] )
{
    int        i, n_volumes, n_tag_points, *structure_ids, *patient_ids;
    Real       **tags1, **tags2, *weights;
    char       **labels;
    int        new_n_tag_points, *new_structure_ids, *new_patient_ids;
    Real       **new_tags1, *new_weights;
    char       **new_labels;

    /*--- input the tag file */

    if( input_tag_file( "input_tags.tag", &n_volumes, &n_tag_points,
                        &tags1, &tags2, &weights, &structure_ids,
                        &patient_ids, &labels ) != OK )
        return( 1 );

    /*--- create a new tag point list of only those tag points
          whose x coordinate is nonnegative */

    new_n_tag_points = 0;

    for_less( i, 0, n_tag_points )
    {
        if( tags1[i][0] >= 0.0 )
        {
            /*--- increase the memory allocation of the tag points */

            SET_ARRAY_SIZE( new_tags1, new_n_tag_points,
                            new_n_tag_points+1, 10 );
            ALLOC( new_tags1[new_n_tag_points], 3 );

            SET_ARRAY_SIZE( new_weights, new_n_tag_points,
                            new_n_tag_points+1, 10 );
            SET_ARRAY_SIZE( new_structure_ids, new_n_tag_points,
                            new_n_tag_points+1, 10 );
            SET_ARRAY_SIZE( new_patient_ids, new_n_tag_points,
                            new_n_tag_points+1, 10 );

            SET_ARRAY_SIZE( new_labels, new_n_tag_points,
                            new_n_tag_points+1, 10 );
            ALLOC( new_labels[new_n_tag_points], strlen(labels[i])+1 );

            /*--- copy from the input tags to the new tags */

            new_tags1[new_n_tag_points][0] = tags1[i][0];
            new_tags1[new_n_tag_points][1] = tags1[i][1];
            new_tags1[new_n_tag_points][2] = tags1[i][2];
            new_weights[new_n_tag_points] = weights[i];
            new_structure_ids[new_n_tag_points] = structure_ids[i];
            new_patient_ids[new_n_tag_points] = patient_ids[i];
            (void) strcpy( new_labels[new_n_tag_points], labels[i] );

            /*--- increment the number of new tags */

            ++new_n_tag_points;
        }
    }

    /*--- output the new tags, the subset of the input tags */

    if( output_tag_file( "output.tag", "Removed negative X's",
                         1, new_n_tag_points, new_tags1, NULL,
                         new_weights, new_structure_ids,
                         new_patient_ids, new_labels ) != OK )
        return( 1 );

    return( 0 );
}
\end{verbatim}
}

\chapter{Transforms}

In dealing with such tasks as inter- and intra-subject registration
both within and across imaging modalities, the concept of
transformations between different coordinate systems arises.  A module
is provided to handle both linear (affine) and non-linear
transformations, and to provide input and output in a standardized
Brain Imaging Centre format, usually to filenames with the extension
\name{.xfm}.

To support these functions, two structure types are provided.  The
first (\name{Transform}) is a four by four linear (affine) transform,
which facilitates rigid transformations, consisting of scaling, rotation,
translation, and shearing.  The second is a higher level transform
(\name{General\_transform}), which represents either a linear
transform, a non-linear transform (thin-plate spline), a user
definable transform, or a concatenation of two or more of these.

\section{Linear Transforms}

The linear transform functions all deal with objects whose type is
\name{Transform}.

{\bf\begin{verbatim}
public  void  make_identity_transform(
    Transform   *transform )
\end{verbatim}}

\desc{Creates a four by four identity transform.}

{\bf\begin{verbatim}
public  void  make_scale_transform(
    Real        sx,
    Real        sy,
    Real        sz,
    Transform   *transform )
\end{verbatim}}

\desc{Creates a scaling transform with separate scaling in each
dimension.  Transforming the point ( 1.0, 1.0, 1.0 ) by the resulting
transform will result in the point (\name{sx}, \name{sy}, \name{sz}).}

{\bf\begin{verbatim}
public  void  make_translation_transform(
    Real        x_trans,
    Real        y_trans,
    Real        z_trans,
    Transform   *transform )
\end{verbatim}}

\desc{Creates a translation transform with separate translation in each
dimension.  Transforming the point ( 0.0, 0.0, 0.0 ) by the resulting
transform will result in the point (\name{x\_trans}, \name{y\_trans},
\name{z\_trans}).}

{\bf\begin{verbatim}
public  void  make_rotation_transform(
    Real       radians,
    int        axis,
    Transform  *transform )
\end{verbatim}}

\desc{Creates a transform which achieves a rotation about one of the
three axes, (\name{X}, \name{Y}, or \name{Z}).}

{\bf\begin{verbatim}
public  void  transform_point(
    Transform  *transform,
    Real       x,
    Real       y,
    Real       z,
    Real       *x_trans,
    Real       *y_trans,
    Real       *z_trans )
\end{verbatim}}

\desc{Transforms a three dimensional point by the given transform,
passing back the three transformed coordinates.}

{\bf\begin{verbatim}
public  void  transform_vector(
    Transform  *transform,
    Real       x,
    Real       y,
    Real       z,
    Real       *x_trans,
    Real       *y_trans,
    Real       *z_trans )
\end{verbatim}}

\desc{Transforms a three dimensional vector by the given transform,
passing back the three transformed coordinates.  The only difference
between transforming a point and a vector is that transforming a
vector does not involve the translational component of the transform.}

{\bf\begin{verbatim}
public  void  inverse_transform_point(
    Transform  *transform,
    Real       x,
    Real       y,
    Real       z,
    Real       *x_trans,
    Real       *y_trans,
    Real       *z_trans )
\end{verbatim}}

\desc{Assuming the transform is an orthogonal transform (no shear
components), the point is transformed by the inverse of the
transform.}

{\bf\begin{verbatim}
public  void  inverse_transform_vector(
    Transform  *transform,
    Real       x,
    Real       y,
    Real       z,
    Real       *x_trans,
    Real       *y_trans,
    Real       *z_trans )
\end{verbatim}}

\desc{Assuming the transform is an orthogonal transform (no shear
components), the vector is transformed by the inverse of the
transform.}

{\bf\begin{verbatim}
public  void   concat_transforms(
    Transform   *result,
    Transform   *t1,
    Transform   *t2 )
\end{verbatim}}

\desc{Multiplies the two transforms \name{t1} and \name{t2}, storing the
product in \name{result}.  Transforming a point by \name{result} is
equivalent to transforming the point by \name{t1}, then transforming
the result by \name{t2}.}

{\bf\begin{verbatim}
public  void   compute_inverse_of_orthogonal_transform(
    Transform  *transform,
    Transform  *inverse )
\end{verbatim}}

\desc{Given a transform that has no shear component (consists of
scales, rotations, and translations), the inverse is computed.}

\section{General Transforms}

General transforms can represent linear transforms, non-linear
(thin-plate splines) transforms, user defined transforms, and
concatenations of these.  All functions dealing with general
transforms involve objects of type \name{General\_transform}.

{\bf\begin{verbatim}
public  void  create_linear_transform(
    General_transform   *transform,
    Transform           *linear_transform )
\end{verbatim}}

\desc{Creates a general transform consisting of a single linear
transform, specified by \name{linear\_transform}.}

{\bf\begin{verbatim}
public  void  create_thin_plate_transform(
    General_transform    *transform,
    int                  n_dimensions,
    int                  n_points,
    float                **points,
    float                **displacements )
\end{verbatim}}

\desc{Creates a general transform consisting of a thin plate spline,
which provides a smooth non-linear mapping of a multidimensional
space.  The \name{points} argument is an array of size
\name{n\_points} by \name{n\_dimensions}, representing a set of points.
The \name{displacements} is a (\name{n\_points} + \name{n\_dimensions}
+ 1) by \name{n\_dimensions} array, which is created by the following
function:}

{\bf\begin{verbatim}
public void  get_nonlinear_warp(
    float **transformed_points,
    float **original_points,
    float **displacements,
    int   n_points,
    int   n_dimensions )
\end{verbatim}}

\desc{Fills in the \name{displacements} array with the appropriate
values to achieve a thin-plate spline transformation.  The mapping
will provide a smooth transformation of space such that each point in
the \name{original\_points} array maps exactly to the corresponding point in
the \name{transformed\_points} array.  These displacements are then
fed into the function \name{create\_thin\_plane\_transform} to create
the general transform.}

{\bf\begin{verbatim}
public  Transform_types  get_transform_type(
    General_transform   *transform )
\end{verbatim}}

\desc{Returns the general transform type, one of \name{LINEAR},
\name{THIN\_PLATE\_SPLINE}, \name{USER\_TRANSFORM}, or
\name{CONCATENATED\_TRANSFORM}.}

{\bf\begin{verbatim}
public  Transform  *get_linear_transform_ptr(
    General_transform   *transform )
\end{verbatim}}

\desc{If the general transform is of type \name{LINEAR}, then returns
a pointer to the linear transform (of type \name{Transform}), for use
with the routines specific to linear transforms, described earlier.
Otherwise prints an error message.}

\section{Reading and Writing General Transforms}

General transforms are stored in files in an ascii format devised at
the McConnell Brain Imaging Centre, and usually have a filename
extension of \name{.xfm}.  The input and output functions are:

{\bf\begin{verbatim}
public  Status  output_transform(
    FILE                *file,
    char                comments[],
    General_transform   *transform )
public  Status  output_transform_file(
    char                filename[],
    char                comments[],
    General_transform   *transform )
\end{verbatim}}

\desc{These two functions write the general transform to a file, in the
appropriate format.  The \name{comments} line is an arbitrary string which is
stored in the file for documentation purposes.  The first form of the
function assumes
the file is already open
and will later be closed.  The second form opens the file, with a
default extension of \name{.xfm}, writes the transform, then closes
the file.  Both functions return \name{ERROR} or \name{OK}.}

{\bf\begin{verbatim}
public  Status  input_transform(
    FILE                *file,
    General_transform   *transform )
public  Status  input_transform_file(
    char                filename[],
    General_transform   *transform )
\end{verbatim}}

\desc{Inputs a general transform from a file.  The first form assumes
the file has already been opened for input, and will later be closed.
The second form opens the file for input, with a default extension of
\name{.xfm}, reads the transform, then closes the file.  Both functions
return \name{ERROR} or \name{OK}.}

\chapter{Final Source Code Example}

This is an example which attempts to illustrate a typical processing
task incorporating use of volumes, tag points, and transformations.
This is a full program which, when linked to the \vol,
reads two volumes in MINC format, a set of tag points from file, and a
transformation from file.  The tag points are assumed to be in the
world space of the first volume, and the transformation is assumed to
transform points in the world space of the first volume to the world
space of the second volume.  The program transforms each tag point by
the transformation input (presumably transforming from the world space
of the first volume to that of the second volume), then transforms the
result to the voxel space of the second volume.  If the
voxel coordinate is within the second volume, then the value of the
corresponding voxel is printed.

{\small
\begin{verbatim}
#include  <volume_io.h>

int  main(
    int   argc,
    char  *argv[] )
{
    int                 v1, v2, v3, sizes[MAX_DIMENSIONS];
    Real                x_world2, y_world2, z_world2;
    Real                voxel2[MAX_DIMENSIONS];
    Real                voxel_value;
    Volume              volume1, volume2;
    int                 i, n_volumes, n_tag_points;
    int                 *structure_ids, *patient_ids;
    Real                **tags1, **tags2, *weights;
    char                **labels;
    General_transform   transform;

    /*--- input the two volumes */

    if( input_volume( "volume1.mnc", 3, NULL, NC_UNSPECIFIED, FALSE,
            0.0, 0.0, TRUE, &volume1,
            (minc_input_options *) NULL ) != OK )
        return( 1 );

    if( input_volume( "volume2.mnc", 3, NULL, NC_UNSPECIFIED, FALSE,
            0.0, 0.0, TRUE, &volume2,
            (minc_input_options *) NULL ) != OK )
        return( 1 );

    /*--- input the tag points */

    if( input_tag_file( "tags_volume1.tag", &n_volumes, &n_tag_points,
                        &tags1, &tags2, &weights, &structure_ids,
                        &patient_ids, &labels ) != OK )
        return( 1 );

    /*--- input the general transform */

    if( input_transform_file( "vol1_to_vol2.xfm", &transform ) != OK )
        return( 1 );

    /*--- convert each tag point */

    get_volume_sizes( volume2, sizes );

    for_less( i, 0, n_tag_points )
    {
        /*--- transform the tag points from volume 1 to volume 2
              world space */

        general_transform_point( &transform,
                                 tags1[i][X], tags1[i][Y], tags1[i][Z],
                                 &x_world2, &y_world2, &z_world2 );

        /*--- transform from volume 2 world space to
              volume 2 voxel space */

        convert_world_to_voxel( volume2, x_world2, y_world2, z_world2,
                                voxel2 );

        /*--- convert voxel coordinates to voxel indices */

        v1 = ROUND( voxel2[0] );
        v2 = ROUND( voxel2[1] );
        v3 = ROUND( voxel2[2] );

        /*--- check if voxel indices inside volume */
     
        if( v1 >= 0 && v1 < sizes[0] &&
            v2 >= 0 && v2 < sizes[1] &&
            v3 >= 0 && v3 < sizes[2] )
        {
            GET_VALUE_3D( voxel_value, volume2, v1, v2, v3 );

            print( "The value for tag point %d (%s) is: %g\n",
                   i, labels[i], voxel_value );
        }
        else
            print( "The tag point %d (%s) is outside.\n" );
    }

    /*--- free up memory */

    delete_volume( volume1 );
    delete_volume( volume2 );
    free_tag_points( n_volumes, n_tag_points, tags1, tags2,
                 weights, structure_ids, patient_ids, labels );
    delete_general_transform( &transform );

    return( 0 );
}
\end{verbatim}
}

\end{document}
