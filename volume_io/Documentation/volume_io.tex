\documentstyle{report}

\title{Programmer's Reference for the MNI Volume IO Library}
\author{David MacDonald}

\newcommand{\path}{{\bf\tt /nil/david/Release/MNI\_lib}}
\newcommand{\name}[1]{{\bf\tt #1}}

\newcommand{\desc}[1]{\begin{tabular}{p{2cm}p{10cm}}
& #1 \end{tabular}}

\begin{document}

\maketitle

\newpage

\tableofcontents

\newpage

\chapter{Introduction}

This document describes a library of routines available at the
McConnell Brain Imaging Centre at the Montreal Neurological Institute.
It was developed as part of a medical imaging software testbed by
David MacDonald, with considerable input from Peter Neelin, Louis
Collins, and others at the Centre.
The library comprises a set of functions for reading and writing
volumes of medical imaging data, as well as some general support
routines useful to many programming tasks.  All images at the Brain
Imaging Centre are stored on disk in a format called {\bf MINC}, which
stands for {\bf Medical Image Net CDF}.  More information on this
format is available in related documentation concerning the format and
the MINC library of routines used to read and write MINC files.  The
Volume IO library is a library built on top of the MINC library to
provide easy access to MINC files, without having to learn too much of
the details of the MINC library.  It provides a structure for internal
storage of volumes and routines to access and modify volumes.  In
addition, it provides routines to manipulate tag points and
transformations and to perform input and output on these objects in
the standardized formats of the Brain Imaging Centre.

This document describes where to find the Volume IO library, how to
integrate it into a user's programs, and what functionality
is provided.  The library is written in C source code, and is designed
to be linked with C source.  It makes calls to three other libraries:
\name{MINC}, \name{netcdf}, and \name{recipes}, the library associated
with the Numerical Recipes handbook.

\chapter{Compiling and Linking}

The relevent directory is \path.  In this directory is the library,
\name{libvolume\_io.a}, and the related include files in the directory
\name{Include}.  In order to compile a program using the Volume IO
library, it is necessary to not only specify this include directory on
the compile line, but also the include directories for the three
libraries called by Volume IO, \name{MINC}, \name{netcdf}, and
\name{recipes}.  The path for these is most likely \name{/usr/local/include},
which is automatically in the search path, so the compile line would look like:
{\bf\begin{verbatim}
        cc -prototypes -I/nil/david/Release/MNI_lib/Include \
           -c test_volume_io.c
\end{verbatim}}

Source files making calls to Volume IO functions must have the
following line at the top, to include the relevent type definitions and
prototypes:

{\bf\begin{verbatim}
        #include  <volume_io.h>
\end{verbatim}}

In order to link with Volume IO, the relevant libraries must be
specified:
{\bf\begin{verbatim}
        cc test_volume_io.o  -o test_volume_io \
           -L/nil/david/Release/MNI_lib -lmni \
           -lminc -lnetcdf -lrecipes -lm -lsun -lmalloc
\end{verbatim}}
The three libraries are usually in /usr/local/lib, which is
automatically in the search path.  The \name{-lm} option includes the
math library, which is sometimes called by the Volume IO library.  The
\name{-lsun} library provides network support useful to all programs.
The \name{-lmalloc} library provides faster and more robust memory
allocation than the default.

\chapter{Types and Macros}

There are several types and macros defined for used with the Volume IO
library.  All functions in the library are prefixed with either the
word \name{public} or \name{private}, which indicates whether the
function is accessible from outside the file in which it resides.
They are defined as follows:
{\bf\begin{verbatim}
        #define  public
        #define  private   static
\end{verbatim}}

A type for logical values is defined:
{\bf\begin{verbatim}
        typedef  int    BOOLEAN
        #define  FALSE  0
        #define  TRUE   1
        #define  OFF    FALSE
        #define  ON     TRUE
\end{verbatim}}

Other useful types defined include:
{\bf\begin{verbatim}
        typedef  double             Real;
        #define  MAX_STRING_LENGTH  200
        typedef  char               STRING[MAX_STRING_LENGTH+1];
        typedef  enum 
             { OK, ERROR, INTERNAL_ERROR, END_OF_FILE, QUIT }
                                    Status;
        typedef  char               Smallest_int;
\end{verbatim}}

\newcommand{\definemacro}[2]{
\begin{flushleft}
\begin{tabular}{p{2cm}p{10cm}}
\multicolumn{2}{l}{{\bf\tt #1}} \\
& #2 \\
\end{tabular}\end{flushleft}}

Some macros useful for general programming include:

\definemacro{SIZEOF\_STATIC\_ARRAY( array )}
{returns the number of elements in a statically allocated array, for example,}
{\bf\begin{verbatim}
        {
            int  size;
            int  array[] = { 1, 2, 3 };

            size = SIZEOF_STATIC_ARRAY( array );     /* == 3 */
        }
\end{verbatim}}

\definemacro{ROUND( x )}{returns the nearest integer to the value.  If
halfway in between, returns the higher of the two.}

\definemacro{IS\_INT( x )}{returns \name{TRUE} if the real argument is exactly an integer.}

\definemacro{FRACTION( x )}{returns the fractional part of the argument.}

\definemacro{FLOOR( x )}{returns the largest integer less than or
equal to the argument.}

\definemacro{CEILING( x )}{returns the smallest integer greater than or
equal to the argument.}

\definemacro{ABS( x )}{returns the absolute value of an integer or real.}

\definemacro{MAX( x, y )}{returns the maximum of two integers or reals.}

\definemacro{MAX3( x, y, z )}{returns the maximum of three integers or
reals.}

\definemacro{MIN( x, y )}{returns the minimum of two integers or reals.}

\definemacro{MIN3( x, y, z )}{returns the minimum of three integers or
reals.}

\definemacro{INTERPOLATE( alpha, a, b )}{returns the interpolation between
\name{a} and \name{b}, where \name{alpha} is in the range zero to
one.}

\definemacro{PI}{returns the value of $\pi$.}

\definemacro{DEG\_TO\_RAD}{returns the number of radians per degrees,
used to multiply an angle in degrees to result in radians.}

\definemacro{RAD\_TO\_DEG}{returns the number of degrees per radian,
used to multiply an angle in radians to result in degrees.}

\definemacro{IJ( i, j, nj )}{converts the indices $\left[ i, j \right]$
of a 2-D \name{ni} by \name{nj} array into a single index, based on
row-major order.}

\definemacro{IJK( i, j, k, nj, nk )}{converts the indices
$\left[ i , j , k \right]$ of a 3-D \name{ni} by \name{nj} by
\name{nk} array into a single index, based on row-major order.}

\definemacro{for\_less( i, start, end )}{performs a for loop where \name{i}
starts at \name{start} and increments until it is greater than or
equal to \name{end}}

\definemacro{for\_inclusive( i, start, end )}{performs a for loop where \name{i}
starts at \name{start} and increments until it is greater than \name{end}}

\definemacro{GLUE(x,y)}{Special C source macro to stick two different
identifiers together, i.e., \name{GLUE(colour,\_name)} results in
\name{colour\_name}.}

\definemacro{GLUE3(x,y,z)}{Special C source macro to stick three different
identifiers together, i.e., \name{GLUE(a,b,c)} results in \name{abc}.}

\chapter{Programming Utilities}

A set of functions which are useful for general purpose, portable
programming are provided.  These include such basic areas as strings,
time, file IO, etc.  Many parts of the Volume IO library refer
to the programming utilities, and it is advisable that users of the
Volume IO library try to use these functions whenever convenient.
The following is a list of all the programming utilities, grouped into
related areas, and ordered alphabetically.

\section{Automatic File Decompression}

The unix command \name{compress} is often used to reduce file space by
compressing files into new files ending in \name{.Z}.  The Volume IO
library includes routines for automatically decompressing such files.
This is done by inserting new definitions of such file operations as
\name{fopen}, \name{fread}, and \name{fscanf}.  Use of this facility
is transparent, as long as the Volume IO include directory is specified
correctly on the compile line, and the executable is linked with the
Volume IO library.

\section{Argument Processing}

Simple command line argument processing is provided.  Note that more
sophisticated and extensive argument processing can be provided by
other existing packages around the lab.  Nevertheless, command line
argument lists can be scanned in command line order as follows:

{\bf\begin{verbatim}
public  void  initialize_argument_processing(
    int     argc,
    char    *argv[] )
\end{verbatim}}

\desc{called once to let the argument processor know what the command line
arguments are.  Initializes the current argument to the first
argument, in preparation for sequential processing of the command line
arguments.}

{\bf\begin{verbatim}
public  BOOLEAN  get_int_argument(
    int   default_value,
    int   *value )
public  BOOLEAN  get_real_argument(
    Real  default_value,
    Real  *value )
public  BOOLEAN  get_string_argument(
    char   default_value[],
    char   *value[] )
\end{verbatim}}

\desc{checks if an argument exists and is of the desired type.  If so,
then it assigns the appropriate \name{value}, advances to the next
argument, and returns \name{TRUE}.  Otherwise, the default value is assigned
to the \name{value} and the current argument is not advanced.}

{\bf\begin{verbatim}
public  BOOLEAN  get_prefix_argument(
    char  prefix[] )
\end{verbatim}}

\desc{checks if the next argument starts with the given prefix and if so,
advances to the point in the argument just after the end of the prefix, or
to the next argument, if there are no characters after the prefix.
Returns \name{TRUE} if the command line argument starts with the prefix.}

\section{Colour Manipulation}

Colours are generally represented as triples of coordinates,
representing red, green, and blue components.  Sometimes, an opacity
is included as a fourth component, to determine how much the colour
will let other objects show through.  Sometimes it is useful
to work with colours in a different space, Hue-Saturation-Light, which
is also a triple of coordinates.  Each colour component is usually either a
real value in the range zero to one, or an integral value in the range
zero to 255.  A colour type is defined (\name{Colour}) which uses a
single long integer to store colours as four eight-bit integers.
Various macros and functions exist to manipulate these colours:

\definemacro{make\_Colour( r, g, b )}{Returns a value of type
\name{Colour} which contains the specified red, green, and blue
components, each of which must be within the range zero to 255.}

\definemacro{make\_rgba\_Colour( r, g, b, a )}{Returns a value of type
\name{Colour} which contains the specified red, green, blue, and alpha
(opacity) components, each of which must be within the range zero to 255.}

\definemacro{get\_Colour\_r( col )}{Given an argument of type
\name{Colour}, returns the value of the red component, in the range
zero to 255.}

\definemacro{get\_Colour\_g( col )}{Given an argument of type
\name{Colour}, returns the value of the green component, in the range
zero to 255.}

\definemacro{get\_Colour\_b( col )}{Given an argument of type
\name{Colour}, returns the value of the blue component, in the range
zero to 255.}

\definemacro{get\_Colour\_a( col )}{Given an argument of type
\name{Colour}, returns the value of the alpha component, in the range
zero to 255.}

\definemacro{make\_Colour\_0\_1( r, g, b )}{Returns a value of type
\name{Colour} which contains the specified red, green, and blue
components, each of which must be within the range zero to 1.0.}

\definemacro{make\_rgba\_Colour\_0\_1( r, g, b )}{Returns a value of type
\name{Colour} which contains the specified red, green, blue, and
alpha components, each of which must be within the range zero to 1.0.}

\definemacro{get\_Colour\_r\_0\_1( col )}{Given an argument of type
\name{Colour}, returns the value of the red component, in the range
zero to 1.0.}

\definemacro{get\_Colour\_g\_0\_1( col )}{Given an argument of type
\name{Colour}, returns the value of the green component, in the range
zero to 1.0.}

\definemacro{get\_Colour\_b\_0\_1( col )}{Given an argument of type
\name{Colour}, returns the value of the blue component, in the range
zero to 1.0.}

\definemacro{get\_Colour\_a\_0\_1( col )}{Given an argument of type
\name{Colour}, returns the value of the alpha component, in the range
zero to 1.0.}

\definemacro{get\_Colour\_luminance( col )}{Given an argument of type
\name{Colour}, returns the brightness of the colour, in the range zero
to 255.  This is used to convert colour to gray-scale.}

A surface property type is also defined, to provide descriptions of
surface lighting characteristics:

\definemacro{fill\_Surfprop( spr, a, d, s, e, o )}{Fills in the
\name{spr} structure, which is of type \name{Surfprop}, with the five
parameters:  ambient coefficient, diffuse coefficient, specular
coefficient, specular exponent, and opacity.}

\definemacro{Surfprop\_a( spr )}{Returns the ambient coefficient, a
number in the range of zero to one.}

\definemacro{Surfprop\_d( spr )}{Returns the diffuse coefficient, a
number in the range of zero to one.}

\definemacro{Surfprop\_s( spr )}{Returns the specular coefficient, a
number in the range of zero to one.}

\definemacro{Surfprop\_se( spr )}{Returns the specular exponent, a
number in the range of zero to infinity, typically less than 100.}

\definemacro{Surfprop\_t( spr )}{Returns the opacity (inverse of
transparency), a number in the range of zero to one.}

Several functions are available which deal with colours and surface
properties:

{\bf\begin{verbatim}
public  void  rgb_to_hsl(
    Real    r,
    Real    g,
    Real    b,
    Real    *h,
    Real    *s,
    Real    *l )
public  void  hsl_to_rgb(
    Real   h,
    Real   sl,
    Real   l,
    Real   *r,
    Real   *g,
    Real   *b )
public  void  convert_colour_to_hsl(
    Colour   rgb,
    Colour   *hsl )
public  void  convert_colour_to_rgb(
    Colour   hsl,
    Colour   *rgb )
\end{verbatim}}

\desc{Converts between red-green-blue and hue-saturation-light space.}

{\bf\begin{verbatim}
public  BOOLEAN  equal_colours(
    Colour  col1,
    Colour  col2 )
\end{verbatim}}

\desc{returns \name{TRUE} if the two colours are equal.}

{\bf\begin{verbatim}
public  int  get_colour_distance(
    int      r,
    int      g,
    int      b,
    Colour   c2 )
\end{verbatim}}

\desc{finds the squared Euclidean distance between the first colour,
specified by its components, and the second colour, \name{c2}.}

{\bf\begin{verbatim}
public  int   find_closest_colour(
    int     r,
    int     g,
    int     b,
    int     n_colours,
    Colour  colours[] )
\end{verbatim}}

\desc{Given the three components of a target colour, and a list of
\name{n\_colours} colours, returns the index in the list of the
closest colour to the target colour.}

{\bf\begin{verbatim}
public  BOOLEAN  lookup_colour(
    char    colour_name[],
    Colour  *col )
\end{verbatim}}

\desc{Given a colour name, such as ``\name{red}'' or ``\name{dark\_blue}''
looks up the name in a list of colours and passes back the
corresponding colour, returning \name{TRUE} if successful.  The
supported colours are from the list of about 300 named X colours.}

{\bf\begin{verbatim}
public  BOOLEAN  lookup_colour_name(
    Colour  col,
    char    colour_name[] )
\end{verbatim}}

\desc{This performs the inverse of the \name{lookup\_colour()}
function, taking a colour and seeing if it matches one of the known
named colours.  If so, the colour name is copied into the argument
\name{colour\_name}, and \name{TRUE} is returned.}

\section{General File IO}

Although one may use the standard UNIX file interface (\name{fprintf},
for example), the Volume IO library contains a set of routines for
doing all file operations, with the potential to be portable to
other operating systems.  There are some other minor advantages,
including the automatic expansion of \name{\~\ } and environment
variables.

{\bf\begin{verbatim}
public  void  expand_filename(
    char  filename[],
    char  expanded_filename[] )
\end{verbatim}}

\desc{Searches the argument \name{filename} for certain patterns, and
expands them appropriately, storing the resulting expanded filename in
the second argument, \name{expanded\_filename}.  Any sequence of
characters starting with a \name{\~\ } up to but not including a slash,
\name{\/}, will be changed to the current users home directory.  Note
that this will only work properly if the \name{\~\ } is used alone, or
in conjunction with the current user's user name.  For example,
if user \name{john} passes this routine a filename of
\name{\~\ frank/text}, it will be expanded to \name{/usr/people/john/text},
which is not what was desired.  Any occurrence of a dollar sign,
\name{\$}, will result in the following text, up to the next slash,
being expanded to the environment variable specified by the text.  This
expansion can be avoided by placing a backslash, \name{$\backslash$},
before any \name{\~\ } or \name{\$}.  In all the following functions which
take filenames, automatic expansion of filenames is performed.}

{\bf\begin{verbatim}
public  BOOLEAN  filename_extension_matches(
    char   filename[],
    char   extension[] )
\end{verbatim}}

\desc{Returns \name{TRUE} if the file ends in the given extension, preceded
by a period. e.g.,
\name{filename\_extension\_matches("volume.mnc","mnc")} returns \name{TRUE}.
Correctly handles compressed files, e.g.,
\name{filename\_extension\_matches("volume.mnc.Z","mnc")} returns \name{TRUE}.}

{\bf\begin{verbatim}
public  void  remove_directories_from_filename(
    char  filename[],
    char  filename_no_directories[] )
\end{verbatim}}

\desc{Strips the directories from the filename, placing the result in
the second argument, \name{filename\_no\_directories}.}

{\bf\begin{verbatim}
public  void  extract_directory(
    char    filename[],
    char    directory[] )
\end{verbatim}}

\desc{Extracts the directory from the filename, placing the result in
the second argument, \name{directory}.}

{\bf\begin{verbatim}
public  void  get_absolute_filename(
    char    filename[],
    char    directory[],
    char    abs_filename[] )
\end{verbatim}}

\desc{Given a filename and a current directory, constructs and
absolute filename (one starting with a slash, \name{/}.}

{\bf\begin{verbatim}
public  BOOLEAN  file_exists(
    char        filename[] )
\end{verbatim}}

\desc{Returns \name{TRUE} if the file exists, \name{FALSE} otherwise.  The filename
may contain the home directory character, \name{\~\ }, and environment
variables, such as \name{\$PET\_IMAGE}.}

{\bf\begin{verbatim}
public  void  remove_file(
    char  filename[] )
\end{verbatim}}

\desc{Removes the specified file.  The filename
may contain the home directory character, \name{\~\ }, and environment
variables, such as \name{\$PET\_IMAGE}.}

{\bf\begin{verbatim}
public  void  unlink_file(
    char        filename[] )
\end{verbatim}}

\desc{Unlinks the specified file, which results in the file being
removed only after all other references to it are closed.}

{\bf\begin{verbatim}
public  Status  open_file(
    char               filename[],
    IO_types           io_type,
    File_formats       file_format,
    FILE               **file )
public  Status  open_file_with_default_suffix(
    char               filename[],
    char               default_suffix[],
    IO_types           io_type,
    File_formats       file_format,
    FILE               **file )
\end{verbatim}}

\desc{The function \name{open\_file()} opens the specified file, where
\name{io\_type} must be one of
\name{WRITE\_FILE} or \name{READ\_FILE}, and
\name{file\_format} must be one of \name{ASCII\_FORMAT} or
\name{BINARY\_FORMAT}.  If successful, the file pointer is passed back
in the last argument and a status of \name{OK} is returned.
Otherwise, a null pointer is passed back, and a status of \name{ERROR}
is returned.  Filename expansion is automatically performed.  The
second function, \name{open\_file\_with\_default\_suffix()}, performs the
same task as \name{open\_file} with the addition that it automatically
adds the specified suffix extension, if needed.  On input, if the
specified file does not exist and the file does not have an extension,
then it looks for the specified file with the default extension.}

{\bf\begin{verbatim}
public  Status  close_file(
    FILE     *file )
\end{verbatim}}

\desc{Closes the file, returning \name{OK} if successful.}

{\bf\begin{verbatim}
public  Status  set_file_position(
    FILE     *file,
    long     byte_position )
\end{verbatim}}

\desc{Seeks to the specified position in the file, where 0 corresponds
to the beginning of the file.}

{\bf\begin{verbatim}
public  Status  flush_file(
    FILE     *file )
\end{verbatim}}

\desc{Flushes the file buffer of any pending output.}

{\bf\begin{verbatim}
public  Status  input_character(
    FILE  *file,
    char   *ch )
\end{verbatim}}

\desc{Inputs a character from the file and passes it back as the
second argument.  If the character is the end
of file character, returns a status of \name{ERROR}, otherwise, \name{OK}.}

{\bf\begin{verbatim}
public  Status  unget_character(
    FILE  *file,
    char  ch )
\end{verbatim}}

\desc{Places the character back on the input queue.}

{\bf\begin{verbatim}
public  Status  input_nonwhite_character(
    FILE   *file,
    char   *ch )
\end{verbatim}}

\desc{Inputs the next non-white character, i.e. the next character
that is neither a blank, a tab, or a newline character.}

{\bf\begin{verbatim}
public  Status   skip_input_until(
    FILE   *file,
    char   search_char )
\end{verbatim}}

\desc{Inputs characters from the file until the specified search
character is found.}

{\bf\begin{verbatim}
public  Status  input_string(
    FILE  *file,
    char  str[],
    int   string_length,
    char  termination_char )
\end{verbatim}}

\desc{Inputs a string from the file, where the maximum storage of the
string is specified by the argument \name{string\_length}.  The file
is read into the string until either the termination character or and
newline character is found.  If the string ends at an newline
character and the termination character is not an newline, the
next character available from the file will be the newline
character.  Otherwise, the next character available is the one just
after the termination character found.
}

{\bf\begin{verbatim}
public  Status  input_quoted_string(
    FILE            *file,
    char            str[],
    int             str_length )
\end{verbatim}}

\desc{Inputs a string from a file, delimited by quotation marks,
returning \name{OK} or \name{ERROR}.  The
maximum storage available in the \name{str} pointer is specified by
the third argument, \name{str\_length}.   After successful reading of
a string, the next character available from the file is the one just
after the ending quotation mark.}

{\bf\begin{verbatim}
public  Status  input_line(
    FILE    *file,
    char    line[],
    int     str_length )
\end{verbatim}}

\desc{Inputs characters from the file into the argument \name{line},
up until the next newline character, or until \name{str\_length}
characters have been read.  If a newline was found, the next available
character will be the one after the newline.}

{\bf\begin{verbatim}
public  Status  input_boolean(
    FILE            *file,
    BOOLEAN         *b )
\end{verbatim}}

\desc{Inputs the next non-white character.
If it is a ``\name{f}'' or ``\name{F}'', then the value
\name{FALSE} is passed back.
If it is a ``\name{t}'' or ``\name{T}'', then the value
\name{TRUE} is passed back.  Otherwise \name{ERROR} is returned.}

{\bf\begin{verbatim}
public  Status  input_short(
    FILE            *file,
    short           *s )
public  Status  input_unsigned_short(
    FILE            *file,
    unsigned short  *s )
public  Status  input_int(
    FILE  *file,
    int   *i )
public  Status  input_real(
    FILE            *file,
    Real            *r )
public  Status  input_float(
    FILE            *file,
    float           *f )
ublic  Status  input_double(
    FILE            *file,
    double          *d )
\end{verbatim}}

\desc{Inputs a value of the specified type from an ascii file.}

{\bf\begin{verbatim}
public  Status  input_newline(
    FILE            *file )
\end{verbatim}}

\desc{Inputs and discards characters from the file up to and including
the next newline character.  Returns \name{OK} if a newline was found,
or \name{ERROR} if the end of file was reached first.}

{\bf\begin{verbatim}
public  Status  input_binary_data(
    FILE            *file,
    void            *data,
    size_t          element_size,
    int             n )
\end{verbatim}}

\desc{Inputs an array of data from a file, in binary format.  The
array contains \name{n} elements, each of size \name{element\_size}.
Returns either \name{OK} or \name{ERROR}.}

{\bf\begin{verbatim}
public  Status  output_character(
    FILE   *file,
    char   ch )
\end{verbatim}}

\desc{Outputs a character to the file, returning \name{OK} or
\name{ERROR}.}

{\bf\begin{verbatim}
public  Status  output_string(
    FILE  *file,
    char  str[] )
\end{verbatim}}

\desc{Outputs the specified string to the file.}

{\bf\begin{verbatim}
public  Status  output_quoted_string(
    FILE            *file,
    char            str[] )
\end{verbatim}}

\desc{Outputs a quotation mark, the specified string, and a closing
quotation mark.}

{\bf\begin{verbatim}
public  Status  output_newline(
    FILE            *file )
\end{verbatim}}

\desc{Outputs a newline character, returning \name{OK} or \name{ERROR}.}

{\bf\begin{verbatim}
public  Status  output_boolean(
    FILE            *file,
    BOOLEAN         b )
\end{verbatim}}

\desc{If the argument is \name{TRUE}, then a space and the letter
``\name{T}'' is output, otherwise, a space and the letter ``\name{F}''
is output.
}

{\bf\begin{verbatim}
public  Status  output_short(
    FILE            *file,
    short           s )
public  Status  output_unsigned_short(
    FILE            *file,
    unsigned short  s )
public  Status  output_int(
    FILE            *file,
    int             i )
public  Status  output_real(
    FILE            *file,
    Real            r )
public  Status  output_float(
    FILE            *file,
    float           f )
public  Status  output_double(
    FILE            *file,
    double          d )
\end{verbatim}}

\desc{Outputs the specified value to an ascii file.}

{\bf\begin{verbatim}
public  Status  output_binary_data(
    FILE            *file,
    void            *data,
    size_t          element_size,
    int             n )
\end{verbatim}}

\desc{Outputs an array of data from a file, in binary format.  The
array contains \name{n} elements, each of size \name{element\_size}.
Returns either \name{OK} or \name{ERROR}.}

{\bf\begin{verbatim}
public  Status  io_binary_data(
    FILE            *file,
    IO_types        io_flag,
    void            *data,
    size_t          element_size,
    int             n )
\end{verbatim}}

\desc{Inputs or outputs the specified binary data, depending on
whether the argument \name{io\_flag} is \name{READ\_FILE} or
\name{WRITE\_FILE}.}

{\bf\begin{verbatim}
public  Status  io_newline(
    FILE            *file,
    IO_types        io_flag,
    File_formats    format )
\end{verbatim}}

\desc{Inputs or outputs an ascii newline character, depending on 
whether the argument \name{io\_flag} is \name{READ\_FILE} or
\name{WRITE\_FILE}.  If the \name{format} argument is
\name{BINARY\_FORMAT}, this function does nothing.}

{\bf\begin{verbatim}
public  Status  io_quoted_string(
    FILE            *file,
    IO_types        io_flag,
    File_formats    format,
    char            str[],
    int             str_length )
\end{verbatim}}

\desc{If the \name{format} argument is \name{ASCII\_FORMAT},
inputs or outputs a quotation mark delimited string, depending
on whether the argument \name{io\_flag} is \name{READ\_FILE} or
\name{WRITE\_FILE}.  IF the \name{format} argument is
\name{BINARY\_FORMAT}, inputs or outputs a string preceded by an
integer indicating the length of the string.}

{\bf\begin{verbatim}
public  Status  io_boolean(
    FILE            *file,
    IO_types        io_flag,
    File_formats    format,
    BOOLEAN         *b )
public  Status  io_short(
    FILE            *file,
    IO_types        io_flag,
    File_formats    format,
    short           *short_int )
public  Status  io_unsigned_short(
    FILE            *file,
    IO_types        io_flag,
    File_formats    format,
    unsigned short  *unsigned_short )
ublic  Status  io_unsigned_char(
    FILE            *file,
    IO_types        io_flag,
    File_formats    format,
    unsigned  char  *c )
public  Status  io_int(
    FILE            *file,
    IO_types        io_flag,
    File_formats    format,
    int             *i )public  Status  io_real(
    FILE            *file,
    IO_types        io_flag,
    File_formats    format,
    Real            *r )
public  Status  io_float(
    FILE            *file,
    IO_types        io_flag,
    File_formats    format,
    float           *f )
public  Status  io_double(
    FILE            *file,
    IO_types        io_flag,
    File_formats    format,
    double          *d )
\end{verbatim}}

\desc{Inputs or outputs a binary or ascii value of the specified type.}

{\bf\begin{verbatim}
public  Status  io_ints(
    FILE            *file,
    IO_types        io_flag,
    File_formats    format,
    int             n,
    int             *ints[] )
\end{verbatim}}

\desc{Inputs or outputs an array of integers in binary or ascii
format.}

{\bf\begin{verbatim}
public  Status  io_unsigned_chars(
    FILE            *file,
    IO_types        io_flag,
    File_formats    format,
    int             n,
    unsigned char   *unsigned_chars[] )
\end{verbatim}}

\desc{Inputs or outputs an array of unsigned characters in binary or ascii
format.}

\section{Global Variables}

This module provides support for defining global variables in such a
way that there is a lookup table of global variable names.  This table
can be used to provide reading of global variables from files at
runtime, or to allow the user to query and modify global variable
values during program execution.  In order to use this facility, the
programmer must have (either directly or indirectly)
{\bf\begin{verbatim}
#include  <globals.h>
\end{verbatim}}
in every program file that refers to a global variable.  The
global variables must be defined in a file called
\name{global\_variables.h} which has the following form:
{\bf\begin{verbatim}
START_GLOBALS
    DEF_GLOBAL(        Mine_1, BOOLEAN,  TRUE )
    DEF_GLOBAL(        Mine_2, int,      3 )
    DEF_GLOBAL(        Mine_3, Real,     2.5 )
    DEF_GLOBAL(        Mine_4, Colour,   RED )
    DEF_GLOBAL5(       Mine_5, Surfprop, 0.3, 0.6, 0.6, 40.0, 1.0 )
    DEF_GLOBAL3(       Mine_6, Vector,   1.0, 0.0, 0.0 )
    DEF_GLOBAL3(       Mine_7, Point,    2.0, 2.0, 2.0 )
    DEF_GLOBAL_STRING( Mine_8, "Initial value" )
END_GLOBALS
\end{verbatim}}
The first argument of each of these macros is the name of the global
variable.  The second argument is the type of the global variable.
Note that only the types in the previous example are supported.  The
final argument or arguments are the initial values of the global
variable.

Finally, one needs to create the lookup table in one program file,
usually the main program file, as follows:
{\bf\begin{verbatim}
#define  GLOBALS_LOOKUP_NAME  my_globals
#include  <globals.h>
\end{verbatim}}
which creates a global variable lookup table called
\name{my\_globals}.  Relevent functions for global variables are:


{\bf\begin{verbatim}
public  Status  input_globals_file(
    int             n_globals_lookup,
    global_struct   globals_lookup[],
    char            filename[] )
\end{verbatim}}

\desc{Inputs global variable assignments from the specified file.
Assignments are of the form \name{variable = value ;}.  The lookup
table and its size are passed to this function.  An example
follows: }
{\bf\begin{verbatim}
    if( input_globals_file( SIZEOF_STATIC_ARRAY(my_globals),
           my_globals, filename ) != OK )
    {
        print( "Error reading globals file.\n" );
    }
\end{verbatim}}

{\bf\begin{verbatim}
public  Status  get_global_variable(
    int              n_globals_lookup,
    global_struct    globals_lookup[],
    char             variable_name[],
    char             value[] )
\end{verbatim}}

\desc{Creates a string containing the value of the global variable
matching the specified name, \name{variable\_name}.  For example, if
the global variable \name{my\_variable} is of type integer,}
{\bf\begin{verbatim}
    if( get_global_variable( SIZEOF_STATIC_ARRAY(my_globals),
            my_globals, "my_variable", value_string ) == OK )
    {
        print( "Value is: %s\n", value_string );
    }
\end{verbatim}}
is equivalent to:
{\bf\begin{verbatim}
    print( "Value is: %d\n", my_variable );
\end{verbatim}}

{\bf\begin{verbatim}
public  Status  set_global_variable(
    int              n_globals_lookup,
    global_struct    globals_lookup[],
    char             variable_name[],
    char             value_to_set[] )
\end{verbatim}}

\desc{Sets the value of the global variable matching the specified name,
\name{variable\_name}, to the value specified by the ascii string
\name{value\_to\_set}.  For example, if the global variable
\name{my\_variable} is of type integer,}
{\bf\begin{verbatim}
    if( set_global_variable( SIZEOF_STATIC_ARRAY(my_globals),
            my_globals, "my_variable", "45" ) != OK )
    {
        print( "Error setting variable.\n" );
    }
\end{verbatim}}
is equivalent to
{\bf\begin{verbatim}
    my_variable = 45;
\end{verbatim}}

{\bf\begin{verbatim}
public  Status  set_or_get_global_variable(
    int              n_globals_lookup,
    global_struct    globals_lookup[],
    char             input_str[],
    char             variable_name[],
    char             value_string[] )
\end{verbatim}}

\desc{If the \name{input\_str} is of the form
``\name{variable = value}'', then the corresponding global variable is
set to the value, and the value is passed back in ascii form in the
\name{value\_string} argument.  Otherwise, the form of
\name{input\_str} is simply ``\name{variable}'' and the corresponding
global variable is not changed, but its current value is passed back
in the \name{value\_string} argument.  This routine is useful in many
programs to read input lines from the user, allowing the user to query
and set global variable values during program execution.}

\section{Memory Allocation}

A set of macros is provided to allow easy allocation and deallocation
of memory, with up to 5 dimensional arrays.  Memory allocation
checking is also performed, to catch errors such as freeing memory
that was not allocated.  Also, the memory allocation automatically
tracks all memory allocated, so that detection of memory leaks
(orphaned memory) is possible.

\subsection{Basic Memory Allocation}

The basic macros are as follows:

{\bf\begin{verbatim}
     ALLOC( ptr, n_items )
\end{verbatim}}

\desc{Allocates \name{n\_items} elements of the correct type,
assigning the result to the argument \name{ptr}.}

{\bf\begin{verbatim}
    FREE( ptr )
\end{verbatim}}

\desc{Frees the memory pointed to by the argument \name{ptr}.}

{\bf\begin{verbatim}
    REALLOC( ptr, n_items )
\end{verbatim}}

\desc{Changes the size of the memory pointed to by \name{ptr} to be of
size \name{n\_items} elements, possibly changing the value of
\name{ptr} in the process.}

{\bf\begin{verbatim}
    ALLOC_VAR_SIZED_STRUCT( ptr, element_type, n_elements )
\end{verbatim}}

\desc{Allocates a variable sized structure, which must be of a
specific form.  The last element of the structure must be an array of
size 1, and this array will constitute the variable-sized part of the
structure.  The argument \name{element\_type} must be the type of this
last element, and the argument \name{n\_elements} is the desired
number of elements to allocate for this array, in addition to the
memory for the first part of the structure.  An example of usage is:}

\begin{verbatim}
        {
            struct {  int   a;
                      float b;
                      char  data[1];
                   }                    *ptr;

            ALLOC_VAR_SIZED_STRUCT( ptr, char, 10 );
        }
\end{verbatim}

{\bf\begin{verbatim}
    ALLOC2D( ptr, n1, n2 )
    ALLOC3D( ptr, n1, n2, n3 )
    ALLOC4D( ptr, n1, n2, n3, n4 )
    ALLOC5D( ptr, n1, n2, n3, n4, n5 )
\end{verbatim}}

\desc{Allocates a 2 to 5 dimensional array of size \name{n1} by \name{n2},
etc.  and stores the result in the specified pointer, \name{ptr}.  In
the 2 dimensional case, this is
accomplished with only 2 memory allocations, one to allocate \name{n1}
times \name{n2} elements for the storage, and the second to allocate
\name{n1} pointers into the first memory area.  In general, there is
one memory allocation for each dimension required.}

{\bf\begin{verbatim}
    FREE2D( ptr )
    FREE3D( ptr )
    FREE4D( ptr )
    FREE5D( ptr )
\end{verbatim}}

\desc{Frees the memory associated with the multi-dimensional array.}

\subsection{Higher Level Array Allocation}

In addition to the basic memory allocation macros described previously,
a set of useful macros for dealing with arrays of dynamically changing
size.

{\bf\begin{verbatim}
    SET_ARRAY_SIZE( array, previous_n_elems, new_n_elems,
                    chunk_size )
\end{verbatim}}

\desc{This macro increases or decreases the size of an array, by
specifying the number of elements previously allocated to the array.
The \name{chunk\_size} argument defines the size of the memory chunks which
are allocated.  For instance, if \name{chunk\_size} is 100, then this
macro will only reallocate the array if the size change crosses to a
different multiple of 100, thus avoiding memory allocation every time
it is called.  This specification of the granularity of the memory
allocation must be consistently specified; if this macro is called
with a given variable and chunk size, then subsequent calls to this
macro with same variable must specify the same chunk size.  Note also
that the number passed in as \name{new\_n\_elems} must be passed in as
\name{previous\_n\_elems} on the next call to this macro.}

{\bf\begin{verbatim}
    ADD_ELEMENT_TO_ARRAY( array, n_elems,
                          elem_to_add, chunk_size )
\end{verbatim}}

\desc{Adds the argument \name{elem\_to\_add} to the array at the
\name{n\_elems}'th index, then incrementing \name{n\_elems}. The
argument \name{chunk\_size} specifies the granularity of memory
allocation.}

{\bf\begin{verbatim}
    DELETE_ELEMENT_FROM_ARRAY( array, n_elems, index_to_remove,
                               chunk_size )
\end{verbatim}}

\desc{Deletes the \name{index\_to\_remove}'th element from the array,
decreasing the number of elements in the array (\name{n\_elems}) and
decreasing the memory allocation, if crossing a multiple of
\name{chunk\_size}.  Again, \name{chunk\_size} must be specified the
same for all invocations of the previous three macros involving a
given pointer.}

{\bf\begin{verbatim}
    ADD_ELEMENT_TO_ARRAY_WITH_SIZE( array, n_alloced, n_elems,
                                    elem_to_add, chunk_size )
\end{verbatim}}

\desc{Adds an element (\name{elem\_to\_add}) to the array,
incrementing \name{n\_elems}.  If necessary, the memory is increased
by the amount specified in \name{chunk\_size} and the \name{n\_alloced}
variable is incremented by this amount.  The usage of this differs
from the use of \name{ADD\_ELEMENT\_TO\_ARRAY} in that the number of
elements (\name{n\_elems}) can be decreased arbitrarily, without
causing memory to be deallocated.}

\section{Numerical Utilities}

A small set of useful numerical functions are provided:

{\bf\begin{verbatim}
public  BOOLEAN  numerically_close(
    Real  n1,
    Real  n2,
    Real  threshold_ratio )
\end{verbatim}}

\desc{Checks if the two numbers are within the given threshold ratio.
For instance, \name{numerically\_close( 4113.0, 4112.0, 0.001 )}
returns \name{TRUE}, since the two numbers are with 0.1 percent of
each other.}

{\bf\begin{verbatim}
public  Real  get_good_round_value(
    Real    value )
\end{verbatim}}

\desc{Returns the largest power of 10 or five times a power of 10
which is less then the specified value.  Useful for determining graph
axis positions.}

{\bf\begin{verbatim}
public  int  solve_quadratic(
    Real   a,
    Real   b,
    Real   c,
    Real   *solution1,
    Real   *solution2 )
\end{verbatim}}

\desc{Finds the real roots of the equation $a * x^2 + b * x + c$.
Returns either 0, 1, or 2, indicating the number of unique real roots,
which are passed back either the argument \name{solution1}, or in both
arguments, \name{solution1} and \name{solution2}.}

{\bf\begin{verbatim}
public  int solve_cubic(
    Real   a,
    Real   b,
    Real   c,
    Real   d,
    Real   s[ 3 ] )
\end{verbatim}}

\desc{Finds the real roots of the equation $a * x^3 + b * x^2 + c * x
+ d$.  Returns the number of unique real solutions, and passes them
back in the array \name{s}.}

{\bf\begin{verbatim}
public  Real  evaluate_polynomial(
    int     n,
    Real    poly[],
    Real    u )
\end{verbatim}}

\desc{Efficiently evaluates a polynomial of the form $poly[0] +
poly[1] * u + poly[2] * u^2 ...$, using Horner's rule.}

\section{Progress Reports}

In order to provide simple monitoring of the progress of a particular
processing task, a progress reporting module is available.  While a
task is in progress, the progress report prints dots across the line
indicating how close to finished the task is.  If the task is going to
take very long, (greater than 2 minutes), the progress report
periodically prints the current percentage done and the estimated time
remaining.  An example of usage, followed by function descriptions, is
presented:

\begin{verbatim}
        {
            int                 i, n_slices;
            progress_struct     progress;

            n_slices = 100;

            initialize_progress_report( &progress, FALSE,
                        n_slices, "Processing Slices" );

            for( i = 0;  i < n_slices;  ++i )
            {
                process_slice( i );

                update_progress_report( &progress, i + 1 );
            }

            terminate_progress_report( &progress );
        }
\end{verbatim}

{\bf\begin{verbatim}
public  void  initialize_progress_report(
    progress_struct   *progress,
    BOOLEAN           one_line_only,
    int               n_steps,
    char              title[] )
\end{verbatim}}

\desc{Initializes a progress report struct, specifying the number of
steps that will occur in  the processing, and the title to print out
for the progress report.  During progress report, the display will
automatically switch from the short-job mode of printing a single row
of dots across a line to the long-job mode of periodically printing
the percentage done and estimated time remaining.  If the
\name{one\_line\_only} flag is \name{TRUE}, this is disabled and only
a single row of dots will be displayed.}

{\bf\begin{verbatim}
public  void  update_progress_report(
    progress_struct   *progress,
    int               current_step )
\end{verbatim}}

\desc{Tells the progress reporting module how many steps have been
done, and causes update of the display of dots on the line or
estimated time remaining.}

{\bf\begin{verbatim}
public  void  terminate_progress_report(
    progress_struct   *progress )
\end{verbatim}}

\desc{Terminates the progress report.}

\section{Random Numbers}

A simple random number package is provided, which produces a sequence
of pseudo-random numbers based on a seed value, which totally
determines the entire sequence of numbers.  If the user does not set a
seed, a seed is automatically created by the current time and date.

{\bf\begin{verbatim}
public  void  set_random_seed(
    int seed )
\end{verbatim}}

\desc{Sets the seed which determines the sequence of pseudo-random numbers.}

{\bf\begin{verbatim}
public  int  get_random_int(
    int n )
\end{verbatim}}

\desc{Returns a random integer in the range zero to \name{n-1}.}

{\bf\begin{verbatim}
public  Real  get_random_0_to_1( void )
\end{verbatim}}

\desc{Returns a random number greater than or equal to zero and
strictly less than one.}

\section{Strings}

Some simple string manipulation techniques are provided:

{\bf\begin{verbatim}
public  BOOLEAN  string_ends_in(
    char   string[],
    char   ending[] )
\end{verbatim}}

\desc{Determines if the string ends in the specified \name{ending}.  For
instance, \name{string\_ends\_in( "rainfall", "fall" )} returns
\name{TRUE}.}

{\bf\begin{verbatim}
public    void   strip_blanks(
    char  str[],
    char  stripped[] )
\end{verbatim}}

\desc{Removes leading and trailing blanks from the string, placing the
result in the argument \name{stripped}.  Works correctly even if both
arguments refer to the same string.}

{\bf\begin{verbatim}
public  void  get_upper_case_string(
    char    string[],
    char    upper_case[] )
\end{verbatim}}

\desc{Converts the string to an all upper case string.  Correctly
handles case where the two string arguments point to the same memory.}

{\bf\begin{verbatim}
public  int  find_character(
    char    string[],
    char    ch )
\end{verbatim}}

\desc{Searches for the given character in the given string \,
returning the index where it was found, or -1 if it was not found.}

{\bf\begin{verbatim}
public  BOOLEAN  blank_string(
    char   string[] )
\end{verbatim}}

\desc{Returns true if the string is empty or consists of only space,
tab, and newline characters.}

\section{Text Output}

Rather than using the standard UNIX function \name{printf} for routine
text output, a module which is similar in appearance is provided,
which allows for installing arbitrary printing output functions.  This
can be used, for instance, to mirror output to a log file for
instance, or to send text to an X window.  The primary change is to
use a function called \name{print} with exactly the same arguments as
for the standard \name{printf} function.

{\bf\begin{verbatim}
public  void  print( char format[], ... )
\end{verbatim}}

\desc{Takes the same arguments as \name{printf()}, but allows
installing of a user output function for the final stage of output.}

{\bf\begin{verbatim}
public  void  set_print_function(
    void     (*function) ( char [] ) )
\end{verbatim}}

\desc{Sets the output function, where all text from calls to
\name{print()} will be sent.  By default, there is no print function,
and output is sent to printf().}

{\bf\begin{verbatim}
public  void  push_print_function()
\end{verbatim}}

\desc{Temporarily sets the print function to go to standard out.}

{\bf\begin{verbatim}
public  void  pop_print_function()
\end{verbatim}}

\desc{Restores the previous user print function.}

\section{Time}

Some basic utilities relating to time, date, and CPU time are
provided.

{\bf\begin{verbatim}
public  Real  current_cpu_seconds( void )
\end{verbatim}}

\desc{Returns the number of CPU seconds used by the current process so
far.}

{\bf\begin{verbatim}
public  Real  current_realtime_seconds( void )
\end{verbatim}}

\desc{Returns the number of seconds that the current process has been
running.}

{\bf\begin{verbatim}
public  void  get_clock_time(
    char  time_str[] )
\end{verbatim}}

\desc{Copies the current clock time (hours and minutes) into a string.}

{\bf\begin{verbatim}
public  void  get_date(
    char  date_str[] )
\end{verbatim}}

\desc{Copies the current date into a string.}

{\bf\begin{verbatim}
public  void  format_time(
    char   str[],
    char   format[],
    Real   seconds )
\end{verbatim}}

\desc{Takes a time in seconds and a format string which has format
components, i.e. "%g %s", and prints the time into the string
\name{str}, in appropriate units of milliseconds, seconds, minutes,
days, etc.}

{\bf\begin{verbatim}
public  void  print_time(
    char   format[],
    Real   seconds )
\end{verbatim}}

\desc{Same as \name{format\_time}, but calls \name{print()} with the
result.}

{\bf\begin{verbatim}
public  void  sleep_program(
    Real seconds )
\end{verbatim}}

\desc{Suspends the program for the specified number of seconds.  Note
that on most systems, this will only be performed to the nearest
multiple of some particular time increment.  On Silicon Graphics
Systems this will be to the nearest hundredth of a second.}

\chapter{Volumes}

Processing tasks within the lab where this software was developed
deals with multidimensional volumes of data such as created by
Magnetic Resonance and PET scanners.  Therefore, an extensive set of
routines is provided to represent volumes, and to read and write
volumes in the MINC format.

The basic type of a volume is \name{Volume}, which is actually a
pointer to an allocated structure which contains all the information
about the type of volume, number of dimensions, voxel values, etc.  In
order to use a volume structure, the volume must first be created, then
the size of the volume set, then the large array of voxel values is
allocated.  Alternately, a volume may be automatically created by
calling the appropriate function to read a MINC file and create a
volume.

A volume has an associated dimension, which must be in the range
from one to five, but typically is three.  The volume is thought of as
multidimensional array of any of a variety of types, including all
sizes of integer and real C scalar types.  Even though a volume may be
stored in a 1 byte type, with values from zero to 255, there is an
optional real value mapping, which provides a mapping to any arbitrary
real range.  In this way, the real range may be any valid real
interval and is independent of the particular storage type.

Since most volumes will be created by reading from a MINC file, this
method will be presented first, followed by a description of how to
create a volume from scratch.

\section{Volume Input}

{\bf\begin{verbatim}
public  Status  input_volume(
    char                 filename[],
    int                  n_dimensions,
    char                 *dim_names[],
    nc_type              volume_nc_data_type,
    BOOLEAN              volume_signed_flag,
    Real                 volume_voxel_min,
    Real                 volume_voxel_max,
    BOOLEAN              create_volume_flag,
    Volume               *volume,
    minc_input_options   *options )
\end{verbatim}}

\desc{This routine reads a volume from a MINC file, first creating
the volume if the \name{create\_volume\_flag} is specified.  The
number of dimensions is the desired number of dimensions for the
volume.  If this is less than the number of dimensions in the file,
then only the first part of the file, corresponding to this number of
dimensions is read.  The argument \name{dim\_names} specifies in what
order the volume is to be stored in the volume variable.  For each
dimension in the stored volume, there is a corresponding name, which
is one of \name{MIxspace}, \name{MIyspace}, \name{MIzspace},
\name{ANY\_SPATIAL\_DIMENSION}, or an empty string.  These are matched
up with corresponding dimensions in the file and the dimension
ordering of the volume array is reordered on input.  So, if the user
wishes to represent the volume in X-Z-Y order, then the the value of
the \name{dim\_names} array should be the three strings,
``\name{MIxspace}'', ``\name{MIzspace}'', and ``\name{MIyspace}''.}

\desc{The four arguments, \name{volume\_nc\_data\_type},
\name{volume\_signed\_flag}, \name{volume\_voxel\_min}, and
\name{volume\_voxel\_max} can be used to specified the desired storage
type within the volume variable, automatically converted from the
storage type in the file.  The \name{volume\_nc\_data\_type} is one of
\name{NC\_UNSPECIFIED}, \name{NC\_BYTE}, \name{NC\_SHORT}, \name{NC\_LONG},
\name{NC\_FLOAT}, or \name{NC\_DOUBLE}.
For the integer types, the
\name{volume\_signed\_flag} is \name{TRUE} if a signed type is
desired, otherwise, it is \name{FALSE}.  The \name{volume\_voxel\_min}
and \name{volume\_voxel\_max} specify the range of valid voxel values,
and are usually set equal to indicate to use the full range of the
type, e.g. zero to 255.0 for unsigned \name{NC\_BYTE}.
If \name{NC\_UNSPECIFIED} is passed, then the type, sign. and voxel
range in the file are used.}

\desc{If the \name{create\_volume\_flag} is \name{TRUE}, the usual
case, then the volume is automatically created.  Otherwise, it is
assumed that the volume already exists and will only be recreated if
its current size is different from the new size resulting from the
file, thus reducing the amount of memory allocation when reading
multiple files.}

\desc{The \name{options} argument specifies some special options to
the input process, and is usually passed just a \name{NULL} pointer,
indicating that the default options should be used.  Currently the
possible options are:}

{\bf\begin{verbatim}
public  void  set_default_minc_input_options(
    minc_input_options  *options )
\end{verbatim}}

\desc{Fills in the default options into an \name{options} structure
which will subsequently be passed to \name{input\_volume}.}

{\bf\begin{verbatim}
public  void  set_minc_input_promote_invalid_to_min_flag(
    minc_input_options  *options,
    BOOLEAN             flag )
\end{verbatim}}

\desc{By default, any voxel value which is outside the valid range of
voxel values is promoted to the minimum valid voxel value.  If this
\name{flag} is set to \name{FALSE}, this promotion is disabled.}

{\bf\begin{verbatim}
public  void  set_minc_input_vector_to_scalar_flag(
    minc_input_options  *options,
    BOOLEAN             flag )
\end{verbatim}}

\desc{By default, any volume which contains a dimension which is of
type vector, for instance, an RGB colour volume, is converted to a
scalar volume by averaging the components of each vector.  This can be
disabled by passing the \name{flag} as \name{FALSE}.}

\section{Volume Output}

Volume output is accomplished by one of two routines, depending on
whether or not the volume is treated as a modified version of another
volume or is an independent volume on its own.

{\bf\begin{verbatim}
public  Status  output_volume(
    char                  filename[],
    nc_type               file_nc_data_type,
    BOOLEAN               file_signed_flag,
    Real                  file_voxel_min,
    Real                  file_voxel_max,
    Volume                volume,
    char                  history[],
    minc_output_options   *options )
\end{verbatim}}

\desc{Outputs the specified volume to the specified filename.  If the
argument \name{file\_nc\_data\_type} is \name{NC\_SPECIFIED} then the
volume is stored in the MINC file in the same type as in the volume
variable.  Otherwise, the four arguments, \name{file\_nc\_data\_type},
\name{file\_signed\_flag}, \name{file\_voxel\_min}, and
\name{file\_voxel\_max}, specify the type and valid voxel range to store
the volume in the file.  If the \name{history} argument is non-null,
then it represents a description of the volume, and will be placed in
the MINC volume.  If the \name{options} argument is \name{NULL}, then
the default options will be used.  Otherwise, some specific output
options can be set through this parameter, and the following
functions:}

{\bf\begin{verbatim}
public  void  set_default_minc_output_options(
    minc_output_options  *options )
\end{verbatim}}

\desc{Sets the \name{options} structure to the default values.  The
user can then override the default values and pass the structure to
the \name{output\_volume} function.  Currently, there are no output
options, but this function exists as a place holder for future
revisions.}

If the volume is a modification of another volume currently stored in
a file, then it is more appropriate to use the following function to
output the volume:

{\bf\begin{verbatim}
public  Status  output_modified_volume(
    char                  filename[],
    nc_type               file_nc_data_type,
    BOOLEAN               file_signed_flag,
    Real                  file_voxel_min,
    Real                  file_voxel_max,
    Volume                volume,
    char                  original_filename[],
    char                  history[],
    minc_output_options   *options )
\end{verbatim}}

\desc{The only difference between this function and the other method
of volume output (\name{output\_volume}), is that this function copies
auxiliary data from the original file (\name{original\_filename}) to
the new file.  This auxiliary data includes such items as patient
name and other scanning data, and is not read into a volume, so the
only way to correctly pass it along to a new MINC file is to use this
function.}

\section{Volume Manipulation}

Once a volume has been created and allocated, there are many macros
and functions for manipulating the volume.  Note that associated with
each volume is a valid voxel range indicating the range of values
actually stored in the volume, for instance, zero to 200 is one
possible range for an unsigned byte volume.  There is a second range,
the \name{real} range, which is the mapping of the valid voxel range
to an arbitrary real range, for instance, the zero to 200 of valid
voxels could map to -1.5 to 1.5 in the \name{real} range.  When
dealing with volumes, one is generally interested in the \name{real}
range.

{\bf\begin{verbatim}
    GET_VOXEL( value, volume, v1, v2, v3, v4, v5 )
\end{verbatim}}

\desc{Given a volume and from one to five voxel indices, depending on
the volume dimensions, stores the corresponding voxel value
in the variable \name{value}, which may be of any type.  For instance,
if the volume is three
dimensional, then the final two arguments are ignored.}

{\bf\begin{verbatim}
    GET_VOXEL_1D( value, volume, v1 )
    GET_VOXEL_2D( value, volume, v1, v2 )
    GET_VOXEL_3D( value, volume, v1, v2, v3 )
    GET_VOXEL_4D( value, volume, v1, v2, v3, v4 )
    GET_VOXEL_5D( value, volume, v1, v2, v3, v4, v5 )
\end{verbatim}}

\desc{Given a volume and the required number of voxel indices for the
volume, assigns the corresponding voxel value to the argument
\name{value}.}

{\bf\begin{verbatim}
    CONVERT_VOXEL_TO_VALUE( volume, voxel )
\end{verbatim}}

\desc{Given a volume and a voxel value, converts this to a value in
the real range, and returns it.}

{\bf\begin{verbatim}
    CONVERT_VALUE_TO_VOXEL( volume, value )
\end{verbatim}}

\desc{Given a volume and a real value, converts this to a voxel value in
the valid voxel range, and returns it.}

{\bf\begin{verbatim}
    GET_VALUE( value, volume, v1, v2, v3, v4, v5 )
\end{verbatim}}

\desc{Given a volume and from one to five voxel indices, depending on
the volume dimensions, stores the corresponding real value in the
variable \name{value},
which may be of any type.  For instance, if the volume is three
dimensional, then the final two arguments are ignored.}

{\bf\begin{verbatim}
    GET_VALUE_1D( value, volume, v1 )
    GET_VALUE_2D( value, volume, v1, v2 )
    GET_VALUE_3D( value, volume, v1, v2, v3 )
    GET_VALUE_4D( value, volume, v1, v2, v3, v4 )
    GET_VALUE_5D( value, volume, v1, v2, v3, v4, v5 )
\end{verbatim}}

\desc{Given a volume and the required number of voxel indices for the
volume, assigns the corresponding real value to the argument
\name{value}.  This is the most efficient method to access the volume
values (as compared to the \name{GET\_VALUE} macro, for instance).
Typically, the \name{GET\_VALUE\_3D} macro is used for
most applications.}

{\bf\begin{verbatim}
    SET_VOXEL( volume, v1, v2, v3, v4, v5, value )
\end{verbatim}}

\desc{Given a volume, one to five voxel indices, and a \name{voxel}
value, assigns this value to the corresponding voxel in the volume.
Note that no conversion from the valid real range to the valid voxel
range is performed, so the user may need to use the
\name{CONVERT\_VALUE\_TO\_VOXEL} macro first.}

{\bf\begin{verbatim}
public  void  delete_volume(
    Volume   volume )
\end{verbatim}}

\desc{Deletes all the memory associated with an volume.}

{\bf\begin{verbatim}
public  nc_type  get_volume_nc_data_type(
    Volume       volume,
    BOOLEAN      *signed_flag )
\end{verbatim}}

\desc{Returns the storage type of the volume (e.g. \name{NC\_SHORT}),
and passes back an indication of whether it is signed or unsigned.}

{\bf\begin{verbatim}
public  int  get_volume_n_dimensions(
    Volume   volume )
\end{verbatim}}

\desc{Returns the number of dimensions of the volume.}

{\bf\begin{verbatim}
public  void  get_volume_sizes(
    Volume   volume,
    int      sizes[] )
\end{verbatim}}

\desc{Stores the size of each dimension in the array \name{sizes}.
This is the number of voxels in each dimension.  The array
\name{sizes} must
be at least as large as the number of dimensions of the volume.}

{\bf\begin{verbatim}
public  void  get_volume_separations(
    Volume   volume,
    Real     separations[] )
\end{verbatim}}

\desc{Stores the slice separation for each dimension in the array
\name{separations}.  The array \name{separations} must
be at least as large as the number of dimensions of the volume.}

{\bf\begin{verbatim}
public  Real  get_volume_voxel_min(
    Volume   volume )
public  Real  get_volume_voxel_max(
    Volume   volume )
public  void  get_volume_voxel_range(
    Volume     volume,
    Real       *voxel_min,
    Real       *voxel_max )
\end{verbatim}}

\desc{The first two functions return the minimum or maximum allowable
voxel value.  The third function passes back both values.}

{\bf\begin{verbatim}
public  Real  get_volume_real_min(
    Volume     volume )
public  Real  get_volume_real_max(
    Volume     volume )
public  void  get_volume_real_range(
    Volume     volume,
    Real       *min_value,
    Real       *max_value )
\end{verbatim}}

\desc{The first two functions return the minimum or maximum real
value.  The third function passes back both values.  The mapping to
this real space linearly maps the minimum \name{voxel} value to the minimum
\name{real} value and the maximum \name{voxel} value to the maximum
\name{real} value.}

{\bf\begin{verbatim}
public  char  **get_volume_dimension_names(
    Volume   volume )
\end{verbatim}}

\desc{Returns a pointer to the list of names of each dimension.}

\subsection{Volume Coordinate Systems}

A volume has two coordinate systems.  The {\bf voxel} coordinate
system is simply the n-dimensional indexing coordinate system for a
volume.  A voxel coordinate of (0.0, 0.0, 0.0), for instance, corresponds
to the centre of the first voxel in a three dimensional volume.  A voxel
coordinate of ( 99.0, 0.0, 0.0 ) corresponds to the centre of the last
voxel in the first direction of a three dimensional volume of size (
100, 200, 150 ).  The second coordinate system is an arbitrary
three dimensional coordinate system, usually referred to as the
{\bf world} coordinate system, often
the Talairach coordinate system.  The following functions provide
conversion to and from these two coordinate systems:

{\bf\begin{verbatim}
public  void  convert_voxel_to_world(
    Volume   volume,
    Real     voxel[],
    Real     *x_world,
    Real     *y_world,
    Real     *z_world )
\end{verbatim}}

\desc{Given a volume and a real valued voxel index, passes back the
corresponding world coordinate.}

{\bf\begin{verbatim}
public  void  convert_3D_voxel_to_world(
    Volume   volume,
    Real     voxel1,
    Real     voxel2,
    Real     voxel3,
    Real     *x_world,
    Real     *y_world,
    Real     *z_world )
\end{verbatim}}

\desc{Same as \name{convert\_voxel\_to\_world} except it applies only
to three dimensional volumes.}

{\bf\begin{verbatim}
public  void  convert_world_to_voxel(
    Volume   volume,
    Real     x_world,
    Real     y_world,
    Real     z_world,
    Real     voxel[] )
\end{verbatim}}

\desc{Converts a world coordinate into a voxel.  In order to use
these voxel coordinates as integer indices, for instance, as arguments
to the \name{GET\_VALUE} macros, each component of the argument
\name{voxel} must first be rounded to the nearest integer.}

{\bf\begin{verbatim}
public  void  convert_3D_world_to_voxel(
    Volume   volume,
    Real     x_world,
    Real     y_world,
    Real     z_world,
    Real     *voxel1,
    Real     *voxel2,
    Real     *voxel3 )
\end{verbatim}}

\desc{Same as \name{convert\_world\_to\_voxel} except it applies only
to three dimensional volumes.}

{\bf\begin{verbatim}
public  void  convert_voxel_normal_vector_to_world(
    Volume          volume,
    Real            voxel1,
    Real            voxel2,
    Real            voxel3,
    Real            *x_world,
    Real            *y_world,
    Real            *z_world )
\end{verbatim}}

\desc{Converts a vector which is assumed to be a surface normal to the
world coordinate system.}

\section{Volume Creation from Scratch}

In some circumstances, it is desireable to create volumes in ways
other than reading from a file.  The following functions provide
methods to create a volume from scratch or to create a volume which is
similar to an existing volume.

{\bf\begin{verbatim}
public   Volume   create_volume(
    int         n_dimensions,
    char        *dimension_names[],
    nc_type     nc_data_type,
    BOOLEAN     signed_flag,
    Real        voxel_min,
    Real        voxel_max )
\end{verbatim}}

\desc{Creates and returns a volume of the given type (e.g.
\name{NC\_BYTE}, \name{signed\_flag} equal \name{FALSE}), and given
valid voxel range.  The \name{dimension\_names} is used to describe
each dimension of the volume and is currently only used when writing
the volume to a file.}

{\bf\begin{verbatim}
public  void  set_volume_voxel_range(
    Volume   volume,
    Real     voxel_min,
    Real     voxel_max )
public  void  set_volume_real_range(
    Volume   volume,
    Real     real_min,
    Real     real_max )
\end{verbatim}}

\desc{After creation of a volume, the valid voxel range or valid real
range can subsequently be changed by using these functions.}

{\bf\begin{verbatim}
public  void  set_volume_sizes(
    Volume       volume,
    int          sizes[] )
\end{verbatim}}

\desc{Sets the sizes of the volume, the number of voxels in each
dimension.  Note that this must be done before calling the function
\name{alloc\_volume\_data} to allocate the voxels.}

{\bf\begin{verbatim}
public  void  alloc_volume_data(
    Volume   volume )
\end{verbatim}}

\desc{After the volume has been created, and its size has been set,
then this function allocates the memory for the voxels.  Note that the
voxel values are not initialized, and the user must fill the volume
with desired values.}

Associated with each volume is a transformation from \name{voxel}
space to \name{world} space.  There are several ways to define this
transformation.  The simplest is just to specify it directly:

{\bf\begin{verbatim}
public  void  set_voxel_to_world_transform(
    Volume             volume,
    General_transform  *transform )
\end{verbatim}}

\desc{Assigns the given transformation to the volume.}

{\bf\begin{verbatim}
public  void  set_volume_separations(
    Volume   volume,
    Real     separations[] )
\end{verbatim}}

\desc{Sets the inter-voxel separations in each of the volume
dimensions.  Note that this will cause the voxel to world
transformation to be updated accordingly.}

{\bf\begin{verbatim}
public  void  set_volume_translation(
    Volume  volume,
    Real    voxel[],
    Real    world_space_voxel_maps_to[] )
\end{verbatim}}

\desc{Sets the translation portion of the voxel to world
transformation.  A voxel coordinate is specified (\name{voxel}), as
well as a real world position to which it is desired that this voxel
map (\name{world\_space\_voxel\_maps\_to}).  The voxel to world
transformation is updated to provide this mapping.}

{\bf\begin{verbatim}
public  void  set_volume_direction_cosine(
    Volume   volume,
    int      dimension,
    Real     dir[] )
\end{verbatim}}

\desc{Sets the real world axis for a specific voxel dimension.  For
instance if \name{dimension} is 1, and \name{dir} is (0.0, 1.0, 1.0),
then voxels along the second dimension of the volume will map to the real
world axis (0.0, 1.0, 1.0) normalized to unit length, then scaled by
the volume separation for the second volume dimension.}

\subsection{Copying Volumes}

Another method of creating a volume is to simply copy the entire
volume definition from an existing volume:

{\bf\begin{verbatim}
public  Volume   copy_volume_definition(
    Volume   existing_volume,
    nc_type  nc_data_type,
    BOOLEAN  signed_flag,
    Real     voxel_min,
    Real     voxel_max )
\end{verbatim}}

\desc{Creates and returns a new volume which has the same definition
(sizes, voxel to world space transform, etc.), as an existing volume.
If the argument \name{nc\_data\_type} is not \name{NC\_UNSPECIFIED},
then the storage type of the new volume differs from the original and
is specified by \name{nc\_data\_type}, \name{signed\_flag},
\name{voxel\_min}, and \name{voxel\_max}.  The voxel values are allocated
but not initialized.}

\section{Source Code Example}

An examples of reading, writing, and manipulating volumes is
presented here.

\subsection{Program to Read, Modify, and Write Volumes}

This is a complete program to read a MINC volume, change all values
over 100.0 to 100.0, then write out the result in a new file.

{\small
\begin{verbatim}
#include  <volume_io.h>

int  main(
    int   argc,
    char  *argv[] )
{
    int        v1, v2, v3, sizes[MAX_DIMENSIONS];
    Real       voxel_value, voxel_100;
    Volume     volume;

    /*--- input the volume */

    if( input_volume( "volume.mnc", 3, NULL, NC_UNSPECIFIED, FALSE,
            0.0, 0.0, TRUE, &volume,
            (minc_input_options *) NULL ) != OK )
        return( 1 );

    get_volume_sizes( volume, sizes );

    /*--- change all values over 100 to 100 */

    voxel_100 = CONVERT_VALUE_TO_VOXEL( volume, 100.0 );

    for( v1 = 0;  v1 < sizes[0];  ++v1 ) {
        for( v2 = 0;  v2 < sizes[1];  ++v2 ) {
            for( v3 = 0;  v3 < sizes[2];  ++v3 ) {
                GET_VALUE_3D( voxel_value, volume, v1, v2, v3 );

                if( voxel_value > 100.0 ) {
                    SET_VOXEL_3D( volume, v1, v2, v3, voxel_100 );
                }
            }
        }
    }

    /*--- output the modified volume */

    if( output_modified_volume( "output.mnc", NC_UNSPECIFIED,
             FALSE, 0.0, 0.0, volume, "volume.mnc",
             "Modified by clamping to 100",
             (minc_output_options *) NULL ) != OK )
        return( 1 );

    return( 0 );
}
\end{verbatim}
}

\chapter{Tag Points}

Tag points are sets of three dimensional points which are used to mark
positions of interest.  Generally these result from manually locating
corresponding positions in multiple volumes to provide a mapping among
the volumes.  Tag points are stored in an ascii format devised at the
McConnell Brain Imaging Centre, with convention of filenames ending in
\name{.tag}.  A tag file can contain either sets of single tag points
or two tag points, depending on whether the file corresponds to one or
two volumes.  Each tag point consists of three coordinates, x, y, and
z.  Each set of one or two tag points has additional optional
information stored with it.  A tag point set may or may not include a
set of three values, a real valued weight, an integer structure id,
and an integer patient id.  A tag point may or may not include a label
string.  Functions are provided to read and write files of this
format are provided:

{\bf\begin{verbatim}
public  Status  input_tag_points(
    FILE      *file,
    int       *n_volumes,
    int       *n_tag_points,
    Real      ***tags_volume1,
    Real      ***tags_volume2,
    Real      **weights,
    int       **structure_ids,
    int       **patient_ids,
    char      ***labels )
public  Status  input_tag_file(
    char      filename[],
    int       *n_volumes,
    int       *n_tag_points,
    Real      ***tags_volume1,
    Real      ***tags_volume2,
    Real      **weights,
    int       **structure_ids,
    int       **patient_ids,
    char      ***labels )
\end{verbatim}}

\desc{These two functions input a set of tag points from a file.  The
first form assumes the file is already open and will be closed by the
calling function, while the second form
opens and closes the file specified by a filename, with a default
extension of \name{.tag}.  The number of
volumes (number of tags in a set, currently either one or two) is
passed back in the \name{n\_volumes} argument.  The number of tag
points is passed back in the \name{n\_tag\_points} argument.  The
three dimensional coordinates of the first tag point in each set is
passed back in the argument \name{tags\_volume1}.  If the number of
volumes is two, then the second tag point in each set is passed back
in the argument \name{tags\_volume2}.  The final four arguments pass
back the auxiliary information associated with each tag point set.  If
the calling program is not interested in any one of the four data,
then it can pass in a \name{NULL} pointer and the values in the file
will not be passed back.}

{\bf\begin{verbatim}
public  void  free_tag_points(
    int       n_volumes,
    int       n_tag_points,
    Real      **tags_volume1,
    Real      **tags_volume2,
    Real      weights[],
    int       structure_ids[],
    int       patient_ids[],
    char      **labels )
\end{verbatim}}

\desc{When finished with a list of tag points, the
associated memory may be freed by calling this function.}

{\bf\begin{verbatim}
public  Status  output_tag_points(
    FILE      *file,
    char      comments[],
    int       n_volumes,
    int       n_tag_points,
    Real      **tags_volume1,
    Real      **tags_volume2,
    Real      weights[],
    int       structure_ids[],
    int       patient_ids[],
    char      **labels )
public  Status  output_tag_file(
    char      filename[],
    char      comments[],
    int       n_volumes,
    int       n_tag_points,
    Real      **tags_volume1,
    Real      **tags_volume2,
    Real      weights[],
    int       structure_ids[],
    int       patient_ids[],
    char      **labels )
\end{verbatim}}

\desc{These two functions output a list of tag points to a tag point
file.  The first form assumes that the file has already been opened
and will be closed later by the calling function, while the second
form opens and closes the file, with the default extension of
\name{.tag}.  The \name{comments} argument is any arbitrary string
documenting the contents of this file.  The number of volumes
(\name{n\_volumes}) must be either one or two.  The number of tag
points is specified by the argument \name{n\_tag\_points}.  The actual
positions of the sets of one or two tag points are in the arguments
\name{tags\_volume1} and \name{tags\_volume2}.  If any of the three
arguments, \name{weights}, \name{structure\_ids}, or
\name{patient\_ids}, are specified as \name{NULL}, then none of these
three pieces or information are written to the file.  Similarly, if
the \name{labels} argument is \name{NULL}, then no labels are written
to the file.}

\section{Source Code Example}

The following is an example which reads a tag volume, removes all tags
which have negative x positions, ignores the second tag point in each
set, if present, and writes the result to a new file.

{\small
\begin{verbatim}
#include  <volume_io.h>

int  main(
    int   argc,
    char  *argv[] )
{
    int        i, n_volumes, n_tag_points, *structure_ids, *patient_ids;
    Real       **tags1, **tags2, *weights;
    char       **labels;
    int        new_n_tag_points, *new_structure_ids, *new_patient_ids;
    Real       **new_tags1, *new_weights;
    char       **new_labels;

    /*--- input the tag file */

    if( input_tag_file( "input_tags.tag", &n_volumes, &n_tag_points,
                        &tags1, &tags2, &weights, &structure_ids,
                        &patient_ids, &labels ) != OK )
        return( 1 );

    /*--- create a new tag point list of only those tag points
          whose x coordinate is nonnegative */

    new_n_tag_points = 0;

    for_less( i, 0, n_tag_points )
    {
        if( tags1[i][0] >= 0.0 )
        {
            /*--- increase the memory allocation of the tag points */

            SET_ARRAY_SIZE( new_tags1, new_n_tag_points,
                            new_n_tag_points+1, 10 );
            ALLOC( new_tags1[new_n_tag_points], 3 );

            SET_ARRAY_SIZE( new_weights, new_n_tag_points,
                            new_n_tag_points+1, 10 );
            SET_ARRAY_SIZE( new_structure_ids, new_n_tag_points,
                            new_n_tag_points+1, 10 );
            SET_ARRAY_SIZE( new_patient_ids, new_n_tag_points,
                            new_n_tag_points+1, 10 );

            SET_ARRAY_SIZE( new_labels, new_n_tag_points,
                            new_n_tag_points+1, 10 );
            ALLOC( new_labels[new_n_tag_points], strlen(labels[i])+1 );

            /*--- copy from the input tags to the new tags */

            new_tags1[new_n_tag_points][0] = tags1[i][0];
            new_tags1[new_n_tag_points][1] = tags1[i][1];
            new_tags1[new_n_tag_points][2] = tags1[i][2];
            new_weights[new_n_tag_points] = weights[i];
            new_structure_ids[new_n_tag_points] = structure_ids[i];
            new_patient_ids[new_n_tag_points] = patient_ids[i];
            (void) strcpy( new_labels[new_n_tag_points], labels[i] );

            /*--- increment the number of new tags */

            ++new_n_tag_points;
        }
    }

    /*--- output the new tags, the subset of the input tags */

    if( output_tag_file( "output.tag", "Removed negative X's",
                         1, new_n_tag_points, new_tags1, NULL,
                         new_weights, new_structure_ids,
                         new_patient_ids, new_labels ) != OK )
        return( 1 );

    return( 0 );
}
\end{verbatim}
}

\chapter{Transforms}

\chapter{Source Code Examples}

\section{Reading and Writing MINC Volumes}

\section{Creating Tag Point Files}

\section{Creating Transformation Files}

\section{Volumes, Tag Points, and Transformations}

\end{document}
