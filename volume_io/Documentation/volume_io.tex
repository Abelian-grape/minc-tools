\documentstyle{article}

\title{Programmer's Reference for the MNI Volume IO Library}
\author{David MacDonald}

\newcommand{\path}{{\bf\tt /nil/david/Release/MNI\_lib}}
\newcommand{\name}[1]{{\bf\tt #1}}

\begin{document}

\maketitle

\newpage

\tableofcontents

\newpage

\section{Introduction}

This document describes a library of routines available at the
McConnell Brain Imaging Centre at the Montreal Neurological Institute.
The library comprises a set of functions for reading and writing
volumes of medical imaging data, as well as some general support
routines useful to many programming tasks.  All images at the Brain
Imaging Centre are stored on disk in a format called {\bf MINC}, which
stands for {\bf Medical Image Net CDF}.  More information on this
format is available in related documentation concerning the format and
the MINC library of routines used to read and write MINC files.  The
Volume IO library is a library built on top of the MINC library to
provide easy access to MINC files, without having to learn too much of
the details of the MINC library.  It also provides a structure for internal
storage of volumes and routines to access and modify volumes.

This document describes where to find the Volume IO library, how to
integrate it into a user's programs, and what functionality
is provided.  The library is written in C source code, and is designed
to be linked with C source.  It makes calls to three other libraries:
\name{MINC}, \name{netcdf}, and \name{recipes}, the library associated
with the Numerical Recipes handbook.

\section{Compiling and Linking}

The relevent directory is \path.  In this directory is the library,
\name{libvolume\_io.a}, and the related include files in the directory
\name{Include}.  In order to compile a program using the Volume IO
library, it is necessary to not only specify this include directory on
the compile line, but also the include directories for the three
libraries called by Volume IO, \name{MINC}, \name{netcdf}, and
\name{recipes}.  The path for these is most likely \name{/usr/local/include},
which is automatically in the search path, so the compile line would look like:
\begin{verbatim}
    cc -prototypes -I/nil/david/Release/MNI_lib/Include \
       -c test_volume_io.c
\end{verbatim}

Source files making calls to Volume IO functions must have the
following line at the top, to include the relevent type definitions and
prototypes:
\begin{verbatim}
    #include  <volume_io.h>
\end{verbatim}

In order to link with Volume IO, the relevant libraries must be
specified:
\begin{verbatim}
    cc test_volume_io.o  -o test_volume_io \
       -L/nil/david/Release/MNI_lib -lmni \
       -lminc -lnetcdf -lrecipes -lm -lsun -lmalloc
\end{verbatim}
The three libraries are usually in /usr/local/lib, which is
automatically in the search path.  The \name{-lm} option includes the
math library, which is sometimes called by the Volume IO library.  The
\name{-lsun} library provides network support useful to all programs.
The \name{-lmalloc} library provides faster and more robust memory
allocation than the default.

\section{Types and Macros}

There are several types and macros defined for used with the Volume IO
library.  All functions in the library are prefixed with either the
word \name{public} or \name{private}, which indicates whether the
function is accessible from outside the file in which it resides.
They are defined as follows:
\begin{verbatim}
#define  public
#define  private   static
\end{verbatim}

A type for logical values is defined:
\begin{verbatim}
typedef  int    BOOLEAN
#define  FALSE  0
#define  TRUE   1
#define  OFF    FALSE
#define  ON     TRUE
\end{verbatim}

Other useful types defined include:
\begin{verbatim}
typedef  double               Real;
#define  MAX_STRING_LENGTH    200
typedef  char                 STRING[MAX_STRING_LENGTH+1];
typedef  enum 
        { OK, ERROR, INTERNAL_ERROR, END_OF_FILE, QUIT } Status;
typedef  char                 Smallest_int;
\end{verbatim}

Some macros useful for general programming include:

\name{SIZEOF\_STATIC\_ARRAY( array )}:
returns the number of elements in a staticly allocated array,
\begin{verbatim}
{
    int  size;
    int  array[] = { 1, 2, 3 };

    size = SIZEOF_STATIC_ARRAY( array );     /* == 3 */
}
\end{verbatim}

\name{ABS( x )}:  returns the absolute value of an integer or real.

\name{MAX( x, y )}:  returns the maximum of two integers or reals.

\section{Programming Utilities}

\subsection{Automatic File Decompression}

\subsection{Argument Processing}

\subsection{Colour Manipulation}

\subsection{General File IO}

\subsection{Global Variables}

\subsection{Memory Allocation}

\subsection{Numerical Utilities}

\subsection{Progress Reports}

\subsection{Random Numbers}

\subsection{Strings}

\subsection{Text Output}

\subsection{Time}

\section{Volume Manipulation}

\subsection{Volume Input}

\subsection{Volume Output}

\end{document}
