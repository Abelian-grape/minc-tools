\documentclass{article}

\usepackage{times}

\title{MINC 2.0 User's Guide}
\author{John G. Sled \\
Robert D. Vincent \\
Leila Baghdadi}
\date{\today}
\begin{document}
\maketitle
\newpage
\tableofcontents
\newpage
\section{Introduction to MINC 2}
MINC is a software system for storing and manipulating medical images,
originally developed in 1993 by Peter Neelin at the McConnell Brain
Imaging Centre.  The name MINC is an acronym for Medical Imaging
NetCDF.  MINC was conceived as a means to allow researchers to use a
common set of tools and files to work with medical images in a variety
of modalities.  The file format was originally defined as a
specialization of the NetCDF (Network Common Data Form) file format
created by the Unidata Program Center at UCAR (Univerity Corporation
for Atmospheric Research).  The NetCDF format, libraries, and tools
were created to store generic datasets of arbitrary dimensionality.
NetCDF was chosen because it implements many of the functions that
were envisioned for the MINC system.

Like most other medical imaging data formats, MINC allows medical
image data to take on a wide range of data types or ranges, and
defines a set of standard supporting data describing the image
acquistion parameters or patient details.

However, MINC is different from most other medical imaging formats in
several respects:
\begin{itemize}
\item MINC is inherently {\it N-dimensional}.  MINC data can be
structured with any number of spatial, temporal, or other dimensions,
and these dimensions may be organized an an arbitrary order.
Actually, NetCDF limits data to at most 100 dimensions, but
this has not proven to be a meaningful restriction.
\item MINC is {\it multi-modal}.  MINC has been used to store CT, MRI,
PET, EEG, and other medical imaging data.
\item MINC is {\it extensible}.  MINC file may contain an arbitrary
collection of supporting attributes and data.  If your study requires
that you keep track of a patient's blood pressure or psychiatric
history, this information can be added to the header of your MINC
files without having to concern yourself with
\item MINC is {\it self-describing}.  Most of the attributes and
variables used in MINC have descriptive names and values which can be
easily interpreted by a user.
\item MINC permits {\it scaling} of voxel data on either a per-image or
per-slice basis.
\item MINC defines both a {\it voxel} and a {\it world} coordinate system.  
A MINC file effectively stores a linear transform which defines the 
relationship between the logical layout of the voxels in the file and 
some reference physical coordinate system.  The physical coordinate
system could be the scanner's native coordinates, or it could be a
more universal coordinate space such as the Talairach system.
\end{itemize}

Like many specialized computing terms, the term ``MINC'' has been used in
several different ways over the years.  It may refer to the file
format itself, that is, the definition of the physical and logical
layout of data within a MINC file. It is also applied to the
programming environment which exists to provide access to MINC format
files.  Lastly, the term sometimes refers to the rapidly evolving set
of programs and scripts which analyze, modify, or display MINC files.

The ``core'' MINC system can be considered to include the following:
\begin{itemize}
\item The file format itself
\item The ``libminc'' programming interface, which allows programmers
full access to the format.
\item The ``volume\_io'' programming interface, which provides a 
simplified but restricted programming interface to the MINC format.
\item A set of tools written using these libraries.
\end{itemize}

In addition to the core MINC tools, a large set of additional
application programs exist which perform more sophisticated operations
on MINC files.  These include programs for visualization, image
enhancement or correction, automatic tissue classification, and image
registration.

\subsection{The evolution of MINC}
MINC 2 has been designed to address a few specific problems that
had been identified in MINC 1. 
\begin{itemize}
\item Limited file size.  The NetCDF file format used 32-bit pointers to
address objects within the file.  This effectively restricted files to a 
maximum size of 2 gigabytes.  With the advent of very high resolution
brain atlas data (from macrotome or other sources) and large fMRI datasets,
it became clear that this restriction might become a serious problem.
\item Restricted data types.  The NetCDF format defines a small fixed
set of data types - integers, floating point, and ASCII strings.  
Neither aggregate data (arrays or structures) nor labeled (enumerated) data
are supported as fundamental data types in NetCDF.
\item Limited storage options.  NetCDF files store data in a
continguous array.  This inhibits the addition of either block
addressible data or internal data compression to the NetCDF format.
\end{itemize}
Since most of these problems were inherent in the MINC 1 file format,
it was clear that the design of MINC 2 would require a major revision
of the file format.  The team developing MINC 2 chose to replace
NetCDF with the HDF5 library to form the basis of the MINC 2 format.
HDF5 provides a number of advanced features which are not available in
NetCDF.

For more information on the specifics of the MINC 2 file format, see the
appropriate document (insert reference here).

\subsection{Definitions}
\begin{itemize}
\item volume - A MINC file.  
\end{itemize}
\newpage
\section{Using the core tools}
The core tools included with the basic MINC software distribution
share a number of characteristics.  First, they all function correctly
with any MINC file, regardless of dimensionality or data type.
Second, they are implemented as UNIX ``command line'' tools, with no
graphical user interface.  While this can make the tools somewhat more
difficult to learn, it has the advantage that it is easy to build more
complex applications which invoke the tools from within scripting
languages or batch programming environments.  Lastly, the tools are
designed so that the input files are never modified: The result of any
MINC tool is always one or more new MINC files, if appropriate.

\subsection{Common command-line options}
The MINC tools use a large and sometimes bewildering set of command line
options to control the behavior of the tools.  Rather than treat
each tool separately, this section will attempt to explain some of the 
more common command line options.

Most options are introduced with a single ``dash'' character.

Many of these command line options have default values, so if you 
don't explicitly specify a value, the program will just assume that
you want the ``normal'' behavior.  The default options are indicated
in bold type in each entry below.

\begin{itemize}
\item {\tt -version} - This option will cause the tool to print its
version number and exit.  No other action will be taken.
\item{\tt -help} - This option asks the tool to print a 
summary of all of the options that it recognizes, with brief descriptions
of each option.
\item \texttt{\textbf{-noclobber}} and {\tt -clobber} - These options control
whether or not a MINC tool will overwrite an existing output file.
Normal behavior of all MINC tools is to refuse to replace an existing
file.  For example, if you already have a file named ``test.mnc'' in
the same directory as ``input.mnc'', and you type the following:\\

\mbox{\tt mincreshape -transverse -noclobber input.mnc test.mnc}\\

The program would refuse to run, because doing so would destroy the
existing {\tt test.mnc} file.  If you specify the {\tt -clobber} option, the
program would replace the existing file with the new output file.

\item \texttt{\textbf{-quiet}} and {\tt -verbose} - Specifying
'-verbose' will cause the program to print a more extensive list of
intermediate actions and calculations, which may be useful for
debugging.  Specifying '-quiet' will suppress most of these messages.

\item \texttt{-2} - This option tells the program to produce a MINC 2
format output file.

\end{itemize}

\subsection{Format conversion}
The new tool ``mincconvert'' has been created for conversion between
MINC 1 and MINC 2 format.  To convert a MINC 1 file to a MINC 2 file you
would use mincconvert like this:\\
\mbox{\tt mincconvert -2 input.mnc output.mnc} \\
To convert from a MINC 2 back to MINC 1 format, omit the ``-2'' option: \\
\mbox{\tt mincconvert input.mnc output.mnc}

For conversion to MINC 2 from other formats such as Analyze or DICOM,
it may be necessary to first convert to MINC 1 format, and then use
mincconvert to convert from MINC 1 to MINC 2.  This limitation will be
removed from future versions of these conversion scripts.

\subsection{Working with files in either format}
Each of the core MINC tools now automatically detects MINC 1 or MINC 2
files as input files and will work correctly in either case.

Additionally, each of the core MINC tools which can generate a MINC file
as output now accepts the ``-2'' option
to indicate that output in MINC 2 format is desired.

\subsection{Synopsis of core tools}
\begin{itemize}
\item Image geometry and discretization
\begin{itemize}
\item mincreshape - This tool is used to restructure the data in a MINC
file.  It can be used to extract a subset of data, or to reorder the 
dimensions in the data.  It does not modify the scaling of data, and it
does not change the world coordinate system of the file.  Performs dimension
length changes, however these are accomplished using very simple voxel
averaging or doubling.  For more sophisticated file reshaping, see
{\tt mincresample}.
\item mincresample - This tool is used to apply an arbitray coordinate
transformation on a MINC file.  Can perform tricubic, trilinear, or
nearest-neighbor interpolation.
\item mincconcat - This tool will concatentate any number of MINC
files along a specific axis, producing a single output file.
\end{itemize}
\item Voxelwise calculations
\begin{itemize}
\item mincmath - Performs simple, ``voxelwise'' math operations
on one or more MINC files.  For example, {\tt mincmath} may be used
to add two volumes, to add a constant to a volume, to calculate the
square or inversion of a volume, etc.
\item mincstats - Calculates a number of basic statistical
characteristics of a MINC file.  These include voxel count, volume,
global minimum and maximum, sum, variance, standard deviation, etc.
Will also calculate histograms and histogram-related statistics.
\item minccalc - Similar to {\tt mincmath}, {\tt minccalc} can perform
a wider and more complex range of voxelwise mathematical operations
on a series of MINC files.
\item mincaverage - Calculates an average MINC volume from a number of
input volumes.
\item minclookup - Performs a ``lookup table'' conversion on a MINC file
replacing values with new values according to an arbitrary function.
For example, the file can be converted to a grayscale, spectral RGB values.
\item mincmakescalar - Convert a vector image to a scalar image.
\item mincmakevector - Convert a group of scalar images into a vector image.
\end{itemize}
\item Utility functions
\begin{itemize}
\item mincheader - Print the header of a MINC file.
\item minc\_modify\_header - Change individual attributes in a MINC file. This
is useful for adding or changing supporting information, or for correcting
coordinate information without altering data.
\item mincedit - Edit a MINC file.
\item mincview - View a MINC file.
\item minchistory - Print the ``history'' attribute of a MINC file, which
provides an audit trail of the commands used to generate this file.
\item mincdiff - Display the differences between two MINC files.
\item minccopy - Copy data from one MINC file to another.
\end{itemize}
\item Basic file format conversion
\begin{itemize}
\item mincconvert - Used to convert a MINC 1 file to a MINC 2 file, or
vice versa.
\item minctoraw - Convert a MINC file to a raw data file.  A raw data file
is a file that consists only of voxel values, stored in a well-ordered
arrangement.
\item rawtominc - Convert a raw datafile to a MINC file.
\item mincextract - Extract an arbitrary chunk of MINC file data.
\item mincpik - Generate graphic images from a MINC file.
\end{itemize}
\end{itemize}
\subsection{Where to get more information}
You can get additional information about each of the core MINC tools from
two sources.  Each tool has a UNIX ``man'' page which gives an extensive
explanation of the function of the tool and the options which control its
behavior.

For a more abbreviated list of the tool's options, use the {\tt -help} 
option with the command.

\newpage
\section{Working with the MINC 2 library}
Programmers may use the MINC 2 library to create new programs that manipulate
MINC files.  This library has been designed and written for use with the ``C''
programming language.  It is not necessary to have an understanding of HDF5
or NetCDF in order to use the MINC 2 library.

It is beyond the scope of this document to provide a complete
explanation of all of the library's functionality.  For an exhaustive
reference to the MINC 2 programming interface, please see the document
\emph{MINC 2.0 Programmer's Reference Manual}.

\subsection{Header and library files}
The MINC 2 library can be included in any C or C++ program by including
the file ``minc2.h'' in the source file, and linking against the libminc2.a
library.  This can be accomplished with most UNIX linking tools by adding
the following to your compiler's link options:\\

{\tt -lminc2 -lhdf5 -lnetcdf} \\

It is necessary to explicitly link with both the HDF5 and the NetCDF
libraries to build a MINC 2 program.

\subsection{Function return values}
Most MINC 2 library functions return a signed integer value that
reports either that the operation succeeded ({\tt MI\_NOERROR}) or
failed ({\tt MI\_ERROR}).  More generally, a negative return value
is interpreted as an error result, while a zero or positive value
is interpreted as signifying a successful operation.
\subsection{Opening and closing a volume}
A MINC 2 volume may be opened with the function call: \\

{\tt miopen\_volume(const char *path, int mode, mihandle\_t *hvol)}; \\

The {\tt path} argument gives the absolute or relative pathname of the
file to be accessed.

The {\tt mode} argument controls whether the file is opened for read
access only ({\tt MI2\_OPEN\_RDONLY}), or for both read and write access
({\tt MI2\_OPEN\_RDWR}). 

The file ``handle'', which must be used in subsequent operations on the
file, is returned in the {\tt hvol} argument.

Once your program is finished using the file, it should close the file
and release the file handle by calling:\\

{\tt miclose\_volume(mihandle\_t hvol);} \\

Here is an example showing the use of these two functions:

\begin{verbatim}

#include <minc2.h>

main() {
    int result;
    mihandle_t hvol;

    result = miopen_volume("/home/sparky/test.mnc", 
                            MI_OPEN_RDONLY, &hvol);
    if (result != MI_NOERROR) {
        fprintf(stderr, "Error opening the input file.\n");
    }

    /* Work with the file here. */

    miclose_volume(hvol);
}
\end{verbatim}

\newpage
\section{Data types}
\subsection{Types}
These types refer to the format of data as it is stored in the file.
Additional scaling or other conversions may apply in some instances, 
depending upon the class of the volume.
\begin{enumerate}
\item MI\_TYPE\_BYTE - 8-bit signed integer.
\item MI\_TYPE\_SHORT - 16-bit signed integer.
\item MI\_TYPE\_INT - 32-bit signed integer.
\item MI\_TYPE\_FLOAT - 32-bit floating point.
\item MI\_TYPE\_DOUBLE - 64-bit floating point.
\item MI\_TYPE\_STRING - ASCII string.
\item MI\_TYPE\_UBYTE - 8-bit unsigned integer.
\item MI\_TYPE\_USHORT - 16-bit unsigned integer.
\item MI\_TYPE\_UINT - 32-bit unsigned integer.
\item MI\_TYPE\_SCOMPLEX - 16-bit signed integer complex
\item MI\_TYPE\_ICOMPLEX - 32-bit signed integer complex
\item MI\_TYPE\_FCOMPLEX - 32-bit floating point complex
\item MI\_TYPE\_DCOMPLEX - 64-bit floating point complex
\item MI\_TYPE\_UNKNOWN - Used for record or array types.
\end{enumerate}
\subsection{Classes}
The class of a volume describes how the voxel data should be interpreted.
The most typical case is of a volume which has integer type, but the class
value ``real''.  This corresponds to the typical MINC 1 file, which defined
scaling
\begin{enumerate}
\item MI\_CLASS\_REAL - Floating point.  The data are to be
interpreted as floating point values regardless of the storage type.
May be used with any integer or floating point type.
\item MI\_CLASS\_INT - Integer.  The data are to be interpreted as integer
values.  May be used with any integer type.
\item MI\_CLASS\_LABEL - Labeled or enumerated data.  The values of
the voxels are assumed to be taken from a discrete set.  These values
may be assigned a sybolic name to clarify the interpretation of the
data.  May be used with any integer type.
\item MI\_CLASS\_COMPLEX - Complex data.  May be used with one of the
complex types only.
\item MI\_CLASS\_UNIFORM\_RECORD - A uniform record structure, essentially an
array with named entries.
\item MI\_CLASS\_NON\_UNIFORM\_RECORD - A non uniform record structure.  This
is not yet implemented.
\end{enumerate}
\section{Dimensions and coordinate systems}
A dimension is simply an axis or index along which the data is stored.  
Typical scans might be two dimensional or three dimensional, while functional 
data may be stored in a four dimensional file with three spatial and one temporal 
dimension.

Each dimension of a MINC image volume/file \footnote {volume and file can 
be used interchangeably as it is not common to use volume for more than
3 dimensions and MINC 2 is N dimensional.} is specified by its ``name''. MINC2 defines
standard naming conventions such as ``xspace'', ``yspace'', ``zspace'' (spatial 
dimensions), ``xfrequency'', ``yfrequency'', ``zfrequency'' 
(spatial frequency dimensions), ``tfrequency'' (temporal frequency dimension), ``time'' 
(time dimension) and finally ``vector-dimension'' (vector or complex data dimension).
``class'' which specifies a general type for the dimensions, is defined as spatial, 
temporal, spatial frequency, temporal frequency or user-defined (i.e. arbitrary) dimension.
So the above defintion creates a more flexible naming convention where any name can be defined
as spatial dimension or vice versa. 
Apart from ``name'' and ``class'' which are both defined as dimension properties, 
in MINC 2 each dimension has a number 
of other properties defined that specify how the data should be interpreted.
For example, the convention for spatial dimensions which is positive xspace coordinates 
run from the patient's left side to right side, 
positive yspace coordinates run from patient posterior to anterior 
and positive zspace coordinates run from inferior to superior is defined in 
``description'' property. The description for any other dimension must be set by the user 
or will be set to empty by default. Each dimension can have a ``unit'' 
value assigned to it which is a text string (e.g. ``mm'', ``seconds'') that 
specifies the units along the dimension's axis. The dimension attribute ``attr''
defines whether the dimension is regularly or irregularly sampled. A regularly 
sampled dimension is defined by an origin or ``start'' value and a ``step'' 
value that defines the distance between points on the axis. An irregularly 
sampled dimension must define an array of step values that specify 
the position of each sample along the axis.``length'' defines the size of the dimension
and ``direction\_cosines'' are a set of three vectors associated with spatial dimensions
and define the precise orientation of the axis relative to "true" x, y, or z coordinates.

For each dimension, the world  coordinate  conversion is specified  
by a pair of attributes: step and start. The xspace world coordinate, 
for example is calculated using x = v*step + start, where x is the x world 
coordinate and v is the voxel count (starting at zero). Thus the magnitude 
of the step attribute specifies the distance between voxels and the sign 
of the step attribute specifies the orientation of the axis. MINC files are 
allowed to have non-orthogonal axes with the dimensions not perfectly 
aligned with the named axis. There can be a direction\_cosine attribute that 
gives the true orientation of the axis. For example, normally the xspace 
dimension should line up with the world x axis, ie. direction cosine = (1,0,0)
; however, it is possible to have a direction cosine of (0.9, 0.43589, 0).
For detailed formulas please refer to \emph{The MINC 2.0 File Format by Robert D. 
Vincent}.

MINC 2 files can define any number of dimensions. In MINC 2 library, a dimension 
is represented by an opaque \footnote {An object oriented
concept used for OOP programming in C mainly defined in the internal header files and only 
used and NOT changed by the user} datatype
called a \emph{dimension handle}, which is defined by the ``C'' typedef
{\tt midimhandle\_t}. Through the use of the dimension handle, the user can access and 
modify all dimension properties. 

Only four essential properties must be specified prior to creating a new dimension. 
Other properties will be set to default values which can be 
changed individually after the dimension is created.
\begin{verbatim}
 micreate_dimension(const char *name, midimclass_t class, 
                    midimattr_t attr,unsigned long length, 
                    midimhandle_t *new_dim_ptr);
\end{verbatim}
Accessing or modifying any of the dimension properties is done by using \\
\\
{\tt miget\_dimension\_ property() }\\
{\tt miset\_dimension\_ property() } 
\\
where ``property'' can be replaced with any of the dimension properties. 
For Example,

\begin{verbatim}
miget_dimension_units(midimhandle_t dimension, 
                      char **units_ptr);
miset_dimension_units(midimhandle_t dimension, 
                      const char *units);
\end{verbatim}

you can retrieve the dimensions associated with an open MINC file using
the function:\\

\begin{verbatim}
miget_volume_dimensions(mihandle_t hvol, midimclass_t class,
                        midimattr_t attr, miorder_t order,
                        int max_dims, midimhandle_t dims[]);
\end{verbatim}
   
where miorder\_t is an enumerated type flag which determines whether the
dimension order is determined by the file (i.e., the order the dimensions
were created) or by the apparent order (i.e., any order other than the file
order) which must be set by the user.       
     
Finally, to delete the entire dimension defintion and its corresponding handle
just pass the dimension handle to the following function:\\
\begin{verbatim}
      mifree_dimension_handle(midimhandle_t dim_ptr)
\end{verbatim}

\section{Volume Properties}
\section{Reading and writing data}
\section{Volume creation}
\section{Multiresolution, blocking and compression}
\end{document}

