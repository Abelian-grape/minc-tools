%
% Copyright 2003  Bert Vincent, McConnell Brain Imaging Centre, 
% Montreal Neurological Institute, McGill University.
% John G Sled and Leila Baghdadi Mouse Imaging Centre, Toronto
% Permission to use, copy, modify, and distribute this
% software and its documentation for any purpose and without
% fee is hereby granted, provided that the above copyright
% notice appear in all copies.
%
\documentclass{article}
\title{MINC 2.0 Application Programming Interface (API)}
\author{Bert Vincent and Leila Baghdadi and John G Sled}
\date{FEB 01, 2004}
\textwidth 6.0in
\oddsidemargin 0.125in
\textheight 8.5in
\topmargin -0.75in

\begin{document}

\maketitle

\tableofcontents

\clearpage

\section{Changes and Updates}

First draft of the API definition for libminc\\*

THRS APR 24/2003\\*

THRS MAY 01/2003\\*

THRS MAY 02/2003\\*

WED  MAY 07/2003 Bert\\*

TUS  MAY 20/2003 Leila and John\\*

THU  MAY 29/2003 Bert\\*
* Corrected a bunch of minor syntax errors/issues.\\*
* Added complex type as a peer to integer, real, record, etc.\\*
* Proposed changing the term "midge" to a more standard "class".\\*
* Added functions for getting/setting labels in la belled volumes.\\*

MON  JUN 02/2003 Leila and John\\*
* Added Attribute Functions.\\*
* Sorted methods in Alphabetical order.\\*

WED  JUN 04/2003 Leila and John\\*
* Bert's suggested changes.\\*

THU  JUN 05/2003 Bert\\*
* Added ICV stuff.\\*

MON  JUN 09/2003 Bert\\*
* Elaborated on complex datatypes.\\*
* Added volume property list functions.\\*
* Added miopen\_volume.\\*
* Added "units" functions.\\*

WED  JUN 11/2003 Leila and John\\* 
* Added the group functions to attribute functions.\\*
* Added the name argument to all attribute functions.\\* 
* Created a new flag mivoxel\_order\_t.\\* 
* Cleaned the description of micreate\_volume function.\\*
* Moved miget\_volume\_dimensions from the volume group into the dimension functions group.\\*
* Added a new flag miorder\_t.\\*
* Added hyper-cube functions.\\*

THU  JUN 12/2003 Bert\\*
* Minor edits.\\*
* Added miclose\_volume().\\*

THU  JUN 13/2003 Leila and John\\*
* Merged the double and string type attribute functions.\\*
* Added valid min/max and range functions.\\*
* Added the hyper\_cube with icv functions.\\*

MON  JUL 21/2003 Leila and John and Bert\\*
* Added uniform and non-uniform records.\\*
* Added template volumes.\\*
* Added additional voxel/world conversion functions.\\*

TUE  JUL 22/2003 Bert\\*
* Minor edits.\\*
* Added miget\_data\_type\_size.\\*

THU  JUL 24/2003 Leila and John\\*
* Added Two new record functions.\\*
* Replaced opague record with non-uniform.\\*

FRI  JUL 25/2003 Leila and John and Bert\\*
* Added volume scale functions\\*
* changed miget\_data\_size to miget\_data\_type\_size\\*

FRI  OCT 31/2003 Bert\\*
* Added missing documentation for some functions\\*
* Added mode argument to miopen\_volume\\*
* Changed hyper\_cube to hyperslab for all relevant functions\\*

SUN FEB 01/2004 Leila\\*
* Updated the entire document according to the current cvs

THU JUL 22/2004 Leila\\*
* Updated miget\_slice\_scaling\_flag : missing parameter

THU AUG 12/2004 Leila\\*
* Removed MI\_TYPE\_CHAR 
* Added the attribute listing functions to API

MON AUG 30/2004 Leila\\*
* Added Attribute functions

WED OCT 27/2004 Leila\\*
* Correct the order of parameters in range functions

THU OCT 28/2004 Leila\\*
* Added an extra function to retrive the number of defined labels

MON DEC 13/2004 Leila\\*
* Added some minor changes to the convert functions
* Some are not yet implemented. 

\section{Introduction}



\section{An Introduction to HDF5}

(For a complete description, see the ``HDF5 User's Guide''
at http://hdf.ncsa.uiuc.edu/HDF5/doc/H5.user.html ).

\subsection{The HDF5 file}

To be completed later. 

\section{ATTRIBUTE/GROUP FUNCTIONS (11)}
\subsection{micreate\_group}

\begin{verbatim}
NAME 

micreate_group - create a new group

SYNOPSIS

#include <minc2.h>

int micreate_group ( mihandle_t      volume,
                     const char      *path,
                     const char      *name)
                    
                       
                                
DESCRIPTION

This method creates a new empty group with the specified path and name.

RETURN VALUE

micreate_group returns MI_NOERROR if it successfully creates a group or 
MI_ERROR otherwise

\end{verbatim}

\subsection{midelete\_attr}

\begin{verbatim}
NAME 

midelete_attr - delete the attribute given its name

SYNOPSIS

#include <minc2.h>

int midelete_attr ( mihandle_t          volume,
                    const char          *path,
                    const char          *name)
                       
                                
DESCRIPTION

This methods deletes the attribute with the specified name. It also clears 
memory and releases handles.


RETURN VALUE

midelete_attr returns MI_NOERROR if it successfully deletes the attribute
or MI_ERROR otherwise

\end{verbatim}

\subsection{midelete\_group}

\begin{verbatim}
NAME 

midelete_group - delete an existing group

SYNOPSIS

#include <minc2.h>

int midelete_group ( mihandle_t      volume,
                     const char      *path,
                     const char      *name)    
                       
                                
DESCRIPTION

This method deletes the group with the given group name.

RETURN VALUE

midelete_group returns MI_NOERROR if it successfully deletes a group 
or MI_ERROR otherwise

\end{verbatim}

\subsection{miget\_attr\_length}

\begin{verbatim}

NAME 

miget_attr_length - get the length of an attribute

SYNOPSIS

#include <minc2.h>

int miget_attr_length ( mihandle_t              volume,
                        const char              *path,
                        const char              *name,
                        int                     *length ) 
               
                                
DESCRIPTION

This method gets the length of the given attribute name.

RETURN VALUE
miget_attr_length returns MI_NOERROR if it successfully gets the
the dimension of a given attribute name or MI_ERROR otherwise
\end{verbatim}

\subsection{miget\_attr\_type}

\begin{verbatim}
NAME 

miget_attr_type - get the data type of the attribute

SYNOPSIS

#include <minc2.h>

int miget_attr_type ( mihandle_t                volume,
                      const char                *path,
                      const char                *name,
                      mitype_t                  *attr_data_type) 
               
                                
DESCRIPTION

This method gets the data_type the given attribute name. For the
definition of mitype_t see miget_data_type().


RETURN VALUE
miget_attr_type returns MI_NOERROR if it successfully gets the
data type of a given attribute name or MI_ERROR otherwise

In the initial implementation, attributes are restricted to type of
either MI_TYPE_DOUBLE or MI_TYPE_CHAR.

\end{verbatim}

\subsection{miget\_attr\_values}

\begin{verbatim}
NAME 

miget_attr_values - get the value(s) of an attribute

SYNOPSIS

#include <minc2.h>

int miget_attr_values ( mihandle_t              volume, 
                        mitype_t                attr_data_type,
                        const char              *path,
                        const char              *name,
                        int                     length,
                        void                    *values)

                                                
DESCRIPTION

This method returns the values of a given attribute name 
(of type double or string) in array "values".


RETURN VALUE

miget_attr_values returns MI_NOERROR if it successfully gets the
attribute value(s) or MI_ERROR otherwise
\end{verbatim}

\subsection{miset\_attr\_values}

\begin{verbatim}

NAME 

miset_attr_values - set the value(s) of the attribute

SYNOPSIS

#include <minc2.h>

int miset_attr_values ( mihandle_t              volume,
                        mitype_t                attr_data_type,
                        const char              *path,
                        const char              *name,
                        int                     length,
                        void                    *values)
                       
                                
DESCRIPTION

This method sets the value(s) of type double or string for the 
specified attribute name.  If the attribute does not exist, 
it will be created.

RETURN VALUE

miset_attr_values returns MI_NOERROR if it successfully sets the
attribute value(s) or MI_ERROR otherwise
\end{verbatim}

\subsection{milist\_start}

\begin{verbatim}
NAME 

milist_start - get list of attributes or groups starting at given path

SYNOPSIS

#include <minc2.h>

int milist_start ( mihandle_t              volume, 
                   const char              *path,
                   int                     flags,
                   milisthandle_t          *handle)

                                                
DESCRIPTION

This function begins listing the attributes or groups in the 
given path.

If the flags argument includes MILIST_RECURSE, then all of the 
paths subgroups will be searched and their attributes or groups 
will be listed as well.

The path string should always end with a slash ("/") character.

The returned handle is used to pass to milist_attr_next() for 
attributes or milist_grp_next() for groups and 
milist_finish() to actually perform the listing operation.

RETURN VALUE

milist_start returns MI_NOERROR if it successfully gets the
attribute(s) or group(s) or MI_ERROR otherwise

\end{verbatim}

\subsection{milist\_attr\_next}

\begin{verbatim}
NAME 

milist_attr_next - get the attributes at given path

SYNOPSIS

#include <minc2.h>

int milist_attr_next  ( milisthandle_t          handle,
                        char                    *path,
                        int                     maxpath
                        char                    *name,
                        int                     maxname)

DESCRIPTION

This function returns each of the attributes associated with the
listing initiated by a call to milist_attr_start().  Each successive
call of the milist_attr_next() function returns the next attribute's
path and name in the corresponding string buffers.  

When all possible attributes have been returned, the function returns
an error.

The maxpath and maxname parameters are used to set the maximum length of the
two string buffers.

If the listing operation is not recursive, the path parameter will always
have length zero (the empty string).

RETURN VALUE

milist_attr_next returns MI_NOERROR if it successfully gets the
attribute name or MI_ERROR otherwise

\end{verbatim}

\subsection{milist\_grp\_next}

\begin{verbatim}

NAME 

milist_grp_next - get the group(s) at given path

SYNOPSIS

#include <minc2.h>

int milist_grp_next  ( milisthandle_t          handle, 
                       char                    *path,
                       int                     maxpath)
                        
DESCRIPTION

This function returns each of the groups associated with the listing 
initiated by a call to milist_start(). Each successive call to 
milist_grp_next() returns the next group's path in the corresponding 
string buffer.

When all possible groups have been returned, the function returns an error.

RETURN VALUE

milist_grp_next returns MI_NOERROR if it successfully gets the
group(s) name or MI_ERROR otherwise

\end{verbatim}

\subsection{milist\_finish}

\begin{verbatim}
NAME

milist_finish - finish listing attributes or groups

SYNOPSIS

#include <minc2.h>

int milist_finish (milisthandle_t handle)

DESCRIPTION

Finishes the attribute or groups listing process, and frees any memory
associated with the handle.

RETURN VALUE 

milist_finish() returns MI_NOERROR if it completes successfully,
or MI_ERROR otherwise.

\end{verbatim}

\section{DATA TYPE/SPACE FUNCTIONS (5)}
\subsection{miget\_data\_class}

\begin{verbatim}
NAME

miget_data_class - get the class type of data

SYNOPSIS

#include <minc2.h>

int miget_data_class ( mihandle_t       volume,
                       miclass_t        *volume_class) 

DESCRIPTION

volume_class is the interpretation of the numerical values of the volume
independent of the numerical type used to represent the data.  The
class type is defined as follows.

tyepdef enum {
MI_CLASS_REAL            = 0,
MI_CLASS_INT             = 1,
MI_CLASS_LABEL           = 2,
MI_CLASS_COMPLEX         = 3,
MI_CLASS_UNIFORM_RECORD  = 4,
MI_CLASS_NON_UNIFORM_RECORD   = 5
} miclass_t;

where MI_CLASS_LABEL is used for enumerated data in which a
description is associated with each value and MI_CLASS_RECORD is used
for aggregate datatypes consisting of multiple values.
 
RETURN VALUE

miget_data_class returns the data class of the volume or MI_ERROR if an 
error occurs.
\end{verbatim}

\subsection{miget\_data\_type}
\begin{verbatim}
NAME

miget_data_type - get the volume's data type

SYNOPSIS

#include <minc2.h>


int  miget_data_type( mihandle_t       volume,
                      mitype_t         *volume_data_type)

DESCRIPTION

miget_data_type gets the date-type of the volume, which in this case refers
to the actual format in which the data is stored on disk. Note that volume
of type string is not supported.

The mitype_t type is defined as follows:

typedef enum {
        MI_TYPE_BYTE = 1,       /* 8-bit signed integer */
        MI_TYPE_SHORT = 3,      /* 16-bit signed integer */
        MI_TYPE_INT = 4,        /* 32-bit signed integer */
        MI_TYPE_FLOAT = 5,      /* 32-bit floating point */
        MI_TYPE_DOUBLE = 6,     /* 64-bit floating point */
        MI_TYPE_STRING = 7,     /* ASCII string */
        MI_TYPE_UBYTE = 100,    /* 8-bit unsigned integer */
        MI_TYPE_USHORT = 101,   /* 16-bit unsigned integer */
        MI_TYPE_UINT = 102,     /* 32-bit unsigned integer */
        MI_TYPE_SCOMPLEX = 1000,   /* 16-bit signed integer complex */
        MI_TYPE_ICOMPLEX = 1001,   /* 32-bit signed integer complex */
        MI_TYPE_FCOMPLEX = 1002,   /* 32-bit floating point complex */
        MI_TYPE_DCOMPLEX = 1003,   /* 64-bit floating point complex */
        MI_TYPE_UNKNOWN  = -1     /* when the type is a  non_uniform record */

} mitype_t;

typedef struct {
        short real;
        short imag;
} miscomplex_t;

typedef struct {
        int real;
        int imag;
} miicomplex_t;

typedef struct {
        float real;
        float imag;
} mifcomplex_t;

typedef struct {
        double real;
        double imag;
} midcomplex_t;


RETURN VALUE

miget_data_type returns the data-type of the volume or MI_ERROR if an
error occurs.
\end{verbatim}

\subsection{miget\_data\_type\_size}
\begin{verbatim}
NAME

miget_data_type_size - get the size of an individual voxel in a MINC volume

SYNOPSIS

#include <minc2.h>

int miget_data_type_size( mihandle_t        volume, 
                          misize_t          *voxel_size);

DESCRIPTION

This function retrieves the size, in bytes, of the "native" voxel data
type of the volume.

RETURN VALUE

Returns MI_ERROR on error (invalid volume handle, for example), or
MI_NOERROR on success.
\end{verbatim}

\subsection{miget\_space\_name/miset\_space\_name}
\begin{verbatim}
NAME 

miget_space_name, miset_space_name - get or set the space type name for
a MINC volume

SYNOPSIS

#include <minc2.h>

int miget_space_name( mihandle_t volume, char **name );

int miset_space_name( mihandle_t volume, const char *name );

DESCRIPTION

miget_space_name retrieves the "space" name of the given volume,
returning a pointer to a string.  The memory allocated
by this function should be released with a call to mifree_name().

miset_space_name will set the space name of the volume.  The new name
must be no greater than 128 characters in length, including the
trailing zero byte.

Space names are used to define the coordinate system of the volume.  Three
standard values are defined by MINC:

MI_NATIVE    "native____"
MI_TALAIRACH "talairach_"
MI_CALLOSAL  "callosal__"

"Native" space specifies the coordinate system of a particular
scanner.  Talairach and callosal are standard coordinate systems for
brain images.

If not explicitly set, the space will be type MI_NATIVE by default.

RETURN VALUE

miget_space_name returns the length of the name retrieved, including
the terminating zero byte.  miset_space_name will return MI_NOERROR
on success.  Both functions return MI_ERROR if an error occurs.

SEE ALSO

mifree_name
\end{verbatim}

\section{DIMENSION FUNCTIONS (39)}
\subsection{miget\_volume\_from\_dimension}
\begin{verbatim}
NAME 

miget_volume_from_dimension - to figure out whether a dimensions is
associated with a volume or not

SYNOPSIS

#include <minc2.h>

int miget_volume_from_dimension ( midimhandle_t         dimension,
                                  mihandle_t            *volume)
                                
DESCRIPTION

This method returns the volume handle associated with a given dimension.


RETURN VALUE

miget_volume_from_dimension returns MI_ERROR if the specified handle
is not associated with the volume and MI_NOERROR otherwise.
\end{verbatim}

\subsection{micopy\_dimension}
\begin{verbatim}
NAME

micopy_dimension - create copy of the given dimension

SYNOPSIS

#include <minc2.h>

int micopy_dimension ( midimhandle_t       dim_ptr,
                       midimhandle_t       *new_dim_ptr)

DESCRIPTION

Creates a copy of the specified dimension and returns the handle to the copy

RETURN VALUE

micopy_dimension returns MI_NOERROR if it successfully copies all the 
attributes of the provided dimension and MI_ERROR otherwise
\end{verbatim}

\subsection{micreate\_dimension}
\begin{verbatim}
NAME

micreate_dimension - define a new dimension in a MINC volume

SYNOPSIS

#include <minc2.h>


int micreate_dimension ( const char     *name,
                         midimclass_t   class, midimattr_t      attr,
                         unsigned long  size, midimhandle_t     *new_dim_ptr);

DESCRIPTION

This function defines a dimension that can be used in the definition
of a new MINC volume (see the create_volume function).  The name may
be an arbitrary string of up to 128 alphanumeric characters. Any of
the "standard" names retained from MINC 1.0 retain their default
behaviors: MIxspace, MIyspace, and MIzspace default to spatial
dimensions, and MItime default to be a time dimension.  MItfrequency
is a temporal frequency axis, and MIxfrequency, MIyfrequency, and
MIzfrequency are spatial frequency axes.  Any other name may be used.

When initially defined, a regularly-sampled dimension will have a
"start" value of zero, and a "separation" or "step" value of 1.0.  An
irregular dimension will be initialized with all offsets equal to
zero.

The type midimclass_t is defined as follows:

typedef enum { 
MI_DIMCLASS_ANY         = 0,            /* Don't care (or unknown) */
MI_DIMCLASS_SPATIAL     = 1,            /* Space */
MI_DIMCLASS_TIME        = 2,            /* Time */
MI_DIMCLASS_SFREQUENCY  = 3,            /* Spatial frequency */
MI_DIMCLASS_TFREQUENCY  = 4,            /* Temporal frequency */
MI_DIMCLASS_USER        = 5,            /* Arbitrary user-defined axis */
MI_DIMCLASS_FLAT_RECORD = 6             /* Record as dimension */

} midimclass_t;

The type midimattr_t is a bit field of dimension attributes, defined as
follows:

typedef unsigned int midimattr_t; 
#define MI_DIMATTR_ALL 0
#define MI_DIMATTR_REGULARLY_SAMPLED 0x1
#define MI_DIMATTR_NOT_REGULARLY_SAMPLED 0x2

The "size" argument may range from 0 to 2^32, which should provide
enough range to represent detail on the order of 10 Angstroms in
typical medical imaging applications.

If successful, the function will return a handle to the newly-defined
dimension in the location specified by "new_dim_ptr".

RETURN VALUE

micreate_dimension returns MI_NOERROR on success, MI_ERROR on failure.
\end{verbatim}

\subsection{mifree\_dimension\_handle}
\begin{verbatim}
NAME

mifree_dimension_handle - delete the dimension definition associated 
with the given handle

SYNOPSIS

#include <minc2.h>

int mifree_dimension_handle ( midimhandle_t       dim_ptr)

DESCRIPTION

Deletes the dimension definition (i.e., dimension handle and dimension itself)
only if the dimension is NOT associated with a volume.

RETURN VALUE

mifree_dimension_handle returns MI_NOERROR if it successfully deletes 
a dimension or MI_ERROR otherwise.
\end{verbatim}

\subsection{miget\_volume\_dimensions}
\begin{verbatim}
NAME  

miget_volume_dimensions - retrieve the list of dimensions defined in a MINC 
volume, according to their class and attribute.

SYNOPSIS

#include <minc2.h>

int miget_volume_dimensions ( mihandle_t        volume, 
                              midimclass_t      class, 
                              midimattr_t       attr, 
                              miorder_t         order,
                              int               array_length,
                              midimhandle_t     dimensions[]);
                              

DESCRIPTION

This function is used to retrieve an array of dimension handles for a
MINC volume.  It will place the handles of the first "array_length"
dimensions into the "dimensions[]" array, returning only those dimension
whose characteristics match the "class" and "attr" parameters. 
The miorder_t is an enumerated type flag which determines whether the
dimension order is determined by the file or by the apparent order set by 
the user. This function will FAIL if the user has not set the apparent 
dimension order by calling either of 
miset_apparent_dimension_order() or miset_apparent_dimension_order_by_name()
before calling this function with MI_DIMORDER_APPARENT flag


The following example will return up to 3 spatial dimensions:

    int result;
    midimhandle_t dimensions[3];
    mihandle_t volume;

    result = miget_vol_dimensions (volume, MI_DIMCLASS_SPATIAL, 
                                   MI_DIMATTR_ALL, MI_DIMORDER_FILE, 
                                   3, dimensions);

RETURN VALUE

miget_volume_dimensions returns the number of dimension handles copied 
to the array, or MI_ERROR on failure.

SEE ALSO

miset_apparent_dimension_order, miset_apparent_dimension_order_by_name
\end{verbatim}

\subsection{miset\_apparent\_dimension\_order}
\begin{verbatim}
NAME 

miset_apparent_dimension_order - set apparent dimension order


SYNOPSIS

#include <minc2.h>

int miset_apparent_dimension_order ( mihandle_t         volume,
                                     int                array_length,
                                     midimhandle_t      dimensions[])
                                     


DESCRIPTION

This method sets an apparent dimension order. The user can sort the
dimensions in any desired order. If the user specifies fewer dimensions
than the existing ones, then they are assumed to be added to the last.
For example, given (z,y,x) for the file dimension order of (x,y,z,t) 
will result in (t,z,y,x) and so on. 

RETURN VALUE

miset_apparent_dimension_order returns MI_NOERROR if it successfully
sets the apparent dimension order or MI_ERROR otherwise

SEE ALSO

miset_apparent_dimension_order_by_name
\end{verbatim}

\subsection{miset\_apparent\_dimension\_order\_by\_name}
\begin{verbatim}
NAME 

miset_apparent_dimension_order_by_name - set apparent dimension order
by name


SYNOPSIS

#include <minc2.h>

int miset_apparent_dimension_order_by_name ( mihandle_t         volume,
                                             int                array_length,
                                             char               **names)
                                             


DESCRIPTION

This method sets an apparent dimension order by dimension name. Note that
all dimension names must be different or an error occurs.


RETURN VALUE

miset_apparent_dimension_order_by_name returns MI_NOERROR if it successfully
sets the apparent dimension order by name or MI_ERROR otherwise
\end{verbatim}

\subsection{miset\_apparent\_record\_dimension\_flag}
\begin{verbatim}
NAME 

miset_apparent_record_dimension_flag - set the record flag

SYNOPSIS

#include <minc2.h>

int miset_apparent_record_dimension_flag ( mihandle_t           volume,
                                           int                  flatten_flag)
                                             


DESCRIPTION

This method causes a volume to appear to have a record dimension. The record
dimension will be set to uniform and flat (i.e., the volume will have one more
dimension (n+1)).


RETURN VALUE

miset_apparent_record_dimension_flag sets a flat uniform_record dimension to
a volume and returns MI_NOERROR or MI_ERROR otherwise. 
\end{verbatim}

\subsection{miget\_dimension\_apparent\_voxel\_order}
\begin{verbatim}
NAME 

miget_dimension_apparent_voxel_order - gets the apparent order of voxels


SYNOPSIS

#include <minc2.h>

int miget_dimension_apparent_voxel_order ( midimhandle_t        dimension,
                                           miflipping_t         *file_order,
                                           miflipping_t         *sign)
                                     



DESCRIPTION

This method gets the apparent order of voxels for the specified dimension
and the sign of the step values.

RETURN VALUE

miget_dimension_apparent_voxel_order returns MI_NOERROR if it successfully
gets the apparent voxel order of the specified dimension or MI_ERROR otherwise
\end{verbatim}

\subsection{miset\_dimension\_apparent\_voxel\_order}
\begin{verbatim}
NAME 

miset_dimension_apparent_voxel_order - sets the apparent order of voxels


SYNOPSIS

#include <minc2.h>

int miset_dimension_apparent_voxel_order ( midimhandle_t        dimension,
                                           miflipping_t         flipping_order)




DESCRIPTION

This method sets the apparent order of voxels for the specified dimension.
The miflipping_t is an enumerated type as follows:

tyepdef enum {
        MI_FILE_ORDER           = 0,   /* no flip */
        MI_COUNTER_FILE_ORDER   = 1,   /* flip    */
        MI_POSITIVE             = 2,   /* check step if positive -> no flip
                                                        negative -> flip    */
        MI_NEGATIVE             = 3    /* check step if positive -> flip
                                                        negative -> no flip */
} miflipping_t;




RETURN VALUE

miset_dimension_apparent_voxel_order returns MI_NOERROR if it successfully
sets the apparent voxel order of the specified dimension or MI_ERROR otherwise
\end{verbatim}

\subsection{miget\_dimension\_class/miset\_dimension\_class}
\begin{verbatim}
NAME

miget_dimension_class, miset_dimension_class - get or set the class of
a MINC dimension.

SYNOPSIS

#include <minc2.h>

int miget_dimension_class ( midimhandle_t       dimension,
                            midimclass_t        *class );

int miset_dimension_class ( midimhandle_t       dimension,
                            midimclass_t        class);

DESCRIPTION

The "class" of a MINC dimension defines the general type of a dimension, 
whether it is a spatial dimension, a time dimension, or a frequency dimension
as transformed from either space or time.  User-defined dimension are also
permitted, with no default handling assumed. Finally, a record can be specified
as a dimension.

The definition of midimclass_t is as follows:

typedef enum { 
MI_DIMCLASS_ANY         = 0,            /* Don't care (or unknown) */
MI_DIMCLASS_SPATIAL     = 1,            /* Space */
MI_DIMCLASS_TIME        = 2,            /* Time */
MI_DIMCLASS_SFREQUENCY  = 3,            /* Spatial frequency */
MI_DIMCLASS_TFREQUENCY  = 4,            /* Temporal frequency */
MI_DIMCLASS_USER        = 5,            /* Arbitrary user-defined axis */
MI_DIMCLASS_FLAT_RECORD = 6             /* Record as dimension */

} midimclass_t;


While the MINC library does not enforce any particular semantics based
upon dimension class, individual MINC programs may define default
behaviors for certain classes of dimensions.

RETURN VALUE

Returns MI_NOERROR on success, or MI_ERROR on failure.
\end{verbatim}

\subsection{miset\_dimension\_cosines/miset\_dimension\_cosines}
\begin{verbatim}
NAME

miget_dimension_cosines, miset_dimension_cosines - Get or set the dimension's
cosine vector.

SYNOPSIS

#include <minc2.h>

int miget_dimension_cosines ( midimhandle_t dimension,
                              double        cosines[3]);

int  miset_dimension_cosines ( midimhandle_t dimension,
                               const double  cosines[3]);

DESCRIPTION

Spatial dimension in MINC volumes may be associated with a vector of direction
cosines which define the precise orientation of the axis relative to "true"
x, y, or z coordinates.

The direction cosine vector always consists of exactly three values
which correspond to the x, y, and z directions, respectively.  This is
true regardless of the ordering of dimensions in a specific volume or
data object.

Because of the direction cosines, it is possible for MINC volumes to
define non-orthogonal dimensions.

These functions fail if the dimension is not of the spatial class.

RETURN VALUE

Returns MI_NOERROR on success, or MI_ERROR on failure.
\end{verbatim}

\subsection{miget\_dimension\_description/miset\_dimension\_description}
\begin{verbatim}
NAME

miget_dimension_description, miset_dimension_description - Get or set the dimension's description.

SYNOPSIS

#include <minc2.h>

int miget_dimension_description ( midimhandle_t dimension,
                                  char **comments_ptr);

int miset_dimension_description ( midimhandle_t dimension,
                                  const char *comments);

DESCRIPTION

Get and Set the dimension description. Note that the spatial dimensions
(xspace, yspace, zspace) are initialized according to minc1 description.
All other dimensions will have an empty description unless set by the user.

RETURN VALUE

Returns MI_NOERROR on success, or MI_ERROR on failure.
\end{verbatim}

\subsection{miget\_dimension\_name/miset\_dimension\_name}
\begin{verbatim}
NAME

miget_dimension_name, miset_dimension_name - get or set the identifier
of a MINC dimension

SYNOPSIS

#include <minc2.h>

int miget_dimension_name ( midimhandle_t dimension, char **name_ptr );
int miset_dimension_name ( midimhandle_t dimension, const char *name );

DESCRIPTION

miget_dimension_name retrieves the name of the given dimension, allocating
the space needed.  The memory allocated by this function should be released
with a call to mifree_name().

miset_dimension_name will rename an existing dimension.  The new name
must be no greater than 128 characters in length, including the
trailing zero byte.

RETURN VALUE

miget_dimension_name returns the length of the name retrieved, including
the terminating zero byte.  miset_dimension_name will return MI_NOERROR
on success.  Both functions return MI_ERROR if an error occurs.

SEE ALSO

mifree_name
\end{verbatim}

\subsection{miget\_dimension\_offsets/miset\_dimension\_offsets}
\begin{verbatim}
NAME

miget_dimension_offsets, miset_dimension_offsets - get or set the absolute
world coordinates of points along a MINC dimension

SYNOPSIS

#include <minc2.h>

int miget_dimension_offsets ( midimhandle_t     dimension, double       offsets[],
                              unsigned long     array_length, 
                              unsigned long     start_position);

int miset_dimension_offsets ( midimhandle_t     dimension, const double offsets[],
                              unsigned long     array_length, 
                              unsigned long     start_position);

DESCRIPTION

These functions get or set the dimension offsets, that is, the
absolute world coordinates of each sampled point along the dimension.

The caller may retrieve up to "array_length" values, starting at the
integer index "start_position".  Thus an arbitrary contiguous subset
of the dimension's offsets may be retrieved or stored.  An error is
returned if the "start_position" exceeds the total size of the
dimension.  If the value of "start_position" is legal, but the sum of
"start_position" and "array_length" exceeds the size of the dimension,
the function will get or set offsets up to the size of the dimension.
Any extra positions in the offsets[] array will be ignored.

It is explicitly legal to call this function for a regularly sampled
dimension.  The result will be a list of values calculated from the
"start" and "separation" values of the dimension.  However, it is 
not possible to set offsets on a regularly sampled dimension.

RETURN VALUE

Returns the number of offset values read or written , or MI_ERROR if
an error is detected.
\end{verbatim}

\subsection{miget\_dimension\_sampling\_flag/miset\_dimension\_sampling\_flag}
\begin{verbatim}
NAME

miget_dimension_sampling_flag, miset_dimension_sampling_flag - get or set
the sampling flag for a MINC dimension

SYNOPSIS

#include <minc2.h>

int miget_dimension_sampling_flag ( midimhandle_t       dimension,
                                    BOOLEAN             *sampling_flag);

int miset_dimension_sampling_flag ( midimhandle_t       dimension,
                                    BOOLEAN             sampling_flag);

DESCRIPTION

The miget_dimension_sampling flag function retrieves the value
of the "sampling" flag for a given MINC dimension.  This flag
is true (non-zero) if the dimension is sampled at regular
intervals, and false if the dimension is sampled irregularly.

If a dimension has regular sampling, the miget_dimension_step
function may be used to retrieve the sampling interval, and the
miget_dimension_start function may be used to retrieve the origin
value along the axis.

If a dimension has irregular sampling, the miget_dimension_offsets
function may be used to retrieve the positions of each sample along
that axis.

RETURN VALUE

These functions returns MI_NOERROR on success, or MI_ERROR if an 
error is detected (for example, if a parameter is invalid).
\end{verbatim}

\subsection{miget\_dimension\_separation/miset\_dimension\_separation}
\begin{verbatim}
NAME

miget_dimension_separation, miset_dimension_separation - set/get the 
sampling interval for a single dimension

SYNOPSIS

#include <minc2.h>

int miget_dimension_separation ( midimhandle_t          dimension,
                                 mivoxel_order_t        voxel_order,
                                 double                 *separation_ptr );

int miset_dimension_separation ( midimhandle_t          dimension,
                                 mivoxel_order_t        voxel_order,
                                 double                 separation );

DESCRIPTION

Gets or sets the constant sampling interval defined on a regularly-sampled
dimension.  While it is legal to call these functions for an irregularly-
sampled dimension, the values will be ignored. The mivoxel_order_t is an 
enumerated type which is defined as follows

typedef enum {
        MI_FILE_ORDER   = 0,
        MI_APPARENT_ORDER  = 1
} mivoxel_order_t;

This flag specifies whether the voxel order is original from file or
is an apparent one which can be the default (i.e., the same as the 
original file) or the order that is specified by the user.

If not explicitly set, the separation will have a default value of one.

RETURN VALUE

Returns MI_NOERROR on success, or MI_ERROR on failure.

SEE ALSO

miget_dimension_separations, miset_dimension_separations
\end{verbatim}

\subsection{miget\_dimension\_separations/miset\_dimension\_separations}
\begin{verbatim}
NAME

miget_dimension_separations, miset_dimension_separations - get/set the
sampling intervals for a list of dimensions.

SYNOPSIS

#include <minc2.h>

int miget_dimension_separations ( const midimhandle_t   dimensions[],
                                  mivoxel_order_t       voxel_order,
                                  int                   array_length,
                                  double                separations[] );

int miset_dimension_separations ( const midimhandle_t   dimensions[],
                                  mivoxel_order_t       voxel_order,
                                  int                   array_length,
                                  const double          separations[] );

DESCRIPTION

These functions get or set the scalar separation (sampling interval)
associated with each of the dimensions in the input "dimensions[]"
array.  The "array_length" parameter specifies the size of both the
input and output arrays. While it is legal to call these functions for
an irregularly- sampled dimension, the values will be ignored.

RETURN VALUE

Returns the number of separations copied to (or from) the array, or
MI_ERROR if an error occurs.

SEE ALSO

miget_dimension_separation, miset_dimension_separation
\end{verbatim}

\subsection{miget\_dimension\_size/miset\_dimension\_size}
\begin{verbatim}
NAME

miget_dimension_size, miset_dimension_size - get or set the length of a
MINC dimension

SYNOPSIS

#include <minc2.h>

int miget_dimension_size ( midimhandle_t        dimension, 
                           unsigned long        *size_ptr );

int miset_dimension_size ( midimhandle_t        dimension, 
                           unsigned long        size );

DESCRIPTION

These functions get or set the size (or length) of a MINC 2 dimension
object used in creating a new volume.  The size of a dimension
associated with an existing volume cannot be changed.  One can, however,
make a copy of an existing dimension and set the size of the copy. 

RETURN VALUE

Returns MI_NOERROR on success, MI_ERROR on failure.
\end{verbatim}

\subsection{miget\_dimension\_sizes}
\begin{verbatim}
NAME

miget_dimension_sizes - retrieve the sizes of an array of dimension handles

SYNOPSIS

#include <minc2.h>

int miget_dimension_sizes ( const midimhandle_t         dimensions[],
                            int                         array_length,
                            unsigned long               sizes[] );

DESCRIPTION

This function will copy the lengths of each of the dimensions listed in the
"dimensions[]" array into the "sizes[]" array.  The parameter "array_length"
specifies the length of both of the arrays.

RETURN VALUE

Returns MI_NOERROR on success, or MI_ERROR on failure.

SEE ALSO

miget_dimension_size, miset_dimension_size
\end{verbatim}

\subsection{miget\_dimension\_start/miset\_dimension\_start}
\begin{verbatim}
NAME

miget_dimension_start, miset_dimension_start - get or set the origin
of a MINC dimension

SYNOPSIS

#include <minc2.h>

int miget_dimension_start ( midimhandle_t       dimension, 
                            mivoxel_order_t     voxel_order,
                            double              *start_ptr);

int miset_dimension_start ( midimhandle_t       dimension,
                            mivoxel_order_t     voxel_order,
                            double              start);

DESCRIPTION

These functions get or set the origin of the dimension in world
coordinates. While a "start" value may be legally associated with any
dimension, it is considered meaningless when associated with an
irregularly sampled dimension.   

RETURN VALUE

Returns MI_NOERROR on success, or MI_ERROR on failure.
\end{verbatim}

\subsection{miget\_dimension\_starts/miset\_dimension\_starts}
\begin{verbatim}
NAME

miget_dimension_starts, miset_dimension_starts - get or set the start
values

SYNOPSIS

#include <minc2.h>

int miget_dimension_starts ( const midimhandle_t        dimensions[],
                             mivoxel_order_t            voxel_order,
                             int                        array_length,
                             double                     starts[]);
      
int miset_dimension_starts ( const midimhandle_t        dimensions[],
                             mivoxel_order_t            voxel_order,
                             int                        array_length,
                             const                      double starts[]);

DESCRIPTION

These functions get or set the start value for an array of
regularly-sampled dimensions.  The start value defines the origin of
that dimension.  While it is legal to call these functions for an
irregularly-sampled dimension, the values will be ignored.

If not explicitly set, the start value defaults to zero.

RETURN VALUE

These functions return the number of start values copied to or from
the starts[] array, or MI_ERROR on failure.
\end{verbatim}

\subsection{miget\_dimension\_units/miset\_dimension\_units}
\begin{verbatim}
NAME

miget_dimension_units, miset_dimension_units - get or set the unit string
for a MINC dimension

SYNOPSIS

#include <minc2.h>

int miget_dimension_units ( midimhandle_t dimension, char **units_ptr );
int miset_dimension_units ( midimhandle_t dimension, const char *units );

DESCRIPTION

miget_dimension_units retrieves the units of the given dimension,
allocating the space needed for the string.  The memory allocated by
this function should be released with a call to mifree_name().

miset_dimension_name will set the units for an existing dimension.
The new string must be no greater than 128 characters in length,
including the trailing zero byte.

Typical values for units include "mm" or "cm" for spatial dimensions
and "seconds" or "msec" for time dimensions.

RETURN VALUE

miget_dimension_units returns the length of the string retrieved,
including the terminating zero byte.  miset_dimension_units will
return MI_NOERROR on success.  Both functions return MI_ERROR if an
error occurs.

SEE ALSO

mifree_name
\end{verbatim}

\subsection{miget\_dimension\_width/miset\_dimension\_width}
\begin{verbatim}
NAME

miget_dimension_width, miset_dimension_width - get or set the full-width
half-maximum value for points along a MINC dimension

SYNOPSIS

#include <minc2.h>

int miget_dimension_width ( midimhandle_t       dimension, 
                            mivoxel_order_t     voxel_order,
                            double              *width_ptr );

int miset_dimension_width ( midimhandle_t       dimension, 
                            mivoxel_order_t     voxel_order,
                            double              width );

DESCRIPTION

These functions get or set the dimension width, that is, the
full-width half-maximum values of each sampled point along the
dimension.  

These functions are used to set a constant width for regularly-sampled
dimensions.

If not explicitly set, the width will be assumed to be equal to the
dimension's step size.

RETURN VALUE

Returns MI_NOERROR on success, or MI_ERROR on failure.  Will fail if
called for an irregularly-sampled dimension.

SEE ALSO

miget_dimension_widths, miset_dimension_widths
\end{verbatim}

\subsection{miget\_dimension\_widths/miset\_dimension\_widths}
\begin{verbatim}
NAME

miget_dimension_widths, miset_dimension_widths - get or set the full-width
half-maximum values for points along a MINC dimension

SYNOPSIS

#include <minc2.h>

int miget_dimension_widths ( midimhandle_t      dimension, 
                             mivoxel_order_t    voxel_order,
                             unsigned long      array_length, 
                             unsigned long      start_position,
                             double             widths[]);

int miset_dimension_widths ( midimhandle_t      dimension, 
                             mivoxel_order_t    voxel_order,
                             unsigned long      array_length, 
                             unsigned long      start_position,
                             const double       widths[]);

DESCRIPTION

These functions get or set the dimension widths, that is, the
full-width half-maximum values of each sampled point along the
dimension.

The caller may retrieve up to "array_length" values, starting at the
integer index "start_position".  Thus an arbitrary contiguous subset
of the dimension's widths may be retrieved or stored.  An error is
returned if the "start_position" exceeds the total size of the
dimension.  If the value of "start_position" is legal, but the sum of
"start_position" and "array_length" exceeds the size of the dimension,
the function will get or set widths up to the size of the dimension.
Any extra positions in the widths[] array will be ignored.

It is explicitly legal to call this function for a regularly sampled
dimension.  The result will be a list of constant width values.
However, it is not possible to set widths on a regularly sampled
dimension.

RETURN VALUE

Returns the number of offset values read or written , or MI_ERROR if
an error is detected.

SEE ALSO

miget_dimension_width, miset_dimension_width
\end{verbatim}

\section{FREE FUNCTIONS (2)}
\subsection{mifree\_name/mifree\_names}
\begin{verbatim}
NAME

mifree_name, mifree_names - free the storage allocated for strings by 
MINC functions

SYNOPSIS

#include <minc2.h>

int mifree_name ( char *name );
int mifree_names ( char **names );

DESCRIPTION

Frees the space allocated for string storage by MINC function such as
miget_dimension_name and miget_space_name.

RETURN VALUE

Returns MI_NOERROR on success, or MI_ERROR on failure.

SEE ALSO

miget_dimension_name, miget_space_name
\end{verbatim}

\section{HYPERSLAB FUNCTIONS (8)}
\subsection{miget\_hyperslab\_size}
\begin{verbatim}
NAME

miget_hyperslab_size - calculate the size of the hyperslab

SYNOPSIS

#include <minc2.h>

int miget_hyperslab_size ( mitype_t            volume_data_type,
                           int                 number_of_dimensions,
                           int                 sizes_of_dimensions[],
                           misize_t            *size)

DESCRIPTION

calculate the size of the hyperslab. i.e., the amount of memory in BYTES
which is needed to store the hyperslab

RETURN VALUE

miget_hyperslab_size returns MI_NOERROR if it successfully calculates the
size of the hyperslab with specified dimensions and MI_ERROR otherwise
\end{verbatim}

\subsection{miget\_hyperslab\_normalized}
\begin{verbatim}
NAME

miget_hyperslab_normalized - get a normalized hyperslab

SYNOPSIS

#include <minc2.h>

int miget_hyperslab_normalized ( mihandle_t    volume,
                                 mitype_t      buffer_data_type,
                                 long          voxel_offsets[],
                                 long          sizes[],
                                 double        min,
                                 double        max,
                                 void          *buffer)
                                  
                                
DESCRIPTION

The real values in the volume from the interval min through max
is mapped to the maximum representable range for the requested
data type. Float type is NOT an allowed data type.

RETURN VALUE

miget_hyperslab_normalized returns MI_NOERROR if it successfully returns
a normalized hyperslab with specified size and type and MI_ERROR otherwise
\end{verbatim}


\subsection{miget\_hyperslab\_with\_icv}
\begin{verbatim}
NAME

miget_hyperslab_with_icv - get hyperslab with the specified icv

SYNOPSIS

#include <minc2.h>

int miget_hyperslab_with_icv ( int             icv,
                               mitype_t        buffer_data_type,
                               long            voxel_offsets[],
                               long            sizes[],
                               void            *buffer)                
                                        
DESCRIPTION

This method gets the hyperslab with the specified icv.

RETURN VALUE

miget_hyperslab_with_icv returns MI_NOERROR if it successfully gets
the hyperslab with specified icv and MI_ERROR otherwise
\end{verbatim}

\subsection{miset\_hyperslab\_with\_icv}
\begin{verbatim}
NAME

miset_hyperslab_with_icv - set hyperslab with the specified icv

SYNOPSIS

#include <minc2.h>

int miset_hyperslab_with_icv ( int             icv,
                               mitype_t        buffer_data_type,
                               long            voxel_offsets[],
                               long            sizes[],
                               void            *buffer)                
                                        
DESCRIPTION

This method sets the hyperslab with the specified icv.

RETURN VALUE

miset_hyperslab_with_icv returns MI_NOERROR if it successfully sets
the hyperslab with specified icv and MI_ERROR otherwise
\end{verbatim}

\subsection{miget\_real\_value\_hyperslab}
\begin{verbatim}
NAME

miget_real_value_hyperslab - get a real value hyperslab

SYNOPSIS

#include <minc2.h>

int miget_real_value_hyperslab( mihandle_t     volume,
                                mitype_t       buffer_data_type,
                                long           voxel_offsets[],
                                long           sizes[],
                                void           *buffer)

DESCRIPTION

This method converts the data from the type that is in the file in a value
preserving way.If real value exceeds range of the requested data type, the
result is UNDEFINED. The void pointer is pointing to memory which has to be 
allocated by the programmer in advance.


RETURN VALUE

miget_real_value_hyperslab returns MI_NOERROR if it successfully returns
the real value hyperslab and MI_ERROR otherwise
\end{verbatim}

\subsection{miset\_real\_value\_hyperslab}
\begin{verbatim}
NAME

miset_real_value_hyperslab - set a real value hyperslab

SYNOPSIS

#include <minc2.h>

int miset_real_value_hyperslab( mihandle_t     volume,
                                mitype_t       buffer_data_type,
                                long           voxel_offsets[],
                                long           sizes[],
                                void           *buffer)

DESCRIPTION

This method sets a real value hyperslab with the specified type and size. The 
void pointer would get casted to the appropriate type once it is used. The data_type
argument will be used to ensure type compatibility with the hyperslab data type. 

RETURN VALUE

miset_real_value_hyperslab returns MI_NOERROR if it successfully sets
the hyperslab with specified size and type and MI_ERROR otherwise
\end{verbatim}

\subsection{miget\_voxel\_value\_hyperslab}
\begin{verbatim}
NAME

miget_voxel_value_hyperslab - get a voxel value hyperslab

SYNOPSIS

#include <minc2.h>

int miget_voxel_value_hyperslab( mihandle_t    volume,
                                 mitype_t      buffer_data_type,
                                 long          voxel_offsets[],
                                 long          sizes[],
                                 void          *buffer)

DESCRIPTION

This method returns a voxel value hyperslab.


RETURN VALUE

miget_voxel_value_hyperslab returns MI_NOERROR if it successfully gets 
the hyperslab with specified size and type and MI_ERROR otherwise
\end{verbatim}

\subsection{miset\_voxel\_value\_hyperslab}
\begin{verbatim}
NAME

miset_voxel_value_hyperslab - set a voxel value hyperslab

SYNOPSIS

#include <minc2.h>

int miset_voxel_value_hyperslab( mihandle_t    volume,
                                 mitype_t      buffer_data_type,
                                 long          voxel_offsets[],
                                 long          sizes[],
                                 void          *buffer)

DESCRIPTION

This method sets a voxel value hyperslab with the specified type and size. If
the type does not match a simple C cast will be applied.

RETURN VALUE

miset_voxel_value_hyperslab returns MI_NOERROR if it successfully sets the
voxel value hyperslab with specified size and type and MI_ERROR otherwise
\end{verbatim}

%\section{IMAGE CONVERSION VARIABLE FUNCTIONS (4)}
%\subsection{miicv\_volume\_attach}
%\begin{verbatim}
%NAME

%miicv_volume_attach - attach a MINC ICV object to a MINC 2.0 volume.

%SYNOPSIS

%#include <minc2.h>

%int miicv_volume_attach ( int           icv, 
                         % mihandle_t    volume);

%DESCRIPTION

%This function attaches a MINC image conversion variable (ICV) object
%to a MINC 2.0 volume.

%Given the flexibility of the MINC format, there are many different
%possible choices available for details such as dimension order, data
%type, and data range. Accounting for all of the possible combinations
%of these items could make MINC programming too complex or unwieldy in
%many situations.

%MINC ICV objects are a solution to this problem.  They are essentially
%a specification of the properties that the programmer would like the
%data to have. The ICV is responsible for making any necessary
%conversions, hiding the details of the actual data format from the
%programmer.

%A program may allocate any number of ICV objects and may configure and
%attach them independently, allowing the program to have several
%different views of the volume's data at the same time.

%Note that ICV properties cannot be modified while a variable is
%attached to the ICV. If a file and variable are already attached to
%the ICV, they will be automatically detached before the new variable
%is attached.

%NOTE

%This interface is being extended to allow use with the new MINC 2.0
%interface, however, the existing ICV interface will be retained as
%well.

%RETURN VALUE

%MI_NOERROR on success, or MI_ERROR on failure.
%\end{verbatim}

%\subsection{miicv\_volume\_detach}
%\begin{verbatim}
%NAME

%miicv_volume_detach - detach an image conversion variable from a volume.

%SYNOPSIS

%#include <minc2.h>

%int miicv_volume_detach ( int   icv);

%DESCRIPTION

%Deletes the association between an image conversion variable (ICV) 
%and a MINC2.0 volume object.

%RETURN VALUE

%MI_NOERROR on success, or MI_ERROR on failure.

%SEE ALSO 

%miicv_volume_attach
%\end{verbatim}

%\subsection{miicv\_volume\_get/miicv\_volume\_put}
%\begin{verbatim}
%NAME

%miicv_volume_get, miicv_volume_put
                
%SYNOPSIS

%#include <minc2.h>

%int miicv_volume_get(int                 icv, 
                 %    mihandle_t          volume, 
                  %   const unsigned long start[], 
                 %    const unsigned long count[], 
                %     void                *value_ptr);

%int miicv_volume_put(int                 icv, 
%                     mihandle_t          volume, 
%                     const unsigned long start[], 
 %                    const unsigned long count[], 
 %                    void                *value_ptr);

%DESCRIPTION

%These functions actually read or write data through the ICV to the attached 
%variable and volume.

%All value/range conversions and dimension conversions are applied before the
%data is read or written.

%RETURN VALUES

%MI_NOERROR on success, MI_ERROR on failure.
%\end{verbatim}

\section{LABEL FUNCTIONS (4)}
\subsection{midefine\_label}
\begin{verbatim}
NAME

midefine_label - define a label value for a labelled volume.

SYNOPSIS

#include <minc2.h>

int midefine_label ( mihandle_t         volume, 
                     int                value, 
                     const char         *name );

DESCRIPTION

This function associates a label name with an integer value for the given
volume. Functions which read and write voxel values will read/write 
in integer values, and must call miget_label_name() to discover the 
descriptive text string which corresponds to the integer value.

RETURN VALUE

MI_NOERROR on success, or MI_ERROR on failure.
\end{verbatim}

\subsection{miget\_label\_name}
\begin{verbatim}
NAME

miget_label_name - convert a label type to a text string

SYNOPSIS

#include <minc2.h>

int miget_label_name ( mihandle_t       volume, 
                       int              value, 
                       char             **name );

DESCRIPTION

For a labelled volume, this function retrieves the text name
associated with a given integer value.

The name pointer returned must be freed by calling mifree_name().

RETURN VALUE

MI_NOERROR on success, MI_ERROR on failure.
\end{verbatim}

\subsection{miget\_label\_value}
\begin{verbatim}

NAME

miget_label_value - translate a label name into the corresponding integer
value.

SYNOPSIS

#include <minc2.h>

int miget_label_value ( mihandle_t      volume, 
                        const char      *name, 
                        int             *value_ptr );

DESCRIPTION

This function is the inverse of miget_label_name. It is called to determine
what integer value, if any, corresponds to the given text string.

RETURN VALUE

MI_NOERROR on success, MI_ERROR on failure.
\end{verbatim}

\subsection{miget\_number\_of\_defined\_labels}
\begin{verbatim}

NAME

miget_number_of_defined_labels - Retrive the number of defined labels

SYNOPSIS

#include <minc2.h>

int miget_number_of_defined_labels ( mihandle_t      volume, 
                                     int             *number_of_labels );

DESCRIPTION

This function returns the number of defined labels, if any, or zero otherwise.

RETURN VALUE

MI_NOERROR on success, MI_ERROR on failure.
\end{verbatim}

\section{RECORD FUNCTIONS (4)}
\subsection{miget\_record\_name}
\begin{verbatim}
NAME

miget_record_name - get the name of the record

SYNOPSIS

#include <minc2.h>

int miget_record_name ( mihandle_t      volume,
                        char            **name)


DESCRIPTION

This method gets the name of the record dimension.

RETURN VALUE

MI_NOERROR on success, MI_ERROR on failure.
\end{verbatim}

\subsection{miget\_record\_length}
\begin{verbatim}
NAME

miget_record_length - get the length of the record

SYNOPSIS

#include <minc2.h>

int miget_record_length ( mihandle_t    volume,
                          int           *length)


DESCRIPTION

This method gets the length (i.e., number of fields in the case of uniform
records and number of bytes for non_uniform ones) of the record.

RETURN VALUE

MI_NOERROR on success, MI_ERROR on failure.
\end{verbatim}

\subsection{miget\_record\_field\_name}
\begin{verbatim}
NAME

miget_record_field_name - get the record's field name

SYNOPSIS

#include <minc2.h>

int miget_record_field_name ( mihandle_t        volume,
                              int               index,
                              char              **name)

DESCRIPTION

This method sets a field name for the given record.

RETURN VALUE

MI_NOERROR on success, MI_ERROR on failure.
\end{verbatim}

\subsection{miset\_record\_field\_name}
\begin{verbatim}
NAME

miset_record_field_name - set the record's field name

SYNOPSIS

#include <minc2.h>

int miset_record_field_name ( mihandle_t        volume,
                              int               index,
                              char              *name)
                                

DESCRIPTION

This method sets a field name for the given record. e.g. field is "red"


RETURN VALUE

MI_NOERROR on success, MI_ERROR on failure.
\end{verbatim}

\section{SLICE/VOLUME SCALE FUNCTIONS (14)}

\subsection{miget\_slice\_max}
\begin{verbatim}
NAME 

miget_slice_max - get maximum real value of all the slice

SYNOPSIS

#include <minc2.h>

int miget_slice_max ( mihandle_t        volume,
                      unsigned long     start_positions[],
                      unsigned long     array_length,
                      double            *slice_max)
                       
NOTE

All of the slice scale (i.e., one scale per slice) functions take an array
of long integer coordinates to specify the slice in particular and the order 
of the coordinates is always set to FILE ORDER of the dimensions (i.e. not the
apparent order). The number of coordinates to specify is the first (n-2) 
dimensions except in the case of a uniform record which is also flattened
(i.e., flatten_flag=1) that is the first (n-3) dimensions.                                


DESCRIPTION

This method returns the slice_max with the minimum real value of
the corresponding slice. Note that this function is not defined for
floating point data type nor it works if the volume is globally scaled.

RETURN VALUE

miget_slice_max returns MI_NOERROR if it successfully returns the slice_max
or MI_ERROR otherwise
\end{verbatim}

\subsection{miset\_slice\_max}
\begin{verbatim}
NAME 

miset_slice_max - set maximum real value for the slice

SYNOPSIS

#include <minc2.h>

int miset_slice_max ( mihandle_t        volume,
                      unsigned long     start_positions[],
                      unsigned long     array_length,
                      const double      slice_max)
                       
                                
DESCRIPTION

This method sets the slice_max with the minimum real value for
the corresponding slice. Note that this function is not defined for
floating point data type nor it works if the volume is globally scaled.

RETURN VALUE

miset_slice_max returns MI_NOERROR if it successfully sets the slice_max
or MI_ERROR otherwise
\end{verbatim}

\subsection{miget\_slice\_min}
\begin{verbatim}
NAME 

miget_slice_min - get the minimum real value of the slice

SYNOPSIS

#include <minc2.h>

int miget_slice_min ( mihandle_t     volume,
                      unsigned long     start_positions[],
                      unsigned long     array_length,
                      double            *slice_min)
                       
                                
DESCRIPTION

This method returns the slice_min with the minimum real value of
the corresponding slice. Note that this function is not defined 
for floating point data type nor it works if the volume is globally scaled.

RETURN VALUE

miget_slice_min returns MI_NOERROR if it successfully returns the slice_min
or MI_ERROR otherwise
\end{verbatim}

\subsection{miset\_slice\_min}
\begin{verbatim}
NAME 

miset_slice_min - set the minimum real value for all the slice

SYNOPSIS

#include <minc2.h>

int miset_slice_min ( mihandle_t        volume,
                      unsigned long     start_positions[],
                      unsigned long     array_length,
                      const double      slice_min)
                       
                                
DESCRIPTION

This method sets the slice_min with the minimum real value for
the corresponding slice. Note that this function is not defined for
floating point data type nor it works if the volume is globally scaled.

RETURN VALUE

miset_slice_min returns MI_NOERROR if it successfully sets the slice_min
or MI_ERROR otherwise
\end{verbatim}

\subsection{miget\_slice\_range}
\begin{verbatim}
NAME 

miget_slice_range - get the min and max real values of the slice range

SYNOPSIS

#include <minc2.h>

int miget_slice_range ( mihandle_t      volume,
                        unsigned long   start_positions[],
                        unsigned long   array_length,
                        double          *slice_max,
                        double          *slice_min)
                       
                                
DESCRIPTION

This method returns the slice range according to their minimum and maximum
real values . Note that this function is not defined for floating point 
data type nor it works if the volume is globally scaled.

RETURN VALUE

miget_slice_range returns MI_NOERROR if it successfully returns min and
max or MI_ERROR otherwise
\end{verbatim}

\subsection{miset\_slice\_range}
\begin{verbatim}
NAME 

miset_slice_range - set the min and max real values of the slice range

SYNOPSIS

#include <minc2.h>

int miset_slice_range ( mihandle_t      volume,
                        unsigned long   start_positions[],
                        unsigned long   array_length,
                        const double    slice_max,
                        const double    slice_min)
                       
                                
DESCRIPTION

This method sets the slice range according to the minimum and maximum
real values . Note that this function is not defined forfloating point 
data type nor it works if the volume is globally scaled.


RETURN VALUE

miset_slice_range returns MI_NOERROR if it successfully sets the slice_max
and the slice_min or MI_ERROR otherwise
\end{verbatim}

\subsection{miget\_slice\_scaling\_flag}
\begin{verbatim}
NAME 

miget_slice_scaling_flag - get the scaling flag for slices

SYNOPSIS

#include <minc2.h>

int miget_slice_scaling_flag ( mihandle_t   volume,
                               Boolean      *scaling_flag)
                                
DESCRIPTION

This method gets the scaling flag for slices which determines whether 
the volume slices have different scale factors. 

RETURN VALUE

miget_slice_scaling_flag returns MI_NOERROR if it successfully gets 
the scaling flag or MI_ERROR otherwise
\end{verbatim}

\subsection{miset\_slice\_scaling\_flag}
\begin{verbatim}
NAME 

miset_slice_scaling_flag - set the scaling flag for slices

SYNOPSIS

#include <minc2.h>

int miset_slice_scaling_flag ( mihandle_t       volume,
                               Boolean          scaling_flag)
                                
DESCRIPTION

This method sets the scaling flag for slices which determines whether 
the volume slices have different scale factors.

RETURN VALUE

miset_slice_scaling_flag returns MI_NOERROR if it successfully sets 
the scaling flag or MI_ERROR otherwise
\end{verbatim}

\subsection{miget\_volume\_max}
\begin{verbatim}
NAME

miget_volume_max - get the global max scale for volume

SYNOPSIS

#include <minc2.h>

int miget_volume_max ( mihandle_t       volume,
                       double           *volume_max)


DESCRIPTION

This method gets the max scale value for the volume. Note that this 
method only works if volume is not slice scaled.

RETURN VALUE

miget_volume_max returns MI_NOERROR if it successfully gets the
global max scale for the volume or MI_ERROR otherwise. 
\end{verbatim}

\subsection{miset\_volume\_max}
\begin{verbatim}
NAME

miset_volume_max - set the global max scale for volume

SYNOPSIS

#include <minc2.h>

int miset_volume_max ( mihandle_t       volume,
                       double           volume_max)


DESCRIPTION

This method sets the max scale value for the volume. Note that this 
method only works if volume is not slice scaled.

RETURN VALUE

miset_volume_max returns MI_NOERROR if it successfully sets the
global max scale for the volume or MI_ERROR otherwise. 
\end{verbatim}

\subsection{miget\_volume\_min}
\begin{verbatim}
NAME

miget_volume_min - get the global min scale for volume

SYNOPSIS

#include <minc2.h>

int miget_volume_min ( mihandle_t       volume,
                       double           *volume_min)


DESCRIPTION

This method gets the min scale value for the volume. Note that this 
method only works if volume is not slice scaled.

RETURN VALUE

miget_volume_min returns MI_NOERROR if it successfully gets the
global min scale for the volume or MI_ERROR otherwise. 
\end{verbatim}

\subsection{miset\_volume\_min}
\begin{verbatim}
NAME

miset_volume_min -  set the global min scale for volume

SYNOPSIS

#include <minc2.h>

int miset_volume_min ( mihandle_t       volume,
                       double           volume_min)


DESCRIPTION

This method sets the min scale value for the volume. Note that this 
method only works if volume is not slice scaled.

RETURN VALUE

miset_volume_min returns MI_NOERROR if it successfully sets the
global min scale for the volume or MI_ERROR otherwise.
\end{verbatim}

\subsection{miget\_volume\_range}
\begin{verbatim}
NAME

miget_volume_range - get the global range scale for volume

SYNOPSIS

#include <minc2.h>

int miget_volume_range ( mihandle_t     volume,
                         double         *volume_max,
                         double         *volume_min)


DESCRIPTION

This method gets the range (min and max) scale value for the volume. Note that this 
method only works if volume is not slice scaled.

RETURN VALUE

miget_volume_range returns MI_NOERROR if it successfully gets the
global range scale for the volume or MI_ERROR otherwise. 
\end{verbatim}

\subsection{miset\_volume\_range}
\begin{verbatim}
NAME

miset_volume_range - set the global range scale for volume

SYNOPSIS

#include <minc2.h>

int miset_volume_range ( mihandle_t     volume,
                         double         volume_max,
                         double         volume_min)


DESCRIPTION

This method sets the range (min and max) scale value for the volume. Note that this 
method only works if volume is not slice scaled.

RETURN VALUE

miset_volume_range returns MI_NOERROR if it successfully sets the
global range scale for the volume or MI_ERROR otherwise.
\end{verbatim}

\section{VALID MIN/MAX AND RANGE FUNCTIONS (6)}
\subsection{miget\_volume\_valid\_max}
\begin{verbatim}

NAME 

miget_volume_valid_max - get the valid maximum value 

SYNOPSIS

#include <minc2.h>

int miget_volume_valid_max ( mihandle_t        volume,
                             double            *valid_max)
                                
DESCRIPTION

This method gets the valid maximum value specific to the data type.

RETURN VALUE

miget_volume_valid_max returns MI_NOERROR if it successfully gets 
the valid maximum or MI_ERROR otherwise
\end{verbatim}

\subsection{miset\_volume\_valid\_max}
\begin{verbatim}
NAME 

miset_volume_valid_max - set the valid maximum value 

SYNOPSIS

#include <minc2.h>

int miset_volume_valid_max ( mihandle_t        volume,
                             double            valid_max)
                                
DESCRIPTION

This method sets the valid maximum value specific to the data type.

RETURN VALUE

miset_volume_valid_max returns MI_NOERROR if it successfully sets 
the valid maximum or MI_ERROR otherwise
\end{verbatim}

\subsection{miget\_volume\_valid\_min}
\begin{verbatim}
NAME 

miget_volume_valid_min - get the valid minimum value 

SYNOPSIS

#include <minc2.h>

int miget_volume_valid_min ( mihandle_t        volume,
                             double            *valid_min)
                                
DESCRIPTION

This method gets the valid minimum value specific to the data type.

RETURN VALUE

miget_volume_valid_min returns MI_NOERROR if it successfully gets 
the valid minimum or MI_ERROR otherwise
\end{verbatim}

\subsection{miset\_volume\_valid\_min}
\begin{verbatim}
NAME 

miset_volume_valid_min - set the valid minimum value 

SYNOPSIS

#include <minc2.h>

int miset_volume_valid_min ( mihandle_t        volume,
                             double            valid_min)
                                
DESCRIPTION

This method sets the valid minimum value specific to the data type.

RETURN VALUE

miset_volume_valid_min returns MI_NOERROR if it successfully sets 
the valid minimum or MI_ERROR otherwise
\end{verbatim}

\subsection{miget\_volume\_valid\_range}
\begin{verbatim}
NAME 

miget_volume_valid_range - get the valid range values

SYNOPSIS

#include <minc2.h>

int miget_volume_valid_range ( mihandle_t      volume,
                               double          *valid_max,
                               double          *valid_min)
                                
DESCRIPTION

This method gets the valid range values specific to the data type.


RETURN VALUE

miget_volume_valid_range returns MI_NOERROR if it successfully gets 
the valid range values or MI_ERROR otherwise
\end{verbatim}

\subsection{miset\_volume\_valid\_range}
\begin{verbatim}
NAME 

miset_volume_valid_range - set the valid range values

SYNOPSIS

#include <minc2.h>

int miset_volume_valid_range ( mihandle_t      volume,
                               double          valid_max,
                               double          valid_min)
                                
DESCRIPTION

This method sets the valid range values specific to the data type.

RETURN VALUE

miset_volume_valid_range returns MI_NOERROR if it successfully sets 
the valid range values or MI_ERROR otherwise
\end{verbatim}

% THE FOLLOWING IS NO LONGER IN THE CODE (leila)
%\subsection{miset\_valid\_range\_to\_default}
%\begin{verbatim}
%NAME 

%miset_valid_range_to_default - set the range to the default value

%SYNOPSIS

%#include <minc2.h>

%int miset_valid_range_to_default ( mihandle_t   volume)
                        
                                
%DESCRIPTION

%This method sets the valid range to the dafault value specific to the
%data type.

%RETURN VALUE

%miset_valid_range_to_default returns MI_NOERROR if it successfully sets 
%the valid range to default values  or MI_ERROR otherwise
%\end{verbatim}

\section{VOLUME FUNCTIONS (6)}
\subsection{micreate\_volume\_image}
\begin{verbatim}
NAME

micreate_volume_image - 

SYNOPSIS

#include <minc2.h>

int micreate_volume_image ( mihandle_t    volume);

DESCRIPTION

Create the actual image for the volume.

RETURN VALUE

Returns MI_NOERROR on success or MI_ERROR on failure.
\end{verbatim}

\subsection{micreate\_volume}
\begin{verbatim}
NAME

micreate_volume - create a volume with the specified properties

SYNOPSIS

#include <minc2.h>

int micreate_volume ( const char        *filename,
                      int               number_of_dimensions,
                      midimhandle_t     dimensions[],
                      mitype_t          volume_type,
                      miclass_t         volume_class,
                      mivolumeprops_t   create_props,
                      mihandle_t        *volume)

DESCRIPTION

Create a volume with the specified filename, data type, dimension handles,
type, class and compression type.

RETURN VALUE

micreate_volume returns MI_NOERROR if it successfully creates a
volume with all the specified properties and MI_ERROR otherwise
\end{verbatim}

\subsection{miget\_volume\_dimension\_count}
\begin{verbatim}
NAME

miget_volume_dimension_count - get the number of dimensions defined in a MINC
volume, according to their class and attribute.

SYNOPSIS

#include <minc2.h>

int miget_volume_dimension_count ( mihandle_t           volume, 
                                   midimclass_t         class,
                                   midimattr_t          attr,
                                   int                  *dimension\_count );

DESCRIPTION

This function may be used to determine the number of dimensions with the
given class and attributes.

RETURN VALUE

MI_NOERROR if the call succeeds, or MI_ERROR if an error is detected.
\end{verbatim}

\subsection{miopen\_volume}
\begin{verbatim}
NAME

miopen_volume - open a volume for reading

SYNOPSIS

#include <minc2.h>

int miopen_volume(const char    *filename, 
                  int            mode,
                  mihandle_t    *volume);

DESCRIPTION

Opens an existing MINC volume for read-only access if mode argument is
MI2_OPEN_READ, or read-write access if mode argument is MI2_OPEN_RDWR.

RETURN VALUE

Returns MI_NOERROR on success or MI_ERROR on failure.
\end{verbatim}

\subsection{miflush\_volume}
\begin{verbatim}
NAME

miflush_volume 

SYNOPSIS

#include <minc2.h>

int miflush_volume ( mihandle_t    volume);

DESCRIPTION

Writes any changes associated with the volume to disk.

RETURN VALUE

Returns MI_NOERROR on success or MI_ERROR on failure.
\end{verbatim}

\subsection{miclose\_volume}
\begin{verbatim}
NAME

miclose_volume - close an open volume, freeing the volume handle

SYNOPSIS

#include <minc2.h>

int miclose_volume ( mihandle_t    volume);

DESCRIPTION

Close an existing MINC volume. If the volume was newly created, all
changes will be written to disk. In all cases this function closes
the open volume and frees memory associated with the volume handle.

RETURN VALUE

Returns MI_NOERROR on success or MI_ERROR on failure.
\end{verbatim}

\section{VOXEL/REAL FUNCTIONS (16)}
\subsection{miconvert\_real\_to\_voxel, miconvert\_voxel\_to\_real}
\begin{verbatim}
NAME 

miconvert_real_to_voxel, miconvert_voxel_to_real  - conversion between
voxel and real values

SYNOPSIS

#include <minc2.h>


int miconvert_real_to_voxel( mihandle_t                 volume,
                             const unsigned long        location[],
                             int                        array_length,
                             double                     value,
                             double                     *voxel_ptr );

int miconvert_voxel_to_real ( mihandle_t                volume, 
                              const unsigned long       location[],
                              int                       array_length,
                              double                    voxel,
                              double                    *value_ptr );

DESCRIPTION

These functions convert values between real (scaled) values and voxel
(unscaled) values.  The voxel value is the unscaled value, and
corresponds to the value actually stored in the file, whereas the
"real" value is the value at the given location after scaling has been
applied.

RETURN VALUE

Returns MI_NOERROR on success, MI_ERROR on failure.
\end{verbatim}

\subsection{miconvert\_3D\_voxel\_to\_world}
\begin{verbatim}
NAME 

miconvert_3D_voxel_to_world - convert voxel to world coordinates

SYNOPSIS

#include <minc2.h>

int  miconvert_3D_voxel_to_world ( mihandle_t        volume,
                                   const double      voxel[3],
                                   double            world[3])
                                
DESCRIPTION
** Not implemented yet!!
This method makes conversion from voxel to world coordinate system. The
voxels can come in any order but the result in world coordinate system 
would be in (x,y,z) order. Use "miconvert_voxel_to_world" if any number
of spatial dimensions other than three (3) is present.

RETURN VALUE

miconvert_3D_voxel_to_world returns MI_ERROR if there are not exactly
three (3) spatial dimensions present or MI_NOERROR otherwise.
\end{verbatim}

\subsection{miconvert\_3D\_world\_to\_voxel}
\begin{verbatim}
NAME 

miconvert_3D_world_to_voxel - convert world to voxel coordinates

SYNOPSIS

#include <minc2.h>

int  miconvert_3D_world_to_voxel ( mihandle_t        volume,
                                   double            voxel[3],
                                   const double      world[3])
                       
                                
DESCRIPTION
** Not implemented yet!!
This method makes conversion from world to voxel coordinate system. The
dimensions in world coordinate system come in (x,y,z) order but the result  
in voxel coordinates will be in the order of spatial dimensions. 
Use "miconvert_world_to_voxel" if any number of spatial dimensions other
than three (3) is present.


RETURN VALUE

miconvert_3D_world_to_voxel returns MI_ERROR if there are not exactly
three (3) spatial dimensions present or MI_NOERROR otherwise.
\end{verbatim}

\subsection{miconvert\_3D\_voxel\_to\_spatial\_frequency}
\begin{verbatim}
NAME 

miconvert_3D_voxel_to_spatial_frequency

SYNOPSIS

#include <minc2.h>

int  miconvert_3D_voxel_to_spatial_frequency ( mihandle_t            volume,
                                               const double          voxel[3],
                                               double                world[3])
                                   
                       
                                
DESCRIPTION
** Not implemented yet!!

RETURN VALUE

miconvert_3D_voxel_to_spatial_frequency
\end{verbatim}

\subsection{miconvert\_3D\_spatial\_frequency\_to\_voxel}
\begin{verbatim}
NAME 

miconvert_3D_spatial_frequency_to_voxel

SYNOPSIS

#include <minc2.h>

int  miconvert_3D_spatial_frequency_to_voxel ( mihandle_t            volume,
                                               double                voxel[3],
                                               const double          world[3])
                       
                                
DESCRIPTION
** Not implemented yet!!

RETURN VALUE

miconvert_3D_spatial_frequency_to_voxel
\end{verbatim}

\subsection{miconvert\_voxel\_to\_world}
\begin{verbatim}
NAME 

miconvert_voxel_to_world

SYNOPSIS

#include <minc2.h>

int  miconvert_voxel_to_world ( midimhandle_t   dimensions[],
                                int             n_dimensions,
                                const double    voxels[],
                                double          worlds[])
                                   
                       
                                
DESCRIPTION
** Not implemented yet!!

RETURN VALUE

miconvert_voxel_to_world
\end{verbatim}

\subsection{miconvert\_world\_to\_voxel}
\begin{verbatim}
NAME 

miconvert_world_to_voxel

SYNOPSIS

#include <minc2.h>

int  miconvert_world_to_voxel ( midimhandle_t   dimensions[],
                                int             n_dimensions,
                                double          voxels[],
                                const double    worlds[])
                                   
                       
                                
DESCRIPTION
** Not implemented yet!!

RETURN VALUE

miconvert_world_to_voxel
\end{verbatim}

\subsection{miget\_real\_value}
\begin{verbatim}
NAME

miget_real_value -  return a specific scaled value from a volume

SYNOPSIS

#include <minc2.h>



int miget_real_value ( mihandle_t               volume,
                       const unsigned long      location[],
                       int                      array_length,
                       double                   *value_ptr );

DESCRIPTION

This function retrieves the real values of a position in the
MINC volume.  The "real" value is the value at the given location 
after scaling has been applied.

RETURN VALUE

Returns MI_NOERROR on success, MI_ERROR on failure.
\end{verbatim}

\subsection{miset\_real\_value}
\begin{verbatim}
NAME

miset_real_value - set the scaled value of a particular position 
in the MINC volume.

SYNOPSIS

#include <minc2.h>


int miset_real_value( mihandle_t                volume,
                      const unsigned long       location[],
                      int                       array_length,
                      double                    value );

DESCRIPTION

This function sets the  real value of a position in the MINC
volume. The "real" value is the value at the given location 
after scaling has been applied.

RETURN VALUE

Returns MI_NOERROR on success, MI_ERROR on failure.
\end{verbatim}

\subsection{miconvert\_spatial\_origin\_to\_start}
\begin{verbatim}
NAME 

miconvert_spatial_origin_to_start

SYNOPSIS

#include <minc2.h>

int  miconvert_spatial_origin_to_start( mihandle_t           volume,
                                        double               world[3],
                                        double               starts[3])
                       
                                
DESCRIPTION
** Not implemented yet!!
This function calculates the start values for the volume dimensions,
assuming that the spatial origin is relocated to the given world
coordinate.

RETURN VALUE

Returns MI_ERROR or MI_NOERROR

\end{verbatim}

\subsection{miconvert\_spatial\_frequency\_origin\_to\_start}
\begin{verbatim}
NAME 

miconvert_spatial_frequency_origin_to_start

SYNOPSIS

#include <minc2.h>

int  miconvert_spatial_frequency_origin_to_start( mihandle_t         volume,
                                                  const double       world[3],
                                                  double             starts[3])
                       
                                
DESCRIPTION

This function calculates the start values for the volume dimensions,
assuming that the spatial origin is relocated to the given world
coordinate.

RETURN VALUE

Returns MI_ERROR or MI_NOERROR

\end{verbatim}

\subsection{miset\_world\_origin}
\begin{verbatim}
NAME 

miset_world_origin

SYNOPSIS

#include <minc2.h>

int  miset_world_origin( mihandle_t volume,
                         double     world[3])
                       
DESCRIPTION

This function sets the world coordinates of the point (0,0,0) in voxel
coordinates.  This changes the constant offset of the two coordinate
systems.

RETURN VALUE

Returns MI_ERROR or MI_NOERROR

\end{verbatim}

\subsection{miset\_spatial\_frequency\_origin}
\begin{verbatim}
NAME 

miset_spatial_frequency_origin

SYNOPSIS

#include <minc2.h>

int  miset_spatial_frequency_origin( mihandle_t volume,
                                     double     world[3])
                       
DESCRIPTION

This function sets the world coordinates of the point (0,0,0) in voxel
coordinates.  This changes the constant offset of the two coordinate
systems.

RETURN VALUE

Returns MI_ERROR or MI_NOERROR

\end{verbatim}

\subsection{miget\_voxel\_value}
\begin{verbatim}
NAME

miget_voxel_value  - return a specific unscaled value from a given volume

SYNOPSIS

#include <minc2.h>

int miget_voxel_value ( mihandle_t              volume,
                        const unsigned long     location[],
                        int                     array_length,
                        double                  *voxel_ptr );


DESCRIPTION

This function retrieves the real values of a position in the
MINC volume. The voxel value is the unscaled value, and corresponds
to the value actually stored in the file.


RETURN VALUE

Returns MI_NOERROR on success, MI_ERROR on failure.
\end{verbatim}

\subsection{miset\_voxel\_value}
\begin{verbatim}
NAME

miset_voxel_value - set the unscaled value of a particular position
in the MINC volume.

SYNOPSIS

#include <minc2.h>

int miset_voxel_value( mihandle_t               volume,
                       const unsigned long      location[],
                       int                      array_length,
                       double                   voxel );


DESCRIPTION

This function sets the voxel value of a position in the MINC
volume.  The voxel value is the unscaled value, and corresponds to the
value actually stored in the file.

RETURN VALUE

Returns MI_NOERROR on success, MI_ERROR on failure.
\end{verbatim}

\section{VOLUME PROPERTIES FUNCTIONS (15)}
\subsection{minew\_volume\_props/mifree\_volume\_props}
\begin{verbatim}
NAME

minew_volume_props, mifree_volume_props - create or destroy a volume
property list.

SYNOPSIS

#include <minc2.h>

int minew_volume_props(mivolumeprops_t          *props);
int mifree_volume_props(mivolumeprops_t          props);

DESCRIPTION

MINC volume properties objects are used to set or query the state of a number
of the internal parameters of the volume.

RETURN VALUE
\end{verbatim}

\subsection{miget\_volume\_props}
\begin{verbatim}
NAME

miget_volume_props - Get a copy of the property list of a volume.

SYNOPSIS

#include <minc2.h>

int miget_volume_props(mihandle_t               volume, 
                       mivolumeprops_t          *props);

DESCRIPTION

Returns a copy of the properties associated with the volume.  Any
changes made to the properties will not take effect on the original volume,
but the resulting properties structure may be used in the creation of a new
volume.

The handle must be freed by calling mifree_volume_props().

RETURN VALUE

MI_NOERROR on success, MI_ERROR on failure
\end{verbatim}

\subsection{miset\_props\_multi\_resolution/miget\_props\_multi\_resolution}
\begin{verbatim}
NAME

miset_props_multi_resolution, miget_props_multi_resolution - get or set the
multi-resolution properties for a property list.

SYNOPSIS

#include <minc2.h>

int miset_props_multi_resolution(mivolumeprops_t        props, 
                                 BOOLEAN                enable_flag,
                                 int                    depth);
int miget_props_multi_resolution(mivolumeprops_t        props, 
                                 BOOLEAN                *enable_flag,
                                 int                    *depth);

DESCRIPTION

Returns the multi-resolution properties for a property list.

RETURN VALUE

Returns MI_NOERROR on success, MI_ERROR on failure
\end{verbatim}

\subsection{miselect\_resolution}
\begin{verbatim}
NAME 

miselect_resolution - compute a different resolution

SYNOPSIS

#include <minc2.h>

int  miselect_resolution ( mihandle_t        volume,
                           int               depth)
                                   
                       
                                
DESCRIPTION

This method computes a new resolution on according to the depth 
(resolution level) provided by the user. 

RETURN VALUE

miselect_resolution returns MI_NOERROR if it successfully computes a
new resolution from the data or MI_ERROR otherwise.
\end{verbatim}

\subsection{miflush\_from\_resolution}
\begin{verbatim}
NAME 

miflush_from_resolution - compute all resolutions

SYNOPSIS

#include <minc2.h>

int  miflush_from_resolution ( mihandle_t            volume,
                               int                   depth)
                                   
                       
                                
DESCRIPTION

This method sets the the resolution level (depth) to the parameter
provided and computes all the other resolutions.

RETURN VALUE

miflush_from_resolution returns MI_NOERROR if it successfully computes 
all the resolutions from the data or MI_ERROR otherwise.
\end{verbatim}

\subsection{miset\_props\_compression\_type/miget\_props\_compression\_type}
\begin{verbatim}
NAME

miset_props_compression_type, miget_props_compression_type - get or set
the compression type for a volume property list.

SYNOPSIS

#include <minc2.h>

int miset_props_compression_type(mivolumeprops_t        props, 
                                 micompression_t        compression_type);

int miget_props_compression_type(mivolumeprops_t        props,
                                 micompression_t        *compression_type);

DESCRIPTION

Set or retrieve the compression type, if any, for the volume properties.
Currently only two compression types are defined:

Enabling compression will automatically enable blocking with default
parameters (see miset_props_blocking).

typedef enum {
        MI_COMPRESS_NONE = 0,
        MI_COMPRESS_ZLIB = 1
} micompression_t ;

RETURN VALUE

MI_NOERROR on success, or MI_ERROR on failure.
\end{verbatim}

\subsection{miset\_props\_zlib\_compression/miget\_props\_zlib\_compression}
\begin{verbatim}
NAME 

miset_props_zlib_compression, miget_props_zlib_compression - get or
set the zlib compression properties for a volume property list.

SYNOPSIS

#include <minc2.h>

int miset_props_zlib_compression(mivolumeprops_t        props, 
                                 int                    zlib_level);
int miget_props_zlib_compression(mivolumeprops_t        props, 
                                 int                    *zlib_level);

DESCRIPTION

Get or set the Zlib compression level for the volume properties.  The 
compression level is an integer from 1 to 9.

RETURN VALUE

MI_NOERROR on success, or MI_ERROR on failure.
\end{verbatim}

\subsection{miset\_props\_blocking/miget\_props\_blocking}
\begin{verbatim}
NAME

miset_props_blocking, miget_props_blocking - get or set the blocking 
structure properties for a volume property list.

SYNOPSIS

#include <minc2.h>

int miset_props_blocking(mivolumeprops_t        props, 
                         int                    edge_count,
                         const int              *edge_lengths);

int miget_props_blocking(mivolumeprops_t        props, 
                         int                    *edge_count, 
                         int                    *edge_lengths, 
                         int                    max_lengths);

DESCRIPTION

Gets or sets the block-structuring properties of a volume property
list.  If this option is set on a MINC volume, image data will be
stored in a series of 2D or 3D chunks rather than as a simple linear
array.

This option is enabled implicitly whenever compression is enabled - all
compressed volumes must be block-structured.

RETURN VALUE

Returns MI_NOERROR on success, MI_ERROR on failure
\end{verbatim}

\subsection{miset\_props\_record}
\begin{verbatim}
NAME

miset_props_record - set properties of a uniform/nonuniform record dimension

SYNOPSIS

#include <minc2.h>

int miset_props_record ( mivolumeprops_t        props,
                         long                   length,
                         char                   *name)

DESCRIPTION

This method sets the properties of arecord given its name and
size (i.e., number of fields included in the record).

RETURN VALUE

miset_props_record returns MI_NOERROR if it successfully sets
the properties of a record or MI_ERROR otherwise.
\end{verbatim}

\subsection{miset\_props\_template}
\begin{verbatim}
NAME

miset_props_template - set the template volume flag

SYNOPSIS

#include <minc2.h>

int miset_props_template ( mivolumeprops_t              props,
                           int                          template_flag)          

DESCRIPTION

This method sets the template volume flag to (1) to create a template volume
(i.e, which is exaclty like an image except the image itself) or (0) for
no template volume. The default flag is set to zero.

RETURN VALUE

miset_props_template returns MI_NOERROR if it successfully sets the template
flag or MI_ERROR otherwise.
\end{verbatim}

%\subsection{miset\_multi\_resolution\_method}
%\begin{verbatim}
%NAME

%miset_multi_resolution_method - set the multi-resolution calculation algorithm

%SYNOPSIS

%#include <minc2.h>

%int miset_multi_resolution_method()
%{
%}

%\end{verbatim}


\end{document}
